\documentclass[11pt]{article}
\usepackage{amsmath}  % Math
\usepackage{amssymb}  % Symbols
\usepackage{graphicx} % Images
\usepackage[utf8]{inputenc}
\usepackage[T1]{fontenc}
\usepackage[margin=1in]{geometry}
\usepackage[spanish]{babel} % Spanish language support

\title{Preguntas Discursos Presidenciales}
\author{Felipe Colli}
\date{\today}

\begin{document}

\maketitle
\newpage
\begin{enumerate}
    \item Lee atentamente los cinso trozos de discursos presidenciales. ¿Qué expresan sobre la participación ciudadana, la equidad social y el bien común? \\

    \item Escoje uno de los discursos e investiga sobre el periodo de gobierno al cual corresponde. ¿Cuáles eran los principales desfiíos para la democracia y la participación?
\end{enumerate}

\section*{Actividad 1: Análisis de los Cinco Trozos de Discursos Presidenciales}
\label{sec:actividad1}

Los cinco fragmentos de discursos presidenciales ofrecen distintas perspectivas y énfasis sobre la participación ciudadana, la equidad social y el bien común, reflejando tanto las prioridades de cada administración como el contexto histórico en el que fueron pronunciados.

\subsection*{Sebastián Piñera Echenique (Discurso de 2018)}
\begin{itemize}
    \item \textbf{Participación Ciudadana:} La concibe principalmente como el ejercicio de la libertad individual y la iniciativa de la sociedad civil ("individuos libres y responsables", "vigorosa sociedad civil"). El rol del Estado es "facilitar las condiciones para que esa libertad se despliegue", permitiendo que los ciudadanos sean "arquitectos de nuestras propias vidas". Es una visión donde la participación emana de la libertad individual y la capacidad de la sociedad para organizarse y crear.
    \item \textbf{Equidad Social:} Se enfoca en la "superación de la pobreza e igualdad de oportunidades". No se menciona explícitamente la redistribución, sino la creación de un piso mínimo y condiciones para que todos puedan desarrollar su potencial.
    \item \textbf{Bien Común:} Lo vincula al "progreso de una nación" que depende del esfuerzo individual y colectivo. El bien común se alcanza cuando todos pueden "gozar plenamente de ella [la libertad]" y desarrollar su potencial.
\end{itemize}

\subsection*{Ricardo Lagos Escobar (Discurso de 2005)}
\begin{itemize}
    \item \textbf{Participación Ciudadana:} Aunque no se menciona directamente, se infiere que la participación ciudadana se canaliza a través del apoyo a una "vocación política" orientada a la igualdad y la implementación de reformas (Educacional, Salud, Judicial, Laboral). La participación se traduce en el respaldo a un proyecto político que busca cambios estructurales.
    \item \textbf{Equidad Social:} Es un eje central. Afirma que "el crecimiento sólo tiene sentido si se transforma en mayor bienestar para todos los chilenos" y que "producir equidad y justicia social requiere de voluntad política". La equidad se busca activamente a través de reformas y una distribución de los frutos del crecimiento.
    \item \textbf{Bien Común:} Se define como el "mayor bienestar para todos los chilenos", logrado cuando los beneficios del desarrollo económico se distribuyen equitativamente entre "todas las familias chilenas".
\end{itemize}

\subsection*{Michelle Bachelet Jeria (Discurso de fines de 2017)}
\begin{itemize}
    \item \textbf{Participación Ciudadana:} Se destaca el fortalecimiento de la democracia a través de reformas que aumentan el "pluralismo, inclusión y competencia" en el sistema político. Menciona el voto de chilenos en el extranjero, el nuevo sistema proporcional y la ley de cuotas como mecanismos que amplían y mejoran la participación y representación.
    \item \textbf{Equidad Social:} En este fragmento, la equidad se enfoca principalmente en la representación política: "más pluralismo, inclusión", y que "más mujeres estén en la primera línea de las responsabilidades". Se busca una mayor equidad en el acceso al poder y la toma de decisiones.
    \item \textbf{Bien Común:} Se expresa como el fortalecimiento y madurez de la democracia, un logro que "nos debe alegrar a todos por igual". Un sistema democrático más representativo e inclusivo es visto como un bien para toda la sociedad.
\end{itemize}

\subsection*{Patricio Aylwin Azócar (Discurso de 1990)}
\begin{itemize}
    \item \textbf{Participación Ciudadana:} Es fundamental y explícita en un contexto de retorno a la democracia. Llama a "abrir cauces de participación democrática para que todos colaboren en la consecución del bien común", involucrando a todos los sectores ("trabajadores o empresarios, obreros o intelectuales"). La participación es vista como una colaboración activa en la reconstrucción nacional.
    \item \textbf{Equidad Social:} Se plantea como una tarea urgente "acortar las agudas desigualdades que nos dividen y, muy especialmente, elevar a niveles dignos y humanos la condición de vida de los sectores más pobres". Hay un fuerte énfasis en la justicia social y la reducción de la brecha socioeconómica.
    \item \textbf{Bien Común:} Se define como la "unidad de la familia chilena", el "respeto y confianza en la convivencia", y la superación de las divisiones. El bien común es la reconstrucción de una sociedad cohesionada y democrática tras la dictadura.
\end{itemize}

\subsection*{Eduardo Frei Ruiz-Tagle (Discurso de 1999)}
\begin{itemize}
    \item \textbf{Participación Ciudadana:} No se menciona explícitamente. El foco está en los logros y las bases sentadas por el gobierno. Se podría inferir que la participación se da en un contexto donde el Estado provee mejores condiciones de vida y oportunidades.
    \item \textbf{Equidad Social:} Destaca los avances en un "camino de equidad", mediante la "reducción drástica de los niveles de pobreza" y grandes inversiones en salud, vivienda, educación, justicia e infraestructura. La equidad se mide por la mejora en el acceso a servicios y la disminución de la pobreza.
    \item \textbf{Bien Común:} Se asocia al "período más próspero de la economía chilena", al establecimiento de "bases sólidas para el éxito futuro" y a un "cambio sustancial en su calidad de vida" para Chile. El progreso económico y la mejora en la calidad de vida son vistos como el bien común alcanzado.
\end{itemize}

\subsubsection*{Síntesis de la Actividad 1}
En síntesis, todos los discursos abordan estos tres conceptos, pero con matices que reflejan sus ideologías y los desafíos particulares de sus respectivos períodos. La participación ciudadana varía desde la iniciativa individual (Piñera) hasta la colaboración en la reconstrucción democrática (Aylwin) o la representación política (Bachelet). La equidad social se enfoca en la igualdad de oportunidades (Piñera), la distribución del crecimiento (Lagos), la reducción de la pobreza (Frei, Aylwin) o la equidad en la representación (Bachelet). El bien común se asocia al progreso, la unidad nacional, el bienestar generalizado o el fortalecimiento democrático, dependiendo del mandatario.

\clearpage % Nueva página para la Actividad 2

\section*{Actividad 2: Análisis del Período de Gobierno de Michelle Bachelet (2014-2018)}
\label{sec:actividad2}

El discurso seleccionado para un análisis más profundo es el de Michelle Bachelet Jeria, correspondiente al fin de año de 2017. Este discurso se enmarca en su \textbf{segundo período presidencial (2014-2018)}.

\subsection*{Investigación sobre el Período de Gobierno (2014-2018)}

El segundo mandato de Michelle Bachelet se caracterizó por una agenda de reformas estructurales ambiciosas, orientadas a disminuir la desigualdad y modernizar el país. Entre las principales iniciativas se encontraron:

\begin{itemize}
    \item \textbf{Reforma Educacional:} Buscaba establecer la gratuidad progresiva en la educación superior, el fin al lucro, la selección y el copago en establecimientos que recibían fondos públicos.
    \item \textbf{Reforma Tributaria:} Orientada a aumentar la recaudación fiscal para financiar, principalmente, la reforma educacional y otras políticas sociales.
    \item \textbf{Reforma Laboral:} Introdujo cambios en la negociación colectiva y el derecho a huelga.
    \item \textbf{Cambio al Sistema Electoral Binominal:} Se reemplazó por un sistema proporcional moderado, que además incluyó una ley de cuotas para promover la participación de mujeres en el Congreso (mencionado en su discurso).
    \item \textbf{Inicio del Proceso Constituyente:} Se realizaron cabildos ciudadanos para recoger propuestas para una nueva Constitución, aunque el proceso no culminó durante su mandato.
    \item \textbf{Despenalización de la Interrupción Voluntaria del Embarazo en Tres Causales.}
\end{itemize}

\subsection*{Principales Desafíos para la Democracia y la Participación}

Durante el período 2014-2018, la democracia chilena y la participación ciudadana enfrentaron desafíos significativos:

\begin{enumerate}[label=\arabic*., wide, labelwidth=!, labelindent=0pt]
    \item \textbf{Crisis de Confianza y Casos de Corrupción:}
    Este fue uno de los desafíos más serios. La revelación de casos de financiamiento irregular de la política (casos Penta y SQM) y el Caso Caval (que involucró a su hijo y nuera) golpearon fuertemente la confianza ciudadana en las instituciones políticas, los partidos y los liderazgos. Esto generó un clima de desafección y escepticismo que erosionó la legitimidad democrática y desincentivó la participación electoral tradicional.

    \item \textbf{Implementación de Reformas y Polarización Política:}
    La ambiciosa agenda de reformas generó una fuerte polarización política y social. La discusión e implementación de cambios estructurales profundos (como la reforma educacional o el inicio del proceso constituyente) dividieron opiniones y pusieron a prueba la capacidad del sistema democrático para procesar disensos y alcanzar acuerdos. El desafío era llevar adelante transformaciones significativas manteniendo la cohesión social y el respeto por las reglas democráticas.

    \item \textbf{Aumento de la Participación Ciudadana No Convencional y Canalización de Demandas:}
    Si bien la participación electoral tendía a la baja, hubo un aumento de la movilización social y formas de participación no electoral (protestas, movimientos sociales por la educación, pensiones --"No+AFP"--, medio ambiente, derechos de las mujeres). El desafío para la democracia era cómo canalizar estas demandas de manera efectiva dentro del sistema institucional, abriendo espacios para que la voz ciudadana fuera escuchada e influyera en las políticas públicas, más allá de la protesta. El proceso constituyente con cabildos fue un intento en esta dirección.

    \item \textbf{Fortalecimiento de la Representatividad y el Nuevo Sistema Electoral:}
    El cambio del sistema electoral binominal a uno proporcional buscaba mejorar la representatividad y permitir la entrada de nuevas fuerzas políticas, respondiendo a una demanda de mayor pluralismo. El desafío era que este nuevo sistema efectivamente contribuyera a una mejor democracia, fomentara la participación y no llevara a una fragmentación excesiva que dificultara la gobernabilidad. La Ley de Cuotas, mencionada por Bachelet, apuntaba a mejorar la participación y representación de las mujeres, un desafío democrático clave.

    \item \textbf{Gestión de Expectativas y Descontento Social:}
    Las reformas generaron altas expectativas en parte de la población. El desafío para el gobierno y el sistema democrático fue gestionar estas expectativas, comunicar adecuadamente los alcances y limitaciones de los cambios, y enfrentar el descontento cuando los resultados no eran percibidos como suficientes o inmediatos. Mantener la legitimidad del proceso democrático en este contexto fue crucial.
\end{enumerate}

En resumen, el segundo gobierno de Bachelet se enfrentó al reto de impulsar transformaciones profundas en un contexto de crisis de confianza y alta movilización social, buscando fortalecer la democracia y ampliar los canales de participación ciudadana, aunque con resultados y percepciones diversas.
\end{document}
