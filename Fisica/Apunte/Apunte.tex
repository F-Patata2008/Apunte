\documentclass[11pt]{article}
\usepackage{amsmath}  % Math
\usepackage{amssymb}  % Symbols
\usepackage{graphicx} % Images
\usepackage[utf8]{inputenc}
\usepackage[T1]{fontenc}
\usepackage[margin=1in]{geometry}
\usepackage{transparent}
\usepackage{eso-pic}
\usepackage{xcolor}

\graphicspath{{/home/fpatata/Clases/Fisica/images/}} % Path to images
\newcommand\BackgroundPic{
    \put(0,0){
        \parbox[b][\paperheight]{\paperwidth}{
            \vfill
            \centering
            \transparent{0.2}\includegraphics[width=\paperwidth]{logo.png} % your image
            \vfill
        }
    }
}



\title{Apuntes Fisica Teorica (AFTIN)}
\author{Felipe Colli \thanks{AFTIN y Profesor Paul Cáceres}}
\date{2025}
\AddToShipoutPicture{\BackgroundPic}


\begin{document}
\maketitle
\tableofcontents
\newpage

\section{Clase de 23/05/2025}
    \subsection{Cinemática}
        \subsubsection{Movimiento Rectilíneo Uniforme (MRU)}
            \begin{enumerate}
                \item Poseen una trayectoria rectilinea
                \item Velocidad constante ($\vec{a}=0\frac{m}{s^2}$)
            \end{enumerate}
            \begin{itemize}    
                \item $v = \frac{\Delta x}{\Delta t}$ $\vec{v}=\frac{\vec{d}}{t}$ $(\frac{m}{s})$ $|\vec{v}|=v$
                \item $x(t) = x_i + vt$
                \item $a = 0$
                \item $|\vec{d}|=d$
            \end{itemize}

        \subsubsection{Movimiento Rectilíneo Uniformemente Acelerado (MRUA)}
            \begin{enumerate}
                \item Poseen una trayectoria rectilinea
                \item Velocidad variable $\vec{a}=\frac{\Delta v}{\Delta t}$ $(\frac{m}{s^2})$ 
                \item Si la aceleración es del mismo signo que la velocidad, el objeto se acelera. Si la aceleración es del signo opuesto a la velocidad, el objeto desacelera.
            \end{enumerate}
            \begin{itemize}
                \item Ecuancion de la Velocidad en Función del Tiempo $v = v_i + a\Delta t$
                \item Ecuación Iterinerario $x(t) = x_i + v_i t + \frac{1}{2}at^2$
                \item Ecuacion Independiente del Tiempo $v_f^2 = v_i^2 + 2a(x_f-x_i)$ 
                \item $|\vec{d}|=d$
            \end{itemize}

        \subsubsection{Graficos}
        \begin{enumerate}
            \item Pendiente (Derivada) de la función
            \item Área bajo la curva (Integral) de la función
        \end{enumerate}
            \begin{itemize}
                \item $x$ vs $t$ $\rightarrow$ Pendiente = Velocidad, Area bajo la curva = Velocidad
                \item $v$ vs $t$ $\rightarrow$ $\Delta{x}=\frac{\Delta{v} \cdot t}{2}+v_i t$ Pendiente = Aceleración, Area bajo la curva = Distancia Recorrida
            \end{itemize}

        \subsubsection{Aplicando la Derivada a la ecuación de la Itininerario}
            \[x_f=x_i+v_i t+\frac{1}{2}at^2 \]
            \[v(t)=x'=v_1+at^2 \]
            \[v(t)'=a \]

            


\end{document}
