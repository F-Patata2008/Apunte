\documentclass[11pt]{article}
\usepackage{amsmath}  % Math
\usepackage{amssymb}  % Symbols
\usepackage{graphicx} % Images
\usepackage[utf8]{inputenc}
\usepackage[T1]{fontenc}
\usepackage[margin=1in]{geometry}
\usepackage{transparent}
\usepackage{eso-pic}
\usepackage{xcolor}
\usepackage{fancyhdr} % Headers and footers
\usepackage{lipsum} % Dummy text
\usepackage{tikz} % For drawing
\usepackage[spanish]{babel} % For custom section titles

\graphicspath{{/home/F-Patata/Apunte/Fisica/images/}} % Path to images
\newcommand\BackgroundPic{
    \put(0,0){
        \parbox[b][\paperheight]{\paperwidth}{
            \vfill
            \centering
            \transparent{0.1}\includegraphics[width=\paperwidth]{logo.png} % your image
            \vfill
        }
    }
}



\title{Apuntes Fisica Teorica (AFTIN)}
\author{Felipe Colli \thanks{AFTIN y Profesor Paul Cáceres}}
\date{2025}
\AddToShipoutPicture{\BackgroundPic}


\begin{document}
\maketitle
\tableofcontents
\newpage

\section{Clase de 23/05/2025}
\subsection{Cinemática}
\subsubsection{Movimiento Rectilíneo Uniforme (MRU)}
\begin{enumerate}
	\item Poseen una trayectoria rectilinea
	\item Velocidad constante ($\vec{a}=0\frac{m}{s^2}$)
\end{enumerate}
\begin{itemize}
	\item $v = \frac{\Delta x}{\Delta t}$ $\vec{v}=\frac{\vec{d}}{t}$ $(\frac{m}{s})$ $|\vec{v}|=v$
	\item $x(t) = x_i + vt$
	\item $a = 0$
	\item $|\vec{d}|=d$
\end{itemize}

\subsubsection{Movimiento Rectilíneo Uniformemente Acelerado (MRUA)}
\begin{enumerate}
	\item Poseen una trayectoria rectilinea
	\item Velocidad variable $\vec{a}=\frac{\Delta v}{\Delta t}$ $(\frac{m}{s^2})$
	\item Si la aceleración es del mismo signo que la velocidad, el objeto se acelera. Si la aceleración es del signo opuesto a la velocidad, el objeto desacelera.
\end{enumerate}
\begin{itemize}
	\item Ecuancion de la Velocidad en Función del Tiempo $v = v_i + a\Delta t$
	\item Ecuación Iterinerario $x(t) = x_i + v_i t + \frac{1}{2}at^2$
	\item Ecuacion Independiente del Tiempo $v_f^2 = v_i^2 + 2a(x_f-x_i)$
	\item $|\vec{d}|=d$
\end{itemize}

\subsubsection{Graficos}
\begin{enumerate}
	\item Pendiente (Derivada) de la función
	\item Área bajo la curva (Integral) de la función
\end{enumerate}
\begin{itemize}
	\item $x$ vs $t$ $\rightarrow$ Pendiente = Velocidad, Area bajo la curva = Velocidad
	\item $v$ vs $t$ $\rightarrow$ $\Delta{x}=\frac{\Delta{v} \cdot t}{2}+v_i t$ Pendiente = Aceleración, Area bajo la curva = Distancia Recorrida
\end{itemize}

\subsubsection{Aplicando la Derivada a la ecuación de la Itininerario}
\[x_f=x_i+v_i t+\frac{1}{2}at^2 \]
\[v(t)=x'=v_1+at^2 \]
\[v(t)'=a \]




\section{Clase del 30/05/2025}
\subsection{Movimiento Circunferencial Uniforme (MCU)}
\begin{itemize}
	\item Movimiento circular con rapidez constante
	\item Trayectoria circular
	\item Aceleración centrípeta $a_c = \frac{v^2}{r}$ $(\frac{m}{s^2})$
	\item Velocidad angular $\omega = \frac{\Delta \theta}{\Delta t}$ $(\frac{rad}{s})$
	\item Relación entre velocidad lineal y angular $v = r\omega$ $(\frac{m}{s})$
\end{itemize}
\subsubsection{Magnitudes Fisicas Temporales asociadas al Movimiento Ciclicos}
\begin{itemize}
	\item Periodo $T = \frac{2\pi r}{v} = \frac{t}{N} = \frac{1}{f} [s]$ \\
	      Tiempo que tarda en completar una vuelta
	\item Frecuencia $f = \frac{1}{T} = \frac{N}{t} [Hz]$
	\item Longitud de onda $\lambda = \frac{v}{f} = \frac{vT}{N}$
\end{itemize}

\subsubsection{Estudio lineal del M.C.U.}
\begin{itemize}
	\item Rapidez tangencial $v = \frac{2\pi r}{T} = 2\pi rf $
	\item Aceleración centrípeta $a_c = \frac{v^2}{r} $
	\item Aceleración tangencial $a_t = r\alpha$ donde $\alpha$ es la aceleración angular
	\item Aceleración total $a = \sqrt{a_c^2 + a_t^2}$
\end{itemize}

\subsubsection{Estudio Angular del M.C.U.}
\begin{itemize}
	\item $1 rad \approx 57.3° $
	\item $360° = 2\pi rad$
	\item Rapidez angular $\omega = \frac{2\pi}{T} = 2\pi f$
	\item Aceleración angular $\alpha = \frac{\Delta \omega}{\Delta t}$
	\item Relación entre aceleración centrípeta y angular $a_c = r\omega^2$
	\item Relación entre aceleración tangencial y angular $a_t = r\alpha$
	\item Aceleración total $a = \sqrt{a_c^2 + a_t^2} $
\end{itemize}

\begin{tikzpicture}
	\filldraw[color=red, fill=white, very thick](-1,0) circle (1.5);
\end{tikzpicture}

\newpage


\section{Clase 22/08/2025 - Mmvimiento Circunferencial Uniformemente Variado (MRUV)}
\subsection{Velocidad Angular media ($\vec{\omega_m} $)}

\begin{minipage}{0.45\textwidth}
	\begin{flushleft}
        \[\vec{\omega_m}= \frac{\Delta \vec{\theta}}{\Delta t} (\frac{rad}{s}) \]
	\end{flushleft}
\end{minipage}
\begin{minipage}{0.45\textwidth}
	\begin{flushleft}
		%Insertar imgane de w>0 y w <0 para el sentido de giro

	\end{flushleft}
\end{minipage}

\subsection{Velocidad Angular Instantanea}
\[\vec{\omega} = \lim_{{\Delta}t \to 0} \vec{\omega_m} = \frac{\Delta \vec{\theta}}{\Delta t} \]

\subsection{Aceleracion Angular Media ($\vec{\alpha_m} $) (Constante)}
\[\vec{\alpha_m} = \frac{\Delta \vec{\omega}}{\Delta t} (\frac{ras}{s^2})\]

\subsection{Aceleracion Angular Instantanea}
\[\vec{\alpha}= \lim_{\Delta t \to 0} \frac{\Delta \vec{\omega}}{\Delta t} \]

\subsection{Analisis Gráfico} 
\subsubsection{$\alpha = f(t)$}
%Inssertar grafico generico de acleacion vs tiempo a la izquiera, a la deracha las formulas con minipage
\[A = \alpha \cdot t = \Delta \vec{\omega} \]

\subsubsection{$\omega = f(t)$}
%Insertar grafico gnerecio de \omega vs tiempo a la izquirda, y ala derecha la formula con minipage
\[\omega(t) = \omega_i + \alpha t \]

\[A_1 = \frac{1}{2} \Delta\omega t \]
\[A_2 = \omega_i t \]

\[A = \Delta\theta = A_1 + A_2 = \omega_i t + \frac{1}{2}\omega_i t \] 
\[\theta_f - \theta_i = \omega_i t + \frac{1}{2}\alpha t^2 \]
\[\theta(t) = \theta_i + \omega_i t + \frac{1}{2}\alpha t^2 \]

\subsection{Cantidades Lineales}
%Insertar imagen de acelarcion tangencial en distintos moentos (t1 y t2)

\subsubsection{Velocidad Lineal Instantanea}
\[v = r\omega \]

\subsubsection{Aceleracion Tangencial}
\[A_t = \frac{\Delta v}{\Delta t} = \frac{r\Delta\omega}{\Delta t} = \alpha \cdot r \]

\subsubsection{Aceleracion Centripeta}
\[A_c = \omega^2 r \]

\[\vec{A} = \vec{A_c} + \vec{A_t} \]

\subsubsection{Aceleracion Neta}
\[|\vec{A}| = \sqrt{A_c^2+ A_t^2} \]

\subsection{Problemas}
\begin{enumerate}
    \item Una cuerda de 50 cm de diamtreo trada 10 segundos en adquirir una rapidez constante de 180 pi rad/s, calcula:
        \begin{enumerate}
            \item la acelarcion angular\\
                $\alpha = \frac{\Delta\omega}{\Delta t} = \frac{\omega_f - (\omega_i =  0)}{10} = \frac{180\pi}{10} = 18 (\frac{rad}{s^2}) $\\

            \item Cual es la rapidez lineal en un putno de la periferia cuando alcanza esa rapidez angular\\
                $v = \omega r = 180\pi \cdot 0.5 = 90\pi (\frac{rad}{s}) $\\

            \item LA acelaraciopn centripeta a laos 5 degundo de inciado el mov\\
                $A_c = \omega^2 r = (90\pi)^2 \cdot 0.5 = 8100 \pi^2 \cdot 0.5 = 4050\pi^2 (\frac{rad}{s^2}) $
        \end{enumerate}
\end{enumerate}

\end{document}
