\documentclass[11pt, a4paper]{article}

% --- Paquetes Esenciales ---
\usepackage[utf8]{inputenc} % Codificación de entrada UTF-8 para acentos y ñ
\usepackage[T1]{fontenc}    % Codificación de fuente moderna para mejor separación silábica y caracteres
\usepackage{geometry}       % Para ajustar márgenes si es necesario
\usepackage{amsmath}        % Mejoras para entorno matemático (aunque no se usa mucho aquí, es bueno tenerlo)
\usepackage{amssymb}        % Símbolos matemáticos
\usepackage{parskip}        % Párrafos separados por espacio vertical en lugar de sangría

% --- Configuración de Geometría (opcional, ajusta márgenes) ---
\geometry{
 a4paper,
 total={170mm,257mm},
 left=20mm,
 top=20mm,
 right=20mm,
 bottom=20mm,
}

% --- Título del Documento ---
\title{Léxico Contextual: \textit{1984} (George Orwell)}
\author{Resolución de Ejercicios} % Puedes poner tu nombre aquí o eliminar la línea
\date{\today} % Fecha actual, o puedes poner una fecha específica o eliminar la línea

% --- Inicio del Documento ---
\begin{document}

\maketitle % Muestra el título, autor y fecha


\begin{enumerate} % Lista numerada principal

    \item \textbf{contradicción}
    \begin{description}
        \item[a) Significado:] Oposición entre dos o más proposiciones, ideas o actitudes. En el contexto, se refiere a la aceptación simultánea de ideas opuestas.
        \item[b) Sinónimos:] Incoherencia, oposición, incompatibilidad.
        \item[c) Oración:] "El Partido fomentaba la \textit{incoherencia} deliberada a través del doblepensar, una habilidad que permitía a los ciudadanos aceptar dos ideas opuestas sin cuestionar su validez, con el fin de someter su capacidad de raciocinio a un control absoluto."
    \end{description}

    \item \textbf{subversivo}
    \begin{description}
        \item[a) Significado:] Que intenta alterar o destruir el orden social o político establecido.
        \item[b) Sinónimos:] Rebelde, sedicioso, insurgente.
        \item[c) Oración:] "La constante manipulación del lenguaje a través de la neolingua tenía como objetivo erradicar cualquier forma de pensamiento \textit{rebelde}, eliminando las palabras que pudieran expresar disidencia o libertad."
    \end{description}

    \item \textbf{redefinir}
    \begin{description}
        \item[a) Significado:] Volver a definir algo, dándole un nuevo significado o delimitación.
        \item[b) Sinónimos:] Reformular, reinterpretar, replantear.
        \item[c) Oración:] "La mentira oficial, presentada como verdad, era una de las herramientas más poderosas del Partido, ya que al \textit{reformular} la realidad, eliminaba toda posibilidad de que la población pudiera cuestionar el sistema establecido."
    \end{description}

    \item \textbf{abstractos}
    \begin{description}
        \item[a) Significado:] Que existe solo en idea, en concepto, o en la mente, sin corresponder a una realidad material. No concreto.
        \item[b) Sinónimos:] Teóricos, conceptuales, inmateriales.
        \item[c) Oración:] "En la sociedad totalitaria de 1984, la libertad y la esclavitud no solo se entrelazaban como conceptos \textit{teóricos}, sino que el Partido utilizaba la confusión conceptual para reforzar su hegemonía sobre la mente humana."
    \end{description}

    \item \textbf{hegemonía}
    \begin{description}
        \item[a) Significado:] Supremacía o dominio que un grupo, estado o entidad ejerce sobre otros. En este caso, dominio sobre la mente.
        \item[b) Sinónimos:] Dominio, supremacía, predominio.
        \item[c) Oración:] "...el Partido utilizaba la confusión conceptual para reforzar su \textit{dominio} sobre la mente humana."
    \end{description}

    \item \textbf{cognitivo}
    \begin{description}
        \item[a) Significado:] Perteneciente o relativo al conocimiento y al proceso mental de percepción y procesamiento de la información.
        \item[b) Sinónimos:] Mental, intelectual, del conocimiento.
        \item[c) Oración:] "La reescritura de la historia, realizada por el Ministerio de la Verdad, era una estrategia de control \textit{mental}; el Partido se aseguraba de que la realidad fuera ajustada de acuerdo con sus necesidades, eliminando cualquier evidencia que contradijera su dominio."
    \end{description}

    \item \textbf{subconsciente}
    \begin{description}
        \item[a) Significado:] Nivel de la mente que almacena información o procesos mentales que no están en el foco de la conciencia, pero influyen en el comportamiento.
        \item[b) Sinónimos:] Inconsciente, interior, mente profunda.
        \item[c) Oración:] "...se infiltraba en el \textit{inconsciente} de la población, utilizando la mentira como una forma de imponer la obediencia absoluta."
    \end{description}

    \item \textbf{omnipresente}
    \begin{description}
        \item[a) Significado:] Que está presente en todas partes al mismo tiempo.
        \item[b) Sinónimos:] Ubicuo, universal, constante.
        \item[c) Oración:] "El Gran Hermano representaba una figura de control \textit{ubicua}, cuyo rostro se encontraba en todos los rincones del Estado, desde las carteles hasta los telescreens, lo que aseguraba que el miedo y la devoción a su imagen fueran perpetuos."
    \end{description}
    
    \item \textbf{devoción}
    \begin{description}
        \item[a) Significado:] Sentimiento de profundo respeto, admiración y amor hacia una persona, deidad o ideal, que impulsa a venerarla o servirla.
        \item[b) Sinónimos:] Veneración, fervor, adhesión.
        \item[c) Oración:] "...lo que aseguraba que el miedo y la \textit{veneración} a su imagen fueran perpetuos."
    \end{description}

    \item \textbf{opresión}
    \begin{description}
        \item[a) Significado:] Acción y efecto de oprimir; someter a una persona, nación o grupo mediante el uso de la autoridad, la violencia u otras formas de abuso.
        \item[b) Sinónimos:] Tiranía, sometimiento, subyugación.
        \item[c) Oración:] "...para justificar la continua \textit{tiranía} y el despojo de la autonomía individual."
    \end{description}
    
    \item \textbf{despojo}
    \begin{description}
        \item[a) Significado:] Acción y efecto de quitar a alguien algo que posee o de lo que disfruta, a menudo con violencia o injusticia.
        \item[b) Sinónimos:] Privación, arrebato, supresión.
        \item[c) Oración:] "...para justificar la continua opresión y la \textit{privación} de la autonomía individual."
    \end{description}

    \item \textbf{intrusiva}
    \begin{description}
        \item[a) Significado:] Que se introduce o entromete en asuntos ajenos o en la privacidad sin derecho o permiso.
        \item[b) Sinónimos:] Invasiva, entrometida, indiscreta.
        \item[c) Oración:] "La vigilancia constante era tan \textit{invasiva} que los ciudadanos no solo temían ser observados, sino que habían internalizado el sistema de control al punto de que autocensurarse se había convertido en un hábito casi involuntario."
    \end{description}

    \item \textbf{coerción}
    \begin{description}
        \item[a) Significado:] Presión ejercida sobre alguien para forzar su voluntad o su conducta, a menudo mediante el uso de la fuerza o amenazas.
        \item[b) Sinónimos:] Compulsión, fuerza, intimidación.
        \item[c) Oración:] "La tortura psicológica empleada en el Ministerio del Amor no solo tenía como objetivo la \textit{compulsión} física, sino también la reestructuración de la voluntad humana, destruyendo la capacidad de pensar de forma autónoma."
    \end{description}

    \item \textbf{coherencia}
    \begin{description}
        \item[a) Significado:] Conexión lógica y relación adecuada entre las distintas partes de un todo; cualidad de lo que tiene sentido y no presenta contradicciones internas.
        \item[b) Sinónimos:] Consistencia, lógica, conexión.
        \item[c) Oración:] "Los ciudadanos se veían atrapados en un ciclo de desinformación perpetua, donde el Ministerio de la Verdad manipulaba los hechos para ajustarlos a la narrativa oficial, destruyendo la \textit{consistencia} de la experiencia humana."
    \end{description}

    \item \textbf{reafirmación}
    \begin{description}
        \item[a) Significado:] Acción y efecto de volver a afirmar o confirmar algo para darle más fuerza o seguridad.
        \item[b) Sinónimos:] Confirmación, ratificación, corroboración.
        \item[c) Oración:] "...asegurar la unidad de la población a través del miedo y la \textit{confirmación} constante de la superioridad del régimen."
    \end{description}

    \item \textbf{subversivos} % Nota: es la misma palabra que el punto 2
    \begin{description}
        \item[a) Significado:] Que intentan alterar o destruir el orden social o político establecido.
        \item[b) Sinónimos:] Rebeldes, sediciosos, insurgentes.
        \item[c) Oración:] "...dificultaba enormemente la capacidad de las personas para formular pensamientos críticos o \textit{rebeldes}."
    \end{description}

    \item \textbf{volátil}
    \begin{description}
        \item[a) Significado:] Que cambia o varía con facilidad; inconstante, inestable.
        \item[b) Sinónimos:] Cambiante, inestable, inconstante.
        \item[c) Oración:] "La destrucción del pasado y la creación de una verdad siempre cambiante mostraban la naturaleza \textit{cambiante} de la realidad en 1984, donde lo que hoy era cierto, mañana podía ser erradicado por el simple deseo del Partido."
    \end{description}

    \item \textbf{instintivo}
    \begin{description}
        \item[a) Significado:] Que pertenece o se relaciona con el instinto; que actúa de manera natural, impulsiva o no razonada.
        \item[b) Sinónimos:] Automático, natural, impulsivo.
        \item[c) Oración:] "El sentimiento de desconfianza estaba diseñado para ser \textit{automático}; la población vivía en un constante estado de paranoia, ya que cualquier gesto, palabra o pensamiento podía ser un indicio de subversión."
    \end{description}
    
    \item \textbf{indicio}
    \begin{description}
        \item[a) Significado:] Señal o circunstancia que permite inferir la existencia de algo o la realización de una acción de la que no se tiene conocimiento directo.
        \item[b) Sinónimos:] Señal, pista, signo.
        \item[c) Oración:] "...ya que cualquier gesto, palabra o pensamiento podía ser una \textit{señal} de subversión."
    \end{description}

    \item \textbf{intervención}
    \begin{description}
        \item[a) Significado:] Acción de tomar parte en un asunto, especialmente para influir en él o modificarlo; intromisión.
        \item[b) Sinónimos:] Injerencia, intromisión, interferencia.
        \item[c) Oración:] "La tecnología de vigilancia, representada por los telescreens, no solo permitía la observación constante, sino que también facilitaba la \textit{injerencia} en los pensamientos y acciones de cada individuo, garantizando la sumisión total del pueblo."
    \end{description}

    \item \textbf{regímenes} % Corregido
    \begin{description}
        \item[a) Significado:] Sistemas políticos por los que se rige una nación o forma de gobierno.
        \item[b) Sinónimos:] Sistemas (políticos), gobiernos, estructuras (de poder).
        \item[c) Oración:] "Los \textit{sistemas} totalitarios que controlan la realidad de sus pueblos a través de la manipulación sistemática del lenguaje y la historia, logran borrar cualquier vestigio de autonomía de pensamiento en sus ciudadanos."
        % Nota: La palabra subrayada originalmente era "regímenes", aquí se usa un sinónimo.
        % La palabra "vestigio" también está en la oración original, pero no era la palabra objetivo.
    \end{description}

    \item \textbf{despersonalización}
    \begin{description}
        \item[a) Significado:] Acción y efecto de privar a algo o a alguien de su carácter personal o individual, tratándolo como un objeto o un número.
        \item[b) Sinónimos:] Deshumanización, cosificación, alienación (en cierto sentido).
        \item[c) Oración:] "En la sociedad de 1984, la \textit{deshumanización} de los individuos y su conversión en simples instrumentos al servicio del Partido aseguraba la uniformidad absoluta de la población."
    \end{description}

    \item \textbf{erige}
    \begin{description}
        \item[a) Significado:] (Del verbo erigir) Levantar, construir o instituir algo; dar a alguien o algo una categoría o posición superior.
        \item[b) Sinónimos:] Se alza, se constituye, se presenta.
        \item[c) Oración:] "El Gran Hermano \textit{se alza} como la figura de una autoridad divina, cuya omnipresencia no solo se simboliza en las paredes y televisores, sino en la constante autocensura a la que se somete cada individuo."
    \end{description}

\end{enumerate} % Fin de la lista numerada principal

% --- Fin del Documento ---
\end{document}
