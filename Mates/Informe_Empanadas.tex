\documentclass[12pt]{article}
\usepackage{amsmath}  % Math
\usepackage{amssymb}  % Symbols
\usepackage{graphicx} % Images
\usepackage[utf8]{inputenc}
\usepackage[T1]{fontenc}
\usepackage[margin=1in]{geometry}
\usepackage[spanish]{babel}
\usepackage{transparent}
\usepackage{eso-pic}
\usepackage{xcolor}

\graphicspath{{images/}} % Path to images
\newcommand\BackgroundPic{
    \put(0,0){
        \parbox[b][\paperheight]{\paperwidth}{
            \vfill
            \centering
            \transparent{0.2}\includegraphics[width=\paperwidth]{logo} % your image
            \vfill
        }
    }
}


\title{Informe Negocio Empanadas: \\
\textit{Las Empanadas Hermanas}} % Title of the report
  \author{Autores: Felipe Colli, Juan Gonzalez, Bastián Ortiz y Javier Robles \thanks{Instituto Nacional General José Miguel Carrera} \\
  Curso: \textit{4°H}, Profesor: \textit{Carlos Morales}} % Add your names and course
  \date{30 de mayo de 2025} % Fecha de Entrega
\AddToShipoutPicture{\BackgroundPic} % Add background image

\begin{document}

\maketitle
%\includegraphics[width=0.95\textwidth]{empanadas} % Logo of the school
\newpage

\tableofcontents
\newpage

\section{Resumen} % Aprox 1/3 de Pagina

\newpage



\section{Introducción} % Max 1 pagina
El presente informe tiene como objetivo presentar el negocio de empanadas "Empanadas de Juancho", un emprendimiento que busca ofrecer empanadas de alta calidad. A través de este documento, se detallarán los aspectos clave del negocio, incluyendo su propuesta de valor, mercado objetivo y proyecciones financieras. \\


Dentro de las proyeciones financieras, se contempla un análisis de costos y precios, así como una estimación de las ganancias esperadas. El negocio se enfoca en la producción y venta de 3 tamaños de empanadas de queso, con un énfasis en la calidad de los ingredientes y la satisfacción del cliente, sin olvidar el marco legal regulatorio, por lo cual este negocio no sera un frente para lavado de activos, evasion de impuestos, generacion de facturas ideologicamnete falsas, o la venta de drogas. \\ % Referencia a Breking Bad y Los Pollos Hermanos 


\subsection{Objetivos del Negocio:}
\begin{enumerate}
    \item \textbf{Propuesta de Valor:} Ofrecer empanadas de alta calidad, elaboradas con ingredientes frescos y locales, destacando la variedad de sabores y la atención al cliente.
    \item \textbf{Mercado Objetivo:} El negocio se enfocará en consumidores locales que valoran la calidad y la autenticidad en sus alimentos, así como en aquellos que buscan opciones convenientes y deliciosas para sus comidas diarias.
    \item \textbf{Proyecciones Financieras:} Se espera que el negocio genere ingresos sostenibles a través de la venta de empanadas, con un enfoque en la rentabilidad a largo plazo. Las proyecciones incluirán un análisis detallado de costos, precios y ganancias esperadas.
    \item \textbf{Análisis de Costos y Precios:} Se realizará un estudio exhaustivo de los costos asociados con la producción y venta de empanadas, sin incluir mano de obra y gastos operativos. Además, se establecerán precios competitivos que reflejen la calidad del producto y permitan un margen de ganancia adecuado.
    \item \textbf{Estudio de Mercado:} Se llevará a cabo un análisis del mercado local para identificar las tendencias y preferencias de los consumidores, así como la competencia existente en el sector de alimentos. Esto permitirá ajustar la estrategia del negocio para maximizar su éxito.
    \item \textbf{Marco Legal:} El negocio se compromete a operar dentro del marco legal regulatorio, asegurando que todas las operaciones sean transparentes y cumplan con las normativas vigentes. Esto incluye la obtención de licencias necesarias y el cumplimiento de las normativas de seguridad alimentaria.
    \item \textbf{Viabilidad del Negocio:} Se evaluará la viabilidad del negocio a través de un análisis financiero detallado, que incluirá proyecciones de ingresos, costos y ganancias. Esto permitirá determinar si el negocio es sostenible a largo plazo y si puede generar un retorno adecuado sobre la inversión inicial.
\end{enumerate}

\newpage



\section{Desarrollo} % 2 - 5 paginas

\subsection{Tabla de Costos Ingredientes}

\begin{table}[h!]
    \centering
    \begin{tabular}{|| c | c | c | c||} %MAsa Prehecha || costo en Negocio 1 || costo por unidad en Negocio 1 || costo en Negocio 2 || costo por unidad en Negocio 2
        \hline
    \textbf{Distribuidor} & Unidades & \textbf{Costo} & \textbf{Costo 100 Unidades} \\ [0.5ex]
        \hline\hline

        \multicolumn{4}{||c||}{\textbf{Masa Prehecha de Queso Chica}} \\ [0.5ex] \hline \hline
        El Palacio de las Empanadas & 20 Un & \$2900 & \$14500 \\ \hline
        Masas Mi Tierra & 25 Un & \$4500 & \$17000 \\ \hline
        --- & --- & --- & --- \\ [1ex] \hline \hline

        \multicolumn{4}{||c||}{\textbf{Masa Prehecha de Queso Mediana}} \\ [0.5ex] \hline \hline
        El Palacio de las Empanadas & 20 Un & \$3500 & \$17500 \\ \hline
        Masas Mi Tierra & 25 Un & \$5750 & \$22000 \\ \hline
        Alimentos La Kosa & 20 Un & \$3200 & \$16000 \\ [1ex] \hline \hline

        \multicolumn{4}{||c||}{\textbf{Masa Prehecha de Queso Grande}} \\ [0.5ex] \hline \hline
        El Palacio de las Empanadas & 20 Un & \$4100 & \$20500 \\ \hline
        Masas Mi Tierra & 25 Un & \$8000 & \$31000 \\ \hline
        Alimentos La Kosa & 20 Un & \$4200 & \$21000 \\ [1ex] \hline \hline

        \multicolumn{3}{||c|}{\textbf{Aceite}} & \textbf{Costo por Litro} \\ [0.5ex] \hline \hline
        Aceite 1 & & \$1200 & \$1200 \\ \hline
        Aceite 2 & & \$1800 & \$1800 \\ \hline
        Aceite 3 & & \$4200 & \$2400 \\ [1ex] \hline \hline



    \end{tabular}
    \caption{Tabla de Costos de Masas Prehechas}
    \label{tab:costos_masas}
\end{table}



\newpage



\section{Conclusiones} % Max 1 pagina



\end{document}
