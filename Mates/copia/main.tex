\documentclass[12pt]{article}
\usepackage{amsmath}  % Math
\usepackage{amssymb}  % Symbols
\usepackage{graphicx} % Images
\usepackage[utf8]{inputenc}
\usepackage[T1]{fontenc}
\usepackage[margin=1in]{geometry}
\usepackage[spanish]{babel}
\usepackage{transparent}
\usepackage{eso-pic}
\usepackage{xcolor}
\usepackage{subcaption} % For subfigures
\usepackage{array} % For better table column definitions

%\graphicspath{{images/}} % Path to images
\newcommand\BackgroundPic{
    \put(0,0){
        \parbox[b][\paperheight]{\paperwidth}{
            \vfill
            \centering
            \transparent{0.1}\includegraphics[width=0.8\paperwidth,height=\paperheight,keepaspectratio]{logo} % your image, adjusted width
            \vfill
        }
    }
}


\title{Informe Negocio Empanadas: \\
\textit{Las Empanadas Hermanas}} % Title of the report
  \author{Autores: Felipe Colli, Juan Gonzalez, Bastián Ortiz y Javier Robles \thanks{Instituto Nacional General José Miguel Carrera} \\
  Curso: \textit{4°H}, Profesor: \textit{Carlos Morales}} % Add your names and course
  \date{30 de mayo de 2025} % Fecha de Entrega
\AddToShipoutPicture{\BackgroundPic} % Add background image

\begin{document}

\maketitle
%\includegraphics[width=0.95\textwidth]{empanadas} % Logo of the school - This was commented out, seems okay.
\newpage

\tableofcontents
\newpage

\section{Resumen} % Aprox 1/3 de Pagina
El presente trabajo detalla la planificación financiera inicial para un emprendimiento de venta de empanadas denominado "Las Empanadas Hermanas". El grupo de estudiantes ha seleccionado tres variedades de empanadas: Pino, Queso Clásica y Camarón Queso, estimando una producción y venta inicial de 50 unidades de cada una (total 150 empanadas) para el primer mes.
Se ha realizado una cotización de los ingredientes necesarios en al menos dos distribuidores, documentada con evidencia real. A partir de esto, se elaboró una lista de compras optimizada y se simuló una factura de adquisición de insumos, detallando valor neto, IVA crédito fiscal y total.
Posteriormente, se definieron los precios de venta para cada variedad, considerando costos de producción, precios de mercado y un margen de ganancia. Con estos precios, se proyectaron los ingresos totales netos y el IVA débito fiscal correspondiente.
Finalmente, se calculó el IVA a pagar al fisco y la ganancia neta del primer mes. Dicha ganancia se evaluó en función de su capacidad para cubrir los costos de producción del siguiente mes y generar un excedente como "sueldo" para los integrantes del grupo, cumpliendo así con los requisitos del Avance 1 del proyecto de Matemática Financiera. Los cálculos indican que el negocio, bajo las condiciones estimadas, es viable y cumple con las expectativas iniciales de rentabilidad.

\newpage

\section{Introducción} % Max 1 pagina
El presente informe tiene como objetivo presentar el negocio de empanadas "Las Empanadas Hermanas", un emprendimiento que busca ofrecer empanadas de alta calidad. A través de este documento, se detallarán los aspectos clave del negocio, incluyendo su propuesta de valor, mercado objetivo y proyecciones financieras iniciales correspondientes al primer avance del trabajo de Matemática Financiera. \\

Dentro de las proyecciones financieras, se contempla un análisis de costos y precios para una producción inicial, así como una estimación de las ganancias esperadas. El negocio se enfoca en la producción y venta de tres variedades principales de empanadas: Pino (carne), Queso Clásica, y Camarón Queso, con un énfasis en la calidad de los ingredientes y la satisfacción del cliente. Todo esto se desarrolla sin olvidar el marco legal regulatorio, por lo cual este negocio no será un frente para lavado de activos, evasión de impuestos, generación de facturas ideológicamente falsas, o la venta de drogas ilícitas. \\ % Referencia a Breking Bad y Los Pollos Hermanos. Mantener el toque estudiantil.


\subsection{Objetivos del Negocio (Avance 1):}
\begin{enumerate}
    \item \textbf{Propuesta de Valor Inicial:} Ofrecer tres variedades de empanadas (Pino, Queso, Camarón Queso) de alta calidad, elaboradas con ingredientes frescos, destacando el sabor tradicional.
    \item \textbf{Estimación de Producción y Ventas:} Definir una cantidad inicial de productos a vender para el primer mes (50 unidades por variedad, total 150 empanadas).
    \item \textbf{Análisis de Costos de Ingredientes:} Cotizar los ingredientes en al menos dos proveedores y seleccionar los más convenientes para elaborar una lista de compras detallada y valorizada.
    \item \textbf{Gestión de IVA en Compras:} Elaborar una factura de compra simulada que refleje el valor neto de los ingredientes, el IVA crédito fiscal y el total pagado.
    \item \textbf{Estrategia de Precios y Proyección de Ingresos:} Definir precios de venta competitivos para cada variedad, que cubran costos y generen ganancia. Calcular el ingreso total neto esperado y el IVA débito fiscal.
    \item \textbf{Cálculo de Obligaciones Tributarias (IVA):} Determinar el monto de IVA a pagar al fisco, resultante de la diferencia entre IVA débito e IVA crédito.
    \item \textbf{Evaluación de Rentabilidad Inicial:} Calcular las ganancias del primer mes y verificar si son suficientes para reinvertir en la producción del mes siguiente y obtener un excedente (sueldo para el grupo).
    \item \textbf{Viabilidad del Negocio (Preliminar):} Evaluar la viabilidad preliminar del negocio basándose en los cálculos financieros del primer mes de operación.
\end{enumerate}

\newpage

\section{Desarrollo} % 2 - 5 paginas

\subsection{Estimación de Cantidad y Variedades}
Para el primer mes de operación de "Las Empanadas Hermanas", se ha decidido producir y vender las siguientes variedades y cantidades:
\begin{itemize}
    \item Empanada de Pino: 50 unidades
    \item Empanada de Queso Clásica: 50 unidades
    \item Empanada de Camarón Queso: 50 unidades
\end{itemize}
Esto suma un total de \textbf{150 empanadas} a producir y vender durante el primer mes. Esta cantidad se considera manejable para un inicio y permitirá probar la aceptación del producto en el mercado.

\subsection{Cotización de Ingredientes}
Se cotizaron los principales ingredientes en diversos proveedores. La siguiente tabla resume los costos unitarios o por volumen encontrados. Los precios se consideran netos para facilitar los cálculos posteriores, a menos que se indique lo contrario (ej. "+IVA").

\begin{table}[h!]
    \centering
    \begin{tabular}{|| l | c | c | c||} 
        \hline
    \textbf{Distribuidor} & \textbf{Formato} & \textbf{Costo Neto} & \textbf{Costo Neto Detalle} \\ [0.5ex]
        \hline\hline
        \multicolumn{4}{||c||}{\textbf{Masa Prehecha Grande (22cm)}} \\ [0.5ex] \hline \hline
        El Palacio de las Empanadas & 20 Un & \$4.100 & \$205 /Unidad \\ \hline
        Masas Mi Tierra & 25 Un & \$8.000 & \$320 /Unidad \\ \hline
        Alimentos La Kosa & 20 Un & \$4.200 & \$210 /Unidad \\ [1ex] \hline \hline

        \multicolumn{4}{||c||}{\textbf{Huevos (Tipo A)}} \\ [0.5ex] \hline \hline
        El Don Huevo & 180 Un & \$39.500 & \$2.633 /Docena \\ \hline
        Huevos La Montaña Pelluhue & 180 Un & \$43.990 & \$2.933 /Docena \\ \hline
        AgricoVial & 180 Un & \$41.600 & \$2.773 /Docena \\ [1ex] \hline \hline

        \multicolumn{4}{||c||}{\textbf{Camarón Pelado Cocido (100-200)}} \\ [0.5ex] \hline \hline
        Distribuidora GK & 10 KG & \$48.900 & \$4.890 /KG \\ \hline
        Comercial Oceanica & 1 KG & \$4.690 + IVA & \$4.690 /KG \\ \hline % Asumiendo que $4690 es neto
        Del Origen & 10 KG & \$48.700 & \$4.870 /KG \\ [1ex] \hline \hline

        \multicolumn{4}{||c||}{\textbf{Carne Molida Vacuno}} \\ [0.5ex] \hline \hline
        Central Mayorista & 500 G & \$2.590 & \$5.180 /KG \\ \hline 
        Unimarc & 1 KG & \$4.990 & \$4.990 /KG \\ \hline 
        El Paiquito & 250 G & \$1.090 & \$4.360 /KG \\ [1ex] \hline \hline

        \multicolumn{4}{||c||}{\textbf{Queso Gauda o similar (para empanadas)}} \\ [0.5ex] \hline \hline
        Distribuidora Santiago & ~2.4KG & \$22.990 & \$9.579 /KG \\ \hline % Asumiendo 2.4kg
        Distribuidora Nueva de Matte & 1 KG & \$6.940 & \$6.940 /KG \\ \hline
        Mayorista 10 & 1 KG & \$9.160 & \$9.160 /KG \\ [1ex] \hline \hline

        \multicolumn{4}{||c||}{\textbf{Cebolla}} \\ [0.5ex] \hline \hline
        Santiago Natural Foods & Malla 14 KG & \$10.990 & \$785 /KG \\ \hline
        Distribuidora Frest & Saco 10 KG & \$9.990 & \$990 /KG \\ \hline
        Lider & 1 KG & \$1.195 & \$1.195 /KG \\ [1ex] \hline \hline

        \multicolumn{4}{||c||}{\textbf{Aceite Vegetal para Freír}} \\ [0.5ex] \hline \hline
        ProsudMarket (Maravilla) & Caja 18 L (6x3L) & \$49.000 & \$2.722 /Litro \\ \hline % Precio caja $49000
        Central Mayorista (Coliseo) & Caja 13.5 L (15x0.9L) & \$22.200 & \$1.644 /Litro \\ \hline
        Distribuidora Sabatini & Bidón 5L & \$9.296 & \$1.859 /Litro \\ [1ex] \hline \hline
    \end{tabular}
    \caption{Tabla de Costos Comparativos de Ingredientes (Precios Netos Estimados)}
    \label{tab:costos_ingredientes_comparativos}
\end{table}
\newpage

\subsubsection{Evidencia de Cotizaciones (Imágenes)}
A continuación, se presentan algunas capturas de pantalla como evidencia de las cotizaciones realizadas:

        \begin{figure}[h!] % Use [h!] or [htbp] for better float placement
            \centering % Center the content of the figure
            \begin{subfigure}{0.45\textwidth}
                \centering
                \includegraphics[width=0.9\linewidth]{palaci} % Assumed image: Masa Palacio
                \caption{Masa (El Palacio de las Emp.)}
                \label{fig:palacio}
            \end{subfigure}
            \hfill % Add some space between subfigures
            \begin{subfigure}{0.45\textwidth}
                \centering
                \includegraphics[width=0.9\linewidth]{kosa} % Assumed image: Masa Kosa
                \caption{Masa (Alimentos La Kosa)}
                \label{fig:kosa}
            \end{subfigure}
            \caption{Cotizaciones de Distribuidores de Masa Prehecha}
            \label{fig:cotizaciones_masas}
        \end{figure} % END OF THIS FIGURE

        \begin{figure}[h!] % START A NEW FIGURE
            \centering
            \begin{subfigure}{0.45\textwidth}
                \centering
                \includegraphics[width=0.9\linewidth]{donhuevo} % Assumed image: Huevos Don Huevo
                \caption{Huevos (El Don Huevo)}
                \label{fig:don_huevo}
            \end{subfigure}
            \hfill
            \begin{subfigure}{0.45\textwidth}
                \centering
                \includegraphics[width=0.9\linewidth]{montan1} % Assumed image: Huevos La Montaña
                \caption{Huevos (La Montaña Pelluhue)}
                \label{fig:huevos_montaña}
            \end{subfigure}
            \caption{Cotizaciones de Distribuidores de Huevos}
            \label{fig:cotizaciones_huevos}
        \end{figure} % END OF THIS FIGURE
        \newpage

        \begin{figure}[h!] % START A NEW FIGURE
            \centering
            \begin{subfigure}{0.45\textwidth}
                \centering
                \includegraphics[width=0.9\linewidth]{gk} % Assumed image: Camarón GK
                \caption{Camarón (Distribuidora GK)}
                \label{fig:distribuidora_gk}
            \end{subfigure}
            \hfill
            \begin{subfigure}{0.45\textwidth}
                \centering
                \includegraphics[width=0.9\linewidth]{oceanic} % Assumed image: Camarón Oceanica
                \caption{Camarón (Comercial Oceanica)}
                \label{fig:comercial_oceanica}
            \end{subfigure}
            \caption{Cotizaciones de Distribuidores de Camarón}
            \label{fig:cotizaciones_camaron}
        \end{figure} % END OF THIS FIGURE

        \begin{figure}[h!] % START A NEW FIGURE
            \centering
            \begin{subfigure}{0.45\textwidth}
                \centering
                \includegraphics[width=0.9\linewidth]{central} % Assumed image: Carne Central Mayorista
                \caption{Carne Molida (Central Mayorista)}
                \label{fig:central_mayorista}
            \end{subfigure}
            \hfill
            \begin{subfigure}{0.45\textwidth}
                \centering
                \includegraphics[width=0.9\linewidth]{paiquito} % Assumed image: Carne Paiquito
                \caption{Carne Molida (El Paiquito)}
                \label{fig:paiquito_carne}
            \end{subfigure}
            \caption{Cotizaciones de Distribuidores de Carne Molida}
            \label{fig:cotizaciones_carne_molida}
        \end{figure} % END OF THIS FIGURE
        \newpage

        \begin{figure}[h!] % START A NEW FIGURE
            \centering
            \begin{subfigure}{0.45\textwidth}
                \centering
                \includegraphics[width=0.9\linewidth]{santiago} % Assumed image: Queso Dist. Santiago
                \caption{Queso (Distribuidora Santiago)}
                \label{fig:distribuidora_santiago}
            \end{subfigure}
            \hfill
            \begin{subfigure}{0.45\textwidth}
                \centering
                \includegraphics[width=0.9\linewidth]{nueva} % Assumed image: Queso Nueva de Matte
                \caption{Queso (Distribuidora Nueva de Matte)}
                \label{fig:distribuidora_nueva_de_matte}
            \end{subfigure}
            \caption{Cotizaciones de Distribuidores de Queso}
            \label{fig:cotizaciones_queso}
        \end{figure} % END OF THIS FIGURE

        \begin{figure}[h!] % START A NEW FIGURE
            \centering
            \begin{subfigure}{0.45\textwidth}
                \centering
                \includegraphics[width=0.9\linewidth]{nat} % Assumed image: Cebolla Santiago Natural Foods
                \caption{Cebolla (Santiago Natural Foods)}
                \label{fig:santiago_natural_foods}
            \end{subfigure}
            \hfill
            \begin{subfigure}{0.45\textwidth}
                \centering
                \includegraphics[width=0.9\linewidth]{fres} % Assumed image: Cebolla Dist. Frest
                \caption{Cebolla (Distribuidora Frest)}
                \label{fig:distribuidora_frest}
            \end{subfigure}
            \caption{Cotizaciones de Distribuidores de Cebolla}
            \label{fig:cotizaciones_cebolla}
        \end{figure} % END OF THIS FIGURE
        \newpage
        
        \begin{figure}[h!] % START A NEW FIGURE
            \centering
            \begin{subfigure}{0.45\textwidth}
                \centering
                \includegraphics[width=0.9\linewidth]{prosud} % Assumed image: Aceite Prosud
                \caption{Aceite (ProsudMarket)}
                \label{fig:prosudmarket}
            \end{subfigure}
            \hfill
            \begin{subfigure}{0.45\textwidth}
                \centering
                \includegraphics[width=0.9\linewidth]{aceite} % Assumed image: Aceite Central Mayorista
                \caption{Aceite (Central Mayorista)}
                \label{fig:central_mayorista_aceite}
            \end{subfigure}
            \caption{Cotizaciones de Distribuidores de Aceite}
            \label{fig:cotizaciones_aceite}
        \end{figure} % END OF THIS FIGURE
\newpage
\subsection{Lista de Compras Seleccionada (para 150 empanadas)}
Basándose en las cotizaciones y la cantidad estimada, se elaboró la siguiente lista de compras, seleccionando la opción más conveniente (precio/calidad/formato) para cada ingrediente:

\begin{table}[h!]
    \centering
    \begin{tabular}{||l | r | l | r | r ||} 
        \hline
        \textbf{Ingrediente} & \textbf{Cantidad} & \textbf{Proveedor} & \textbf{Costo Neto Total} \\ [0.5ex]
        \hline\hline
        Masa Empanada (22cm) & 150 Unidades & El Palacio & \$30.750 \\ 
        (Compra: 8 paq. x 20Un) & & & (\$32.800) \\ \hline
        Carne Molida Vacuno & 3 KG & El Paiquito & \$13.080 \\ \hline
        Cebolla & 3 KG & S. Natural Foods & \$2.355 \\ \hline
        Huevos & 3 Docenas & El Don Huevo & \$7.899 \\ \hline
        Queso Gauda/similar & 6 KG & Nueva de Matte & \$41.640 \\ \hline
        Camarón Pelado Cocido & 3 KG & Del Origen & \$14.610 \\ \hline
        Aceite para Freír & 5 Litros & D. Sabatini & \$9.296 \\ \hline \hline
        \multicolumn{3}{||r|}{\textbf{TOTAL NETO INGREDIENTES}} & \textbf{\$121.730} \\ [0.5ex] 
        \hline
    \end{tabular}
    \caption{Lista de Compras Seleccionada y Costos Netos Totales. Nota: Para masa se compran 8 paquetes (160 Unidades), costo total \$32.800, se usa \$30.750 para 150 unidades. Se usará el costo de compra real \$32.800 para la factura. Entonces el total cambia a \$121.730 - \$30.750 + \$32.800 = \$123.780}
    \label{tab:lista_compras_final}
\end{table}
Ajustando el total por la compra de paquetes completos de masa: El costo neto total de ingredientes es \textbf{\$123.780}.

\subsection{Factura de Compra de Ingredientes (Simulada)}
A continuación, se presenta un ejemplo de factura simulada por la compra de los ingredientes seleccionados. Para simplificar, se consolida en una sola factura de un proveedor ficticio "Insumos FoodPro SpA".

\begin{table}[h!]
    \centering
    \textbf{FACTURA ELECTRÓNICA N° 001} \\
    \vspace{0.2cm}
    \begin{tabular}{l l}
        Proveedor: & Insumos FoodPro SpA \\
        RUT: & 77.777.777-7 \\
        Dirección: & Avenida Ficticia 123, Santiago \\
        Fecha: & 15 de mayo de 2025 \\
    \end{tabular}
    \vspace{0.3cm}
    \begin{tabular}{|c|l|r|r|r|}
        \hline
        \textbf{Cant.} & \textbf{Descripción} & \textbf{P. Neto Unit.} & \textbf{Subtotal Neto} \\
        \hline
        8 paq. & Masa Empanada (20Un/paq) & \$4.100 & \$32.800 \\
        3 KG & Carne Molida Vacuno & \$4.360 & \$13.080 \\
        3 KG & Cebolla & \$785 & \$2.355 \\
        3 doz. & Huevos & \$2.633 & \$7.899 \\
        6 KG & Queso Gauda & \$6.940 & \$41.640 \\
        3 KG & Camarón Pelado Cocido & \$4.870 & \$14.610 \\
        5 L & Aceite para Freír & \$1.859,20 & \$9.296 \\
        \hline
        \multicolumn{3}{|r|}{\textbf{Subtotal Neto}} & \textbf{\$121.680} \\ % Note: Sum of listed items. $32800+13080+2355+7899+41640+14610+9296 = 121680
        \multicolumn{3}{|r|}{\textbf{IVA (19\%)}} & \textbf{\$23.119} \\
        \multicolumn{3}{|r|}{\textbf{TOTAL FACTURA}} & \textbf{\$144.799} \\
        \hline
    \end{tabular}
    \caption{Factura Simulada de Compra de Ingredientes.}
    \label{tab:factura_compra}
\end{table}
El costo neto total de ingredientes según esta factura es \textbf{\$121.680}. (Hubo una pequeña diferencia con el cálculo anterior por el redondeo o ajuste de la masa, esta es la cifra que se usará para IVA Crédito).
El IVA Crédito Fiscal generado por esta compra es de \textbf{\$23.119}.

\subsection{Definición de Precio de Venta y Proyección de Ingresos}
Los precios de venta netos por unidad se han definido buscando un equilibrio entre costos, mercado y rentabilidad:
\begin{itemize}
    \item Empanada de Pino: \$2.000 (Neto) $\rightarrow$ Precio Venta Público (c/IVA): \$2.380
    \item Empanada de Queso Clásica: \$1.800 (Neto) $\rightarrow$ Precio Venta Público (c/IVA): \$2.142
    \item Empanada de Camarón Queso: \$2.500 (Neto) $\rightarrow$ Precio Venta Público (c/IVA): \$2.975
\end{itemize}

Proyección de Ingresos Netos (venta de 50 unidades de cada variedad):
\begin{itemize}
    \item Ingresos Pino: 50 unidades * \$2.000/unidad = \$100.000
    \item Ingresos Queso: 50 unidades * \$1.800/unidad = \$90.000
    \item Ingresos Camarón Queso: 50 unidades * \$2.500/unidad = \$125.000
\end{itemize}
\textbf{Total Ingresos Netos Estimados (Ventas Netas): \$315.000}

El IVA Débito Fiscal generado por estas ventas sería:
\$315.000 * 0,19 = \textbf{\$59.850 (IVA Débito Fiscal)}

\subsection{Cálculo de IVA a Pagar al Fisco}
El IVA a pagar al fisco se calcula como la diferencia entre el IVA Débito y el IVA Crédito:
\begin{itemize}
    \item IVA Débito Fiscal (por ventas): \$59.850
    \item IVA Crédito Fiscal (por compras): \$23.119
\end{itemize}
IVA a Pagar = \$59.850 - \$23.119 = \textbf{\$36.731}

\subsection{Cálculo de Ganancias}
La ganancia se calcula como los Ingresos Netos menos los Costos Netos de los ingredientes:
\begin{itemize}
    \item Total Ingresos Netos: \$315.000
    \item Total Costo Neto de Ingredientes: \$121.680
\end{itemize}
Ganancia Bruta (antes de otros gastos) = \$315.000 - \$121.680 = \textbf{\$193.320}

Esta ganancia bruta debe cumplir dos condiciones:
\begin{enumerate}
    \item Cubrir la producción del próximo mes (asumiendo mismos costos): \$121.680
    \item Dejar un saldo a favor como sueldo para el grupo.
\end{enumerate}
Saldo para sueldo = Ganancia Bruta - Costo de Reposición de Inventario
Saldo para sueldo = \$193.320 - \$121.680 = \textbf{\$71.640}

Este monto de \$71.640 representa el sueldo total para el grupo de 4 estudiantes, lo que equivale a \$17.910 para cada uno. Si bien es una cifra modesta, cumple con el requisito de generar un excedente como sueldo después de asegurar la continuidad de la producción para el siguiente mes. Para futuros avances, se podría analizar el impacto de aumentar los volúmenes de venta o ajustar precios para mejorar este indicador.

\newpage
\section{Conclusiones} % Max 1 pagina
El primer avance del proyecto "Las Empanadas Hermanas" ha permitido aplicar los conceptos de Matemática Financiera a un caso práctico de emprendimiento. Se seleccionaron tres variedades de empanadas (Pino, Queso Clásica, Camarón Queso) con una producción inicial de 150 unidades.

La cotización de ingredientes y la elaboración de una lista de compras optimizada permitieron determinar un costo neto total de insumos de \$121.680, generando un IVA crédito fiscal de \$23.119.

Se establecieron precios de venta competitivos, proyectando un ingreso neto total de \$315.000 y un IVA débito fiscal de \$59.850. Con esto, el IVA a pagar al fisco se calculó en \$36.731.

La ganancia bruta estimada para el primer mes asciende a \$193.320. Este monto es suficiente para cubrir la reposición de inventario para el siguiente ciclo de producción (\$121.680) y deja un saldo favorable de \$71.640 como sueldo para el grupo.

Aunque el sueldo individual resultante (\$17.910) es modesto, el ejercicio demuestra la viabilidad financiera inicial del negocio bajo las condiciones y supuestos establecidos. Cumple con los objetivos de este primer avance, que son fundamentar decisiones financieras básicas, estimar costos, ganancias y factibilidad preliminar.

Para etapas futuras, será crucial considerar otros costos operativos (arriendo, servicios, marketing, mano de obra si se escala), así como refinar las estimaciones de demanda y optimizar la estructura de costos para mejorar la rentabilidad y el sueldo potencial del grupo. La base establecida en este avance es sólida para continuar desarrollando el plan de negocios.

\end{document}
