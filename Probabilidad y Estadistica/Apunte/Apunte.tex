\documentclass[12pt, letterpaper]{article}
\usepackage[utf8]{inputenc} % Input encoding
\usepackage[T1]{fontenc} % Font encoding
\usepackage{amsmath} % Math symbols and environments
\usepackage{amssymb} % More math symbols
\usepackage{graphicx} % Include graphics
%\usepackage[spanish]{babel} % Spanish language support
\usepackage{tabularx}
\renewcommand{\arraystretch}{1.5} % Incrementa el espaciado vertical

\title{Apuntes Probabilidad y Estadística}
\author{Felipe Colli}
\date{2024}

\begin{document}
\maketitle
\tableofcontents
\newpage % Start sections on a new page after ToC

\section{14/03}
\subsection{Materia}
\textbf{\textit{\underline{Población:}}}
Conjunto de todos los elementos que se quieren estudiar. Cuando la información deseada está disponible para todos los objetos de la población, lo llamamos \textbf{censo}. En la práctica es muy difícil o casi imposible realizar un censo.

\textbf{\textit{\underline{Muestra:}}}
Subconjunto de la población que se mide u observa.

\textbf{\textit{\underline{Parámetro:}}}
Es una medición numérica que describe algunas características de una población. % Corrected typo pobalción -> población

\textbf{\textit{\underline{Estadístico (estadígrafo):}}} Es una medición numérica que describe algunas características de la muestra.

\textbf{\textit{\underline{Variables cualitativas:}}}
\begin{itemize}
    \item Se describen mediante palabras o categorías.
    \item Se usan para categorizar a los individuos o para identificar.
    \item Sirven para comprender aspectos subjetivos y complejos.
    \item Se pueden clasificar en nominales y ordinales.
    \item Ejemplos: el color del cabello, el deporte favorito, la comida favorita, el lugar de nacimiento.
\end{itemize}

\textbf{\textit{\underline{Variables cuantitativas:}}}
\begin{itemize}
    \item Se expresan mediante números, es decir, se pueden contar o medir.
    \item Permiten más operaciones matemáticas.
    \item Se pueden usar para conocer fenómenos o situaciones a través de la recolección y generación de números y datos.
    \item Ejemplos: la edad, los ingresos, el peso, la altura, la presión, la humedad o cantidad de hermanos.
\end{itemize}

\subsection{Ejercicios:}
Para cada una de las siguientes situaciones, identifica la población de interés, la variable estadística, clasifícala, y entrega un ejemplo de cuál podría ser una posible muestra.

\begin{enumerate}
    \item Un investigador universitario desea estimar la proporción de ciudadanos chilenos de la \textit{GEN X} que están interesados en iniciar sus propios negocios.
        \begin{enumerate}
            \item \textbf{Población:} Chilenos de Gen X
            \item \textbf{Muestreo:} Santiaguinos de Gen X
            \item \textbf{Variable:} Quiere iniciar o no un negocio (cualitativa nominal)
        \end{enumerate}

    \item Durante más de un siglo, la temperatura corporal normal en seres humanos ha sido aceptada como 37°C. ¿Es así realmente? Los investigadores desean estimar el promedio de temperatura de adultos sanos en Chile.
        \begin{enumerate}
            \item \textbf{Población:} Adultos Sanos en Chile % Added "en Chile" for clarity
            \item \textbf{Muestreo:} Santiaguinos Adultos Sanos
            \item \textbf{Variable:} Temperatura Corporal (cuantitativa continua) % Corrected classification: cuantitativa continua is more appropriate than nominal
        \end{enumerate}

    \item Un ingeniero municipal desea estimar el promedio de consumo semanal de agua para unidades habitacionales unifamiliares en la ciudad.
        \begin{enumerate}
            \item \textbf{Población:} Unidades Habitacionales unifamiliares de la ciudad % Added "unifamiliares" for precision
            \item \textbf{Muestreo:} Unidades habitacionales unifamiliares de un sector de la ciudad % Added "unifamiliares"
            \item \textbf{Variable:} Consumo semanal de agua (cuantitativa continua) % Corrected classification: cuantitativa continua is more appropriate than nominal
        \end{enumerate}
    \item El National Highway Safety Council desea estimar la proporción de llantas para automóvil con dibujo o superficie de rodadura insegura, entre todas las llantas manufacturadas por una empresa específica durante el presente año de producción.
        \begin{enumerate}
            \item \textbf{Población:} Llantas de Empresa X manufacturadas este año de producción % Corrected "manufacturadas"
            \item \textbf{Muestreo:} Una selección aleatoria de llantas producidas en diferentes lotes/días. % More specific sampling method
            \item \textbf{Variable:} Estado de la rodadura (segura/insegura) (cualitativa nominal) % Corrected variable description and classification
        \end{enumerate}

    \item Un politólogo desea estimar si la mayoría de los residentes adultos de una región están a favor de una legislatura unicameral.
        \begin{enumerate}
            \item \textbf{Población:} Residentes adultos de la región
            \item \textbf{Muestreo:} Residentes adultos de una comuna (o varias comunas seleccionadas)
            \item \textbf{Variable:} Opinión sobre la legislatura unicameral (a favor/en contra/indeciso) (cualitativa nominal)  % Corrected variable description
        \end{enumerate}

    \item Un científico del área médica desea determinar el tiempo promedio para que se vuelva a presentar cierta enfermedad infecciosa, una vez que las personas se recuperan de ella por primera vez.
        \begin{enumerate}
            \item \textbf{Población:} Personas que se han recuperado de la enfermedad por primera vez
            \item \textbf{Muestreo:} Pacientes recuperados seleccionados de registros médicos
            \item \textbf{Variable:} Tiempo hasta una 2ª infección (cuantitativa continua) % Corrected classification: cuantitativa continua is more appropriate
        \end{enumerate}

    \item Un ingeniero electricista desea determinar si el promedio de vida útil de transistores de cierto tipo es mayor que 500 horas.
        \begin{enumerate}
            \item \textbf{Población:} Transistores del tipo X
            \item \textbf{Muestreo:} 20 transistores (o una muestra mayor seleccionada aleatoriamente)
            \item \textbf{Variable:} Vida útil del transistor (cuantitativa continua). (O, alternativamente, si la vida útil es > 500 horas (cualitativa nominal))  % Clarified variable options
        \end{enumerate}
\end{enumerate}
\newpage

\section{17/03}
\subsection{Materia}
Medidas de Tendencia Central 
Medidas de Posición 
Medidas de Dispersión 

\textbf{\textit{\underline{Tablas de Frecuencia: Conceptos Básicos}}}
\begin{itemize}
    \item \textbf{Dato o Intervalo:} Información (variable) que se estudia en estadística.
    \item \textbf{Marca de Clase ($c_i$):} Promedio entre los extremos de un intervalo. % Added ($c_i$)
    \item \textbf{Amplitud de un intervalo:} Es la diferencia entre el límite superior y el límite inferior.
\end{itemize}

\textbf{\textit{\underline{Tipos de Frecuencia:}}}
\begin{itemize}
    \item \textbf{Frecuencia Absoluta ($f$ o $f_i$):} Cantidad de veces que se repite un dato o que cae en un intervalo. % Clarified "que se repite"
    \item \textbf{Frecuencia Acumulada ($F$ o $F_i$):} Suma de las frecuencias absolutas hasta determinado dato o intervalo. % Corrected "terminado" -> "determinado", added "o intervalo"
    \item \textbf{Frecuencia Relativa ($f_r$ o $h_i$):} Es la fracción, decimal o porcentaje de cierto valor o intervalo. ($\frac{f_i}{n}$) % Added index i
    \item \textbf{Frecuencia Relativa Acumulada ($H_i$):} Es la fracción, decimal o porcentaje de la frecuencia acumulada hasta cierto dato o intervalo. ($\frac{F_i}{n}$) % Added index i, corrected formula symbol
\end{itemize}

\textbf{\textit{\underline{Medidas de Tendencia Central:}}}
\begin{itemize}
    \item \textbf{Media Aritmética ($\bar{x}$):} Es el cociente entre la suma de todos los datos y el número total de datos ($n$). Si se tienen $n$ datos $x_1, x_2, \dots, x_n$:
        \[ \bar{x}=\frac{x_1+x_2+\dots+x_n}{n} = \frac{\sum_{i=1}^{n} x_i}{n} \] \\ % Added summation notation
    Para datos agrupados en una tabla de frecuencia con $k$ clases:
    \begin{center} % Centra ambos elementos horizontalmente
        \begin{minipage}{0.45\textwidth} % Ajusta el ancho según sea necesario
            \begin{tabularx}{\textwidth}{|X|X|}
                \hline
                \textbf{Marca de clase ($c_i$)} & \textbf{Frecuencia ($f_i$)} \\
                \hline
                $c_1$ & $f_1$ \\
                \hline
                $c_2$ & $f_2$ \\
                \hline
                $\vdots$ & $\vdots$ \\
                \hline
                $c_k$ & $f_k$ \\ % Changed n to k for number of classes
                \hline
             \end{tabularx}
        \end{minipage}
        \hfill % Espacio flexible entre los elementos
        \begin{minipage}{0.45\textwidth} % Ajusta el ancho según sea necesario
            \centering
            \[
            \bar{x}=\frac{c_1 \cdot f_1 + c_2 \cdot f_2 + \dots + c_k \cdot f_k}{f_1+f_2+\dots+f_k} = \frac{\sum_{i=1}^{k} c_i f_i}{n} % Changed n to k, added summation notation
            \]
        \end{minipage}
    \end{center}


    \item \textbf{Mediana ($M_e$):} Es el dato que ocupa la posición central de la muestra cuando estos se encuentran ordenados en forma creciente o decreciente. \textbf{Si la muestra tiene un número par de datos, la mediana es la media aritmética de los dos términos centrales.} Se aplica principalmente a variables cuantitativas (y ordinales). % Added ordinales
        \begin{itemize}
            \item Si $n$ es impar, la posición es $\frac{n+1}{2}$. $M_e = x_{(\frac{n+1}{2})}$ % Corrected index notation
            \item Si $n$ es par, las posiciones son $\frac{n}{2}$ y $\frac{n}{2}+1$. $M_e = \frac{x_{(\frac{n}{2})} + x_{(\frac{n}{2}+1)}}{2}$ % Corrected index notation
        \end{itemize}
    \item \textbf{Moda ($M_o$):} Es el dato o los datos (o intervalo) que presentan la mayor frecuencia absoluta. La muestra puede ser: % Added "(o intervalo)"
        \begin{itemize}
            \item \textbf{Amodal:} No presenta moda.
            \item \textbf{Unimodal:} Un solo dato (o intervalo) tiene la frecuencia máxima. % Clarified
            \item \textbf{Bimodal:} Dos datos (o intervalos) tienen la misma frecuencia máxima. % Clarified
            \item \textbf{Polimodal (o Multimodal):} Más de dos datos (o intervalos) tienen la misma frecuencia máxima. % Corrected "Multiples datos presentan la mismma frecuencia baosluta y son los datos que mas se repiten"
            \item \textbf{Intervalo Modal:} (Para datos agrupados) El intervalo que presenta la mayor frecuencia absoluta.
        \end{itemize}
\end{itemize}
\newpage

\section{26/03}
\subsection{Ejercicios:}

\noindent % Prevent indentation for the first line
Si las notas de Esteban en una asignatura son \(3, 4, 6, 3, 5, 5, 6, 3, 4\) y de estas notas se cambian un $6$ por un $7$. ¿Cuál(es) de las siguientes medidas de tendencia central cambia(n)?
\begin{enumerate}
    \item La moda
    \item La mediana
    \item La media aritmética
\end{enumerate}
\textit{Solución:}
Notas originales (ordenadas): \(3, 3, 3, 4, 4, 5, 5, 6, 6\). $n=9$.
\begin{itemize}
    \item Moda original: 3 (frecuencia 3)
    \item Mediana original: El dato en la posición $\frac{9+1}{2}=5$. Mediana = 4.
    \item Media original: $\frac{3+3+3+4+4+5+5+6+6}{9} = \frac{39}{9} = 4.33...$
\end{itemize}
Notas nuevas (ordenadas): \(3, 3, 3, 4, 4, 5, 5, 6, 7\). $n=9$.
\begin{itemize}
    \item Moda nueva: 3 (sigue siendo la frecuencia más alta) $\rightarrow$ No cambia.
    \item Mediana nueva: El dato en la posición 5. Mediana = 4. $\rightarrow$ No cambia.
    \item Media nueva: $\frac{3+3+3+4+4+5+5+6+7}{9} = \frac{40}{9} = 4.44...$ $\rightarrow$ Cambia.
\end{itemize}
\textbf{Respuesta:} Solo la media aritmética cambia (3).

\vspace{1em} % Add some vertical space
\noindent La siguiente tabla muestra los valores de una variable \(X\) y sus respectivas frecuencias. ¿Cuál es el valor de la mediana?
\begin{center}
    \begin{tabular}{|c|c|c|} % Added cumulative frequency column
        \hline
        \textbf{\(X\)} & \textbf{Frecuencia ($f_i$)} & \textbf{Frecuencia Acumulada ($F_i$)} \\
        \hline
        4 & 4 & 4 \\
        \hline
        5 & 8 & 12 \\
        \hline
        6 & 10 & 22 \\
        \hline
        7 & 20 & 42 \\
        \hline
        8 & 8 & 50 \\
        \hline
        \textbf{Total} & \textbf{n=50} & \\ % Added total n
        \hline
    \end{tabular}
\end{center}
\textit{Solución:}
Total de datos $n=50$ (par). La mediana es el promedio de los datos en las posiciones $\frac{n}{2} = \frac{50}{2}=25$ y $\frac{n}{2}+1 = 26$.
Buscamos en la Frecuencia Acumulada ($F_i$) dónde caen estas posiciones.
\begin{itemize}
    \item Hasta $X=6$, hay 22 datos.
    \item Los datos desde la posición 23 hasta la 42 corresponden al valor $X=7$.
\end{itemize}
Por lo tanto, tanto el dato en la posición 25 como el dato en la posición 26 son $7$.
La mediana es $M_e = \frac{7+7}{2} = 7$. \\
\textbf{Respuesta:} La mediana es 7.

\vspace{1em}
\noindent De acuerdo a la siguiente muestra \(a+2, a+4, a+6, a+6, a+6, a+4, a+2\), la suma de la mediana y la moda es: \\
\textit{Solución:}
Muestra ordenada: \(a+2, a+2, a+4, a+4, a+6, a+6, a+6\). $n=7$.
\begin{itemize}
    \item \textbf{Moda:} El dato más frecuente es \(a+6\) (frecuencia 3). $M_o = a+6$.
    \item \textbf{Mediana:} Como $n=7$ (impar), la mediana es el dato en la posición $\frac{7+1}{2}=4$. El cuarto dato es \(a+4\). $M_e = a+4$.
\end{itemize}
\(\textbf{Suma} = M_o + M_e = (a+6) + (a+4) = 2a+10\)\\
\textbf{Respuesta:} $2a+10$.

\vspace{1em}
\noindent Los datos de una muestra son todos números naturales consecutivos, si no hay ningún dato repetido y la mediana de la muestra es 11.5, entonces ¿Qué cantidad de datos no puede ser?
\textit{Solución:}
La mediana es 11.5, lo que significa que es el promedio de los dos datos centrales. Esto solo ocurre cuando el número de datos ($n$) es par.
Sean los dos datos centrales $x_k$ y $x_{k+1}$ (donde $k=n/2$). Como son consecutivos y no repetidos, $x_{k+1} = x_k + 1$.
La mediana es $\frac{x_k + x_{k+1}}{2} = \frac{x_k + (x_k+1)}{2} = \frac{2x_k+1}{2} = x_k + 0.5$.
Se nos dice que la mediana es 11.5, entonces $x_k + 0.5 = 11.5$, lo que implica $x_k = 11$. El siguiente dato consecutivo es $x_{k+1} = 12$.
Los dos datos centrales son 11 y 12.
Para que estos sean los datos centrales, la cantidad de datos $n$ debe ser par.
Si $n$ fuera impar, la mediana sería uno de los datos (un número natural), no 11.5.
Por lo tanto, la cantidad de datos no puede ser impar. \\
\textbf{Respuesta:} La cantidad de datos no puede ser un número impar.
\newpage

\section{04/04}
\subsection{Población y Muestra}
¿Qué inconvenientes puede implicar realizar un censo?
\begin{itemize}
    \item Cardinalidad (tamaño) de la población: Puede ser muy grande o infinita. % Added clarification
    \item Destrucción de los objetos de estudio (ej. control de calidad destructivo). % Added example
    \item Costos asociados (tiempo, dinero, recursos humanos). % Added details
    \item Dificultad de acceso a toda la población. % Added another point
\end{itemize}

\subsection{Muestreo}
Proceso de diseñar e implementar mecanismos para escoger los elementos que conformarán la muestra. \\
\textbf{Es fundamental que la muestra esté bien escogida para realizar una inferencia estadística válida.} % Added "válida"

\subsection{Muestra}
\subsubsection{Representatividad}
Para que una muestra sea representativa, debe reflejar las características relevantes de la población. Claves para lograrlo:
\begin{itemize}
    \item El \textbf{tamaño} de la muestra ($n$): Se abordará más adelante (criterios probabilísticos). % Added ($n$)
    \item \textbf{Aleatoriedad:} El mecanismo de selección debe asegurar que todos los elementos (o grupos) tengan una probabilidad conocida (y a menudo igual) de ser seleccionados. % Clarified definition
\end{itemize}
Por lo general designaremos la letra \textbf{N} para la cardinalidad de la población y \textbf{n} para la cardinalidad de la muestra. % Corrected "cardinalidad de la muestra"

\subsection{Tipos de Muestreo Probabilístico} % Added heading for clarity
\subsubsection{Muestreo Aleatorio Simple (M.A.S)}
Una \textbf{M.A.S} de tamaño \textbf{n} sujetos se selecciona de modo que cada muestra posible del mismo tamaño $n$ tiene la misma probabilidad de ser elegida. Requiere un listado completo de la población (marco muestral).

\textbf{\textit{Ejemplos:}}
\begin{itemize}
    \item Seleccionar 200 pacientes al azar de una lista completa de registros médicos de un hospital.
    \item Seleccionar al azar 500 estudiantes de media de una lista nacional sin agruparlos por colegio.
    \item Elegir 100 tornillos de una producción sin distinguir lotes o turnos de fabricación (asumiendo homogeneidad).
\end{itemize}

\subsubsection{Muestreo Estratificado}
Se utiliza cuando la población no es homogénea respecto a la variable de estudio, pero puede dividirse en subgrupos (estratos) que sí son homogéneos internamente.
\begin{enumerate}
    \item Se divide la población en estratos mutuamente excluyentes y colectivamente exhaustivos.
    \item Se selecciona una muestra aleatoria simple (u otro método probabilístico) dentro de cada estrato.
    \item Mayor precisión si la variabilidad dentro de los estratos es baja y entre estratos es alta. % Corrected description
\end{enumerate}

\subsubsection*{Muestreo Estratificado Proporcional} % Changed to subsubsection* for formatting
\begin{itemize}
    \item El número de elementos extraído de cada estrato ($n_i$) es proporcional al tamaño relativo del estrato en la población ($N_i/N$).
    \item Se utiliza cuando el propósito principal es estimar parámetros poblacionales con buena representatividad global. % Corrected typo "pobalcionales"
    \item La fracción de muestreo es la misma en todos los estratos: $n_i = n \cdot \frac{N_i}{N}$, donde $N$ es el tamaño de la población, $N_i$ el tamaño del estrato $i$, y $n$ el tamaño total de la muestra ($n = \sum n_i$). % Corrected "pobalcion", added sum clarification
\end{itemize}

\subsubsection*{Muestreo Estratificado No Proporcional (Óptimo)} % Changed to subsubsection*
\begin{enumerate}
    \item El número de elementos incluidos en la muestra de cada estrato ($n_i$) no es proporcional a $N_i/N$. A menudo se asigna mayor tamaño muestral a estratos con mayor variabilidad interna o menor costo de muestreo. % Clarified purpose
    \item Los elementos de la población no tienen la misma probabilidad global de ser incluidos en la muestra (a menos que se usen pesos). % Clarified consequence
    \item No garantiza la equiprobabilidad inicial, pero puede ser más eficiente para estimar parámetros si se pondera adecuadamente en el análisis. Requiere conocimiento previo de la variabilidad o costos dentro de los estratos. % Corrected typo "equiporbabilidad", "pobalcion", "desproprorcion", "uetsra", "conocimeinto"
\end{enumerate}

\subsubsection{Muestreo por Conglomerados}
\begin{itemize}
    \item Se utiliza cuando la población está dividida naturalmente en grupos (conglomerados), como ciudades, escuelas, bloques de viviendas.
    \item Se escogen algunos conglomerados al azar.
    \item Se muestrean \textbf{todos} los individuos dentro de los conglomerados seleccionados (muestreo en una etapa) o se realiza un muestreo adicional dentro de los conglomerados seleccionados (muestreo en varias etapas).
    \item \textbf{Idealmente, cada conglomerado debe ser internamente heterogéneo (una mini-representación de la población)} y los conglomerados deben ser similares entre sí. Es más eficiente si la variabilidad \textit{dentro} de los conglomerados es alta y \textit{entre} conglomerados es baja (opuesto al estratificado). % Corrected typo "elemnmtos", "pobalcion", "separdaos", "naturalmnte", "cirtitrios", "geograficpos", "agrupcaiones", "fundfamental", "muiestreo"
\end{itemize}

\subsubsection{Muestreo Aleatorio Sistemático}
\begin{enumerate}
    \item Se utiliza cuando se dispone de una lista ordenada de la población.
    \item Se elige un punto de partida aleatorio ($a$) entre 1 y $k$.
    \item Se selecciona cada $k$-ésimo elemento de la lista, donde $k = N/n$ (intervalo de muestreo). Los elementos seleccionados son $a, a+k, a+2k, \dots$.
    \item Si la lista está ordenada respecto a la variable de interés, puede ser más preciso que un M.A.S. Cuidado si hay periodicidad en la lista que coincida con $k$. % Corrected typo "pobalcion"
\end{enumerate}

\subsection{Muestreos No Aleatorios (No Probabilísticos)} % Added heading for clarity
\subsubsection{Muestreo por Cuotas}
\begin{itemize} % Added missing \begin{itemize}
    \item Técnica común en estudios de mercado y sondeos de opinión. No es probabilístico.
    \item La población se divide en grupos según características (sexo, edad, etc.).
    \item Se fija una cuota de individuos a entrevistar para cada grupo.
    \item La selección de los individuos dentro de cada grupo queda a criterio del entrevistador (no es aleatoria). % Corrected typo "estduios", "pobalcion", "divindida"
\end{itemize}

\subsubsection{Muestreo Bola de Nieve}
\begin{enumerate}
    \item Indicado para estudiar poblaciones difíciles de localizar o contactar (minoritarias, clandestinas, dispersas pero conectadas). % Corrected typo "pobalciones"
    \item Se contacta a unos pocos individuos iniciales.
    \item Estos individuos iniciales ayudan a localizar y contactar a otros miembros de la población, y así sucesivamente.
\end{enumerate}

\subsubsection{Muestreo por Juicio (o Intencional)}
\begin{itemize}
    \item La selección de la muestra se basa en el juicio o criterio del investigador, quien elige a los individuos que considera más representativos o apropiados para el estudio, basándose en su experiencia. % Corrected typo "exxperiencia", "porfesional"
\end{itemize}

\subsection{Preguntas}
\begin{enumerate}
    \item ¿Cuándo ocupar un muestreo estratificado en vez de uno por conglomerados?
    \textit{Respuesta:} Usar \textbf{estratificado} cuando la población es heterogénea globalmente pero se puede dividir en estratos homogéneos internamente (baja varianza intra-estrato, alta varianza inter-estrato). El objetivo es asegurar representación de cada estrato y aumentar precisión. Usar \textbf{conglomerados} cuando la población está agrupada naturalmente, es costoso acceder a individuos dispersos, y los conglomerados son internamente heterogéneos (alta varianza intra-conglomerado, baja varianza inter-conglomerado). El objetivo es la eficiencia operativa/costos.
    \item ¿En qué se diferencia un muestreo por cuotas de un muestreo estratificado?
    \textit{Respuesta:} Ambos dividen la población en grupos. La diferencia clave está en la selección \textit{dentro} de los grupos: en el \textbf{estratificado} (probabilístico), se usa un método aleatorio (como M.A.S.) dentro de cada estrato. En el de \textbf{cuotas} (no probabilístico), la selección dentro de cada grupo la realiza el entrevistador según su conveniencia o juicio, hasta completar la cuota asignada. El estratificado permite inferencia estadística formal, el de cuotas no.
\end{enumerate}
\newpage

\section{10/04}
\textbf{\textit{\underline{Obj:} Aplicar y comprender propiedades de las medidas de dispersión}}
\subsection{Medidas de Dispersión}
Las medidas de tendencia central no son suficientes para describir un conjunto de datos. Consideremos dos conjuntos con la misma media $\bar{x}=0$:
\[ A = \{-4, 4, -4, 4\} \]
\[ B = \{7, 1, -6, -2\} \]
Ambos tienen $\bar{x}=0$, pero los datos en $A$ están más concentrados alrededor de la media que en $B$. Las medidas de dispersión cuantifican esta variabilidad.

\subsubsection{Rango:}
Se define como la diferencia entre el valor máximo y el valor mínimo de los datos.
\[ Rango = x_{max} - x_{min} \]
Es sensible a valores extremos.

\subsubsection{Desviación Media:}
Dada la variable $X$, con $n$ datos $x_1, x_2, \dots, x_n$ y media $\bar{x}$. Se define la desviación media como el promedio de las desviaciones absolutas respecto a la media:
\[ DM = \frac{|x_1-\bar{x}|+|x_2-\bar{x}|+\dots+|x_n-\bar{x}|}{n} = \frac{\sum_{i=1}^{n} |x_i - \bar{x}|}{n} \] % Corrected typo "Vatiable", "deviacion"

\subsubsection{Varianza ($\sigma^2$ o $s^2$):} % Added s^2 for sample variance notation
Es el promedio de las desviaciones al cuadrado respecto a la media. Es la medida de dispersión más utilizada junto con su raíz cuadrada.
\[ \sigma^2 = \frac{(x_1-\bar{x})^2+(x_2-\bar{x})^2+\dots+(x_n-\bar{x})^2}{n} = \frac{\sum_{i=1}^{n} (x_i - \bar{x})^2}{n} \]
Nota: Esta es la varianza \textit{poblacional} ($\sigma^2$). La varianza \textit{muestral} insesgada ($s^2$) usa $n-1$ en el denominador. En este curso parece usarse la fórmula con $n$. % Added clarification on population vs sample variance

\subsubsection{Desviación Estándar (o típica) ($\sigma$ o $s$):}
Es la raíz cuadrada positiva de la varianza. Tiene las mismas unidades que los datos originales.
\[ \sigma = \sqrt{\sigma^2} = \sqrt{\frac{\sum_{i=1}^{n} (x_i - \bar{x})^2}{n}} \] % Corrected typo "desvicon"

\subsubsection{Propiedades de $\sigma$ y $\sigma^2$}
\begin{center}
    \begin{enumerate}
        \item $\sigma \ge 0$ y $\sigma^2 \ge 0$. Son siempre no negativas. % Clarified notation
        \item $\sigma = 0 \iff x_i = \bar{x}$ para todo $i \iff x_i = x_j$ para todo $i, j \in \{1, \dots, n\}$. La desviación estándar (y varianza) es cero si y sólo si todos los datos son iguales. % Corrected symbols and wording, "dtos" -> "datos", "igaul" -> "igual"
        \item Si a todos los datos se les suma una constante $k$ (transformación $y_i = x_i + k$), la media cambia ($\bar{y} = \bar{x} + k$) pero la dispersión no: $\sigma_y = \sigma_x$ y $\sigma_y^2 = \sigma_x^2$. % Clarified property
        \item Si todos los datos se multiplican por una constante $k$ (transformación $y_i = k \cdot x_i$), la media se multiplica por $k$ ($\bar{y} = k\bar{x}$) y la dispersión cambia: $\sigma_y = |k| \cdot \sigma_x$ y $\sigma_y^2 = k^2 \cdot \sigma_x^2$. % Corrected typo "variabe", added relation for variance
        \item Fórmula computacional para la varianza: $\sigma^2 = \frac{\sum x_i^2}{n} - (\bar{x})^2 = \overline{x^2} - (\bar{x})^2$. Es decir, la varianza es la media de los cuadrados menos el cuadrado de la media. % Corrected formula notation
        \item $\sigma^2=\sigma \iff \sigma=0 \vee \sigma=1$
        \item $\sigma^2 < \sigma \iff 0<\sigma<1 $
        \item $\sigma^2 > \sigma \iff \sigma>1$
    \end{enumerate}
\end{center}
\newpage

\section{16/04}
\subsection{Demostración Propiedad 4}
Sea $Y$ la variable definida por $y_i = k \cdot x_i$. Sabemos que $\bar{y} = k \cdot \bar{x}$.
La varianza de $Y$ es:
\[\sigma_y^2=\frac{\sum_{i=1}^{n} (y_i - \bar{y})^2}{n} \]
Sustituyendo $y_i$ y $\bar{y}$:
\[\sigma_y^2=\frac{\sum_{i=1}^{n} (kx_i - k\bar{x})^2}{n}\]
Factorizando $k$ dentro del paréntesis:
\[\sigma_y^2=\frac{\sum_{i=1}^{n} [k(x_i - \bar{x})]^2}{n}\]
\[\sigma_y^2=\frac{\sum_{i=1}^{n} k^2(x_i - \bar{x})^2}{n}\]
Sacando $k^2$ de la sumatoria (es constante):
\[\sigma_y^2=k^2 \cdot \frac{\sum_{i=1}^{n} (x_i - \bar{x})^2}{n}\]
Reconociendo la definición de $\sigma_x^2$:
\[\sigma_y^2=k^2 \cdot \sigma_x^2\]
Tomando la raíz cuadrada (positiva, ya que $\sigma$ es no negativa):
\[\sigma_y = \sqrt{k^2 \cdot \sigma_x^2} = \sqrt{k^2} \cdot \sqrt{\sigma_x^2} = |k| \cdot \sigma_x\]

\subsection{Ejercicio}
Dados los datos -2, 0, 2, 4, 6. Determinar:
\begin{enumerate}
    \item $\bar{x}$ \\
        R. $\bar{x} = \frac{-2+0+2+4+6}{5} = \frac{10}{5} = 2$ \\
    \item $\sigma$ \\
        R. Calculamos la varianza primero:
        $\sigma^2 = \frac{(-2-2)^2+(0-2)^2+(2-2)^2+(4-2)^2+(6-2)^2}{5}$
        $\sigma^2 = \frac{(-4)^2+(-2)^2+(0)^2+(2)^2+(4)^2}{5}$
        $\sigma^2 = \frac{16+4+0+4+16}{5} = \frac{40}{5} = 8$ \\
        Ahora la desviación estándar: $\sigma = \sqrt{8} = \sqrt{4 \cdot 2} = 2\sqrt{2}$ \\
    \item $\overline{x^2}$ (el promedio de los cuadrados de los datos) \\
        R. Cuadrados de los datos: $(-2)^2=4, 0^2=0, 2^2=4, 4^2=16, 6^2=36$.
        $\overline{x^2} = \frac{4+0+4+16+36}{5} = \frac{60}{5} = 12$ \\
    \item Calcular $\overline{x^2}-(\bar{x})^2$ \\
        R. $\overline{x^2}-(\bar{x})^2 = 12 - (2)^2 = 12 - 4 = 8$. \\
        (Note que esto coincide con $\sigma^2$, como afirma la propiedad 5).
\end{enumerate}

\subsection{Demostración Propiedad 5}
Partimos de la definición de varianza:
\[\sigma^2=\frac{\sum_{i=1}^{n} (x_i-\bar{x})^2}{n}\]
Expandimos el cuadrado:
\[\sigma^2=\frac{\sum_{i=1}^{n} (x_i^2 - 2x_i\bar{x} + (\bar{x})^2)}{n}\]
Separamos la sumatoria:
\[\sigma^2=\frac{\sum x_i^2 - \sum 2x_i\bar{x} + \sum (\bar{x})^2}{n}\]
\[\sigma^2=\frac{\sum x_i^2}{n} - \frac{\sum 2x_i\bar{x}}{n} + \frac{\sum (\bar{x})^2}{n}\]
Sacamos las constantes de las sumatorias ($2\bar{x}$ y $(\bar{x})^2$ son constantes respecto a $i$):
\[\sigma^2=\frac{\sum x_i^2}{n} - \frac{2\bar{x} \sum x_i}{n} + \frac{n(\bar{x})^2}{n}\]
Reconocemos: $\frac{\sum x_i^2}{n} = \overline{x^2}$ y $\frac{\sum x_i}{n} = \bar{x}$:
\[\sigma^2=\overline{x^2} - 2\bar{x} (\bar{x}) + (\bar{x})^2\]
\[\sigma^2=\overline{x^2} - 2(\bar{x})^2 + (\bar{x})^2\]
\[\sigma^2=\overline{x^2} - (\bar{x})^2\]
\newpage

    \section{23/04}
        \subsection{Demostraciones}
            \subsubsection{Propiedad 6}
                \[\sigma^2=\sigma \]
                \[\sigma^2-\sigma=0 \]
                \[\sigma(\sigma-1)=0 \]
                \[\sigma=0 \vee \sigma=1 \]
            \subsubsection{Propiedad 7}
                \[\sigma^2<\sigma \]
                \[\sigma^2-\sigma<0 \]
                \[\sigma(\sigma-1)<0 \]
                \begin{center}
                    \begin{tabular}{ccccc}
                        \phantom{ppp}  & $-\infty$ & $0$ & $1$ & $+\infty$ \\
                    \end{tabular}

                    \begin{tabular}{|c|c|c|c|c|} 
                         \hline
                        $\sigma$ & - & + & + \\ 
                        \hline
                        $\sigma-1$ & - & - & + \\ 
                        \hline 
                         & + &- & + \\
                        \hline                        
                    \end{tabular}
                \end{center}
                \[0<\sigma<1 \]
            \subsubsection{Propiedad 8}
                \[\sigma^2>\sigma \]
                \[\sigma^2-\sigma>0 \]
                \[\sigma(\sigma-1)>0 \]
                \begin{center}
                    \begin{tabular}{ccccc}
                        \phantom{ppp}  & $-\infty$ & $0$ & $1$ & $+\infty$ \\
                    \end{tabular}

                    \begin{tabular}{|c|c|c|c|c|} 
                         \hline
                        $\sigma$ & - & + & + \\ 
                        \hline
                        $\sigma-1$ & - & - & + \\ 
                        \hline 
                         & + &- & + \\
                        \hline                        
                    \end{tabular}
                \end{center}

                Por: $\sigma\ge0$
                \[\sigma>1\]
                \newpage

    \section{}

%puedes revisar este documento
                %add a list

\end{document}
