\documentclass[12pt, letterpaper]{article}
\usepackage{graphicx}
\graphicspath{ {./images/} }
\usepackage{eso-pic}

\AddToShipoutPicture*{%
    \put(1cm, \LenToUnit{\paperheight-5cm}){%
        \makebox[0pt][l]{\includegraphics[width=3cm]{logo}}
    }
}

\title{ Guía N°1 Probabilidad y Estadística}
\author{Felipe Colli}
\date{31 de Marzo, 2024}

\begin{document}
\maketitle
\tableofcontents

\section{Actividad 1}
\subsection{¿Cuál es el origen lingüístico e histórico de la palabra Estadística?}
Proviene del alemán \textit{Statistik}, que a su vez se deriva del latín \textit{status} ("estado","condición"). Originalmnete esta tenía que ver con el analisis de datos del Estado, como poblacioón, impuestos, recursos, entre otros.

\subsection{Identifica o investiga al menos 5 obejtivos de la Estadística en la actualidad. ¿Cómo se relacionan estos objetivos entre sí?}
\begin{enumerate}
    \item \textbf{Recopilar datos:} Obtener información relevante de manera sistemática.
    \item \textbf{Describir la información:} Resumir los datos usando medidas estadísticas y representaciones gráficas.
    \item \textbf{Analizar patrones:} Identificar relaciones, tendencias y comportamientos en los datos.
    \item \textbf{Tomar decisiones:} Basarse en los datos para elegir acciones o políticas.
    \item \textbf{Predecir resultados:} Estimar eventos futuros mediante modelos estadísticos.
\end{enumerate}
Estos objetivos forman un proceso cíclico:

\begin{itemize}
    \item Se \textbf{recopilan} datos para poder \textbf{describirlos}.
    \item Esa descripción permite \textbf{analizar patrones}.
    \item Con base en el análisis, se pueden \textbf{tomar decisiones}.
    \item Luego, los modelos ayudan a \textbf{predecir el futuro}.
    \item Las predicciones generan nuevas preguntas, reiniciando el ciclo.
\end{itemize}


\subsection{¿Por qué la Estadística es importante en la toma de decisiones, tanto en la vida cotidiana como el ámbito profesional?}
La estadística es fundamental ya que nos permite tomar decisiones basadas en datos reales, y no es suposiciones/intuiciones. Esto tambien se aplica en el ámbito profesional, donde cumple un rol importante en la toma de decisiones.

\subsection{Investiga sobre el trabajo de John Snow en torno a un brote de cólera en Londres en el siglo \textit{$XIX$}. ¿Cómo se relaciona este trabajo con los objetivos actuales de la Estadística?}
John Snow demostró que el cólera se transmitia por el consumo de aguas contaminadas con material fecal, ya que comprobo que la mayor parte de los casos se agrupaban en zonas donde el agua se contaminaba con facilidad, en el Londres de 1854. El trabajo de John Snow se relaciona esterchamente con los objetivos de la estadística, al ser este uno de los ejemplos mas temparanos del uso del metodo geográfico para la descripcion de los casos de una epidemia.

\section{Actividad 2}
\subsection{Señalar múltiples ejemplos de variables cuantitativas y discutir si son discretas o continuas.}
\begin{itemize}
    \item \textbf{Numero de alummnos en un curso (Discreta):} -No se puede partir a alguien a la mitad.
    \item \textbf{Altura de los alummnos en un curso (Continua):} -Puede medir un 1,50 metros, o 1.515 metros. 
    \item \textbf{Errores en el codigo (Continua):} -No exixte medio error en la programacion, solo warnings.
\end{itemize}


\subsection{¿Puede una variable discreta tomar valores racioanles ("con deciamles")? Por ejemplo, una variable que solo puede adoptar los valores $1,0; 1,1; 1,2$ y $1,3$, ¿es discreta o continua?}
Es discreta. La característica definitoria de una variable discreta no es que sus valores sean enteros, sino que sean contables y separados, es decir, que entre dos valores consecutivos que la variable puede tomar, no existe ningún otro valor posible para esa variable.


\subsection{Señalar múltiples ejemplos de variables cualitativas, distinguiendo entre nominales y ordinales.}
\textbf{Nominales:}
\begin{itemize}
    \item \textbf{Género}: Masculino, Femenino, No binario
    \item \textbf{Color de auto:} Rojo, azul, negro, gris 
    \item \textbf{Tipo de pelo:} Largo, corto, rizado, lacio
\end{itemize}

\textbf{Ordinales:}
\begin{itemize}
    \item \textbf{Calificacion de un restaurante:} Una, 2 o las 3 estrellas Michelin
    \item \textbf{Calidad de un candidato:} Buena, media, mala
    \item \textbf{Nivel Educativo:} Básica, Media, Superior, Posgrado, Completa/incompleta.
\end{itemize}

\subsection{¿Puede una variable cualitativa adoptar valores numéricos? Buscar y discutir ejemplos.}
Sí, una variable cualitativa puede representarse mediante valores numéricos, pero estos números actúan como códigos o etiquetas, tales como los códigos postales on la escala Likert


\section{Actividad 3:} 
\textbf{Para cada una de las siguientes situaciones, identifica la población de interés, la variable estadística y la clasificación de ésta.}

\begin{enumerate}
    \item Un investigador universitario desea estimar el nivel de riesgo que están dispuestos a aceptar ciudadamos chilenos de la \textit{"Generación X"} al iniciar sus propios negocios.
        \begin{enumerate}
            \item \textbf{Poblacion:} Chilenos de Gen X
            \item \textbf{Variable:} Quiere inicar o no un negocio (cualitativa nominal) \\
        \end{enumerate}
 
    \item Durante más de un siglo, la tremperatura corporal normal en seres humanos ha sido aceptada como 37°C. ¿Es así realmente? Los investigadores desean estimar el promedio de temperatura de adultos sanos en Chile.
        \begin{enumerate}
            \item \textbf{Poblacion:} Adultos Sanos
            \item \textbf{Variable:} Temperatura Corporal (cuantitativa nominal) \\
        \end{enumerate}

    \item Un ingeniero municipal desea estimar el promedio de consumo semanal de agua para unidades habitacionales unifamiliares en la ciudad.
        \begin{enumerate}
            \item \textbf{Poblacion:} Unidades Habitacionales de la ciudad
            \item \textbf{Variable:} Consumo semanal de agua (cuantitativa nominal) \\
        \end{enumerate}
    \item El National Highway Safety Council desea estimar la proporción de llantas para automóvil con dibujo o superficie de rodadura insegura, entre todas las llantas manufacturadas por una empresa específica durante el presente año de producción.
        \begin{enumerate}
            \item \textbf{Poblacion:} Llantas de Empresa X este año de produccion
            \item \textbf{Variable:} Cuantos autos udan esa llanta (cuantitativa nominal) \\
        \end{enumerate}

    \item Un politólogo desea estimar si la mayoría de los residentes adultos de una región están a favor de una legislatura unicameral.
        \begin{enumerate}
            \item \textbf{Poblacion:} Residente adultos de la región
            \item \textbf{Variable:} Están a favor de la legislatura (cualitativa nominal) \\
        \end{enumerate}

    \item Un científico del área médica desea determinar el tiempo promedio para que se vuelva a presentar cierta enfermedad infecciosa, una vez que las personas se recuperan de ella por primera vez.
        \begin{enumerate}
            \item \textbf{Poblacion:} Personas que ya se enfermaron una vez
            \item \textbf{Variable:} Tiempo hasta una 2° infección (cuantitativa nominal) \\
        \end{enumerate}

    \item Un ingeniero electricista desea determinar si el promedio de vida útil de transistores de cierto tipo es mayor que 500 horas.
        \begin{enumerate}
            \item \textbf{Poblacion:} Transistores del tipo X
            \item \textbf{Variable:} La vida util es $>$ 500 horas (cualitativa nominal) \\
        \end{enumerate}
\end{enumerate}

\section{Actividad 4:}
\textbf{Haz lo mismo que en la actividad anterior: a partir de los siguientes títulos de papers, determina o infiere la o variables estudiadas y las poblaciones. Clasificar las variables estadísticas.}

\begin{enumerate}

\item \textbf{Efectos del cambio climático en la biodiversidad de insectos en los bosques tropicales de América del Sur} \\ 
\textbf{Población(es):} Ecosistemas/Comunidades de insectos en bosques tropicales de América del Sur. \\ 
\textbf{Variables:}
\begin{itemize}
  \item Indicador de cambio climático: Cuantitativa Continua
  \item Medida de biodiversidad: Cuantitativa Continua o Discreta
  \item Tipo de bosque tropical: Cualitativa Nominal
  \item Grupo taxonómico de insectos: Cualitativa Nominal
\end{itemize}

\item \textbf{Relación entre la contaminación del aire y la tasa de mortalidad en comunidades urbanas de China} \\ 
\textbf{Población(es):} Comunidades urbanas en China. \\ 
\textbf{Variables:}
\begin{itemize}
  \item Nivel de contaminación del aire: Cuantitativa Continua
  \item Tasa de mortalidad: Cuantitativa Continua o Discreta
  \item Características de la comunidad: Cualitativa Nominal/Ordinal o Cuantitativa
  \item Periodo de tiempo: Cualitativa Ordinal o Cuantitativa Discreta
\end{itemize}

\item \textbf{Impacto de la acidificación oceánica en el crecimiento de corales en el Mar Caribe} \\ 
\textbf{Población(es):} Colonias o especies de coral en el Mar Caribe. \\ 
\textbf{Variables:}
\begin{itemize}
  \item Nivel de acidificación oceánica: Cuantitativa Continua
  \item Tasa de crecimiento del coral: Cuantitativa Continua
  \item Especie de coral: Cualitativa Nominal
  \item Temperatura del agua: Cuantitativa Continua
  \item Profundidad: Cuantitativa Continua
\end{itemize}

\item \textbf{Variabilidad genética en poblaciones de lobos ibéricos en la península ibérica} \\ 
\textbf{Población(es):} Poblaciones de lobo ibérico. \\ 
\textbf{Variables:}
\begin{itemize}
  \item Medida de variabilidad genética: Cuantitativa Continua
  \item Marcadores genéticos: Cualitativa Nominal
  \item Ubicación geográfica: Cualitativa Nominal
  \item Tamaño estimado de población: Cuantitativa Discreta/Continua
\end{itemize}

\item \textbf{Distribución y abundancia de microplásticos en peces comerciales del Golfo de México} \\ 
\textbf{Población(es):} Peces comerciales en el Golfo de México. \\ 
\textbf{Variables:}
\begin{itemize}
  \item Presencia/Ausencia de microplásticos: Cualitativa Nominal
  \item Cantidad/Concentración de microplásticos: Cuantitativa Discreta o Continua
  \item Tipo de microplástico: Cualitativa Nominal
  \item Especie de pez: Cualitativa Nominal
  \item Tamaño/Peso del pez: Cuantitativa Continua
  \item Zona de captura: Cualitativa Nominal
\end{itemize}

\item \textbf{Prevalencia de diabetes tipo 2 en adultos mayores de 60 años en España} \\ 
\textbf{Población:} Adultos mayores de 60 años en España. \\ 
\textbf{Variables:}
\begin{itemize}
  \item Diagnóstico de diabetes tipo 2: Cualitativa Nominal
  \item Edad: Cuantitativa Discreta o Continua
  \item Género: Cualitativa Nominal
  \item Región de residencia: Cualitativa Nominal
\end{itemize}

\item \textbf{Eficacia de la terapia cognitivo-conductual en pacientes con trastorno de ansiedad en Argentina} \\ 
\textbf{Población:} Pacientes con trastorno de ansiedad. \\ 
\textbf{Variables:}
\begin{itemize}
  \item Tipo de tratamiento: Cualitativa Nominal
  \item Severidad de ansiedad: Cuantitativa Ordinal o Continua
  \item Resultado del tratamiento: Cualitativa Ordinal o Nominal
  \item Tipo de centro de salud: Cualitativa Nominal
  \item Características del paciente: Cuantitativa/Cualitativa
\end{itemize}

\item \textbf{Relación entre el consumo de ultraprocesados y la obesidad infantil en EE.UU.} \\ 
\textbf{Población:} Niños de 6 a 12 años en EE.UU. \\ 
\textbf{Variables:}
\begin{itemize}
  \item Nivel de consumo de ultraprocesados: Cuantitativa Continua o Cualitativa Ordinal
  \item Estado nutricional/IMC: Cualitativa Ordinal o Cuantitativa Continua
  \item Edad: Cuantitativa Discreta
  \item Género: Cualitativa Nominal
  \item Nivel socioeconómico: Cualitativa Ordinal o Cuantitativa
  \item Nivel de actividad física: Cualitativa Ordinal o Cuantitativa Continua
\end{itemize}

\item \textbf{Impacto del ejercicio físico en la presión arterial en mujeres postmenopáusicas en Brasil} \\ 
\textbf{Población:} Mujeres postmenopáusicas en Brasil. \\ 
\textbf{Variables:}
\begin{itemize}
  \item Tipo/Intensidad/Duración del ejercicio: Cualitativa o Cuantitativa
  \item Presión arterial (sistólica y diastólica): Cuantitativa Continua
  \item Cambio en la presión arterial: Cuantitativa Continua
  \item Edad: Cuantitativa Continua/Discreta
  \item Uso de medicación: Cualitativa Nominal
\end{itemize}

\item \textbf{Prevalencia del uso de antibióticos sin prescripción en adolescentes en México} \\ 
\textbf{Población:} Adolescentes en México. \\ 
\textbf{Variables:}
\begin{itemize}
  \item Uso de antibióticos sin prescripción: Cualitativa Nominal
  \item Frecuencia de uso: Cualitativa Ordinal o Cuantitativa Discreta
  \item Tipo de antibiótico: Cualitativa Nominal
  \item Motivo de uso: Cualitativa Nominal
  \item Edad: Cuantitativa Discreta
  \item Género: Cualitativa Nominal
  \item Nivel socioeconómico/educativo: Cualitativa Ordinal o Cuantitativa
\end{itemize}

\item \textbf{Nivel de satisfacción laboral en trabajadores del sector tecnológico en Japón} \\ 
\textbf{Población:} Trabajadores del sector tecnológico en Japón. \\ 
\textbf{Variables:}
\begin{itemize}
  \item Nivel de satisfacción laboral: Cualitativa Ordinal
  \item Edad: Cuantitativa Continua/Discreta
  \item Género: Cualitativa Nominal
  \item Años de experiencia: Cuantitativa Continua/Discreta
  \item Tipo de puesto: Cualitativa Nominal
  \item Salario: Cuantitativa Continua
  \item Tamaño de la empresa: Cualitativa Ordinal o Nominal
\end{itemize}

\item \textbf{Efecto del nivel socioeconómicoen el rendimiento académico de estudiantes universitarios en Chile} \\ 
\textbf{Población:} Universitarios Chilenos. \\ 
\textbf{Variables:}
\begin{itemize}
  \item Nivel socioeconómico: Cualitativa Ordinal
  \item Rendimiento académico: Cuantitativa Continua
  \item Tipo de Univeridad: Cualitativa Nominal
  \item Tipo de financiamiento: Cualitativa Nominal
\end{itemize}


\item \textbf{Influencia del uso de redes sociales en la autoestima de adolescentes en España} \\ 
\textbf{Población:} Adolescentes residentes en España. \\ 
\textbf{Variables:}
\begin{itemize}
  \item Uso de redes sociales: Cuantitativa Continua
  \item Autoestima: Cuantitativa Continua o Ordinal
  \item Red Social Utilizada: Cualitativa nominal
\end{itemize}


\item \textbf{Factores que afectan la participación política en jóvenes entre 18 y 25 años en Alemania} \\ 
\textbf{Población:} Jovenes entre 18 y 25 años Alemanes. \\ 
\textbf{Variables:}
\begin{itemize}
  \item Participacion Politica: Cuantitativa Discreta
  \item Nivel de educacion: Cualitativa Ordinal
  \item Interes politico: Cualitativa Ordinales
  \item Acceso a Informacion Politica: Cualitativa Ordinal o Cuantitativa Continua
  \item Influencias sociales: Cualitativa Nomi o Ordinal
  \item Confianza en las instituciones políticas: Cuantitativa Ordinal
\end{itemize}

\item \textbf{Impacto de la migración en la percepción de identidad cultural en comunidades indígenas en Canadá} \\ 
\textbf{Población:} Comunidades indígenas en Canadá, específicamente migrantes o descendientes que han experimentado procesos migratorios. \\ 
\textbf{Variables:}
\begin{itemize}
  \item Percepción de identidad cultural: Cualitativa Ordinal o Cualitativa Nominal 
  \item Grado de migración o desplazamiento: Cualitativa Ordinal
  \item Conexión con las tradiciones culturales indígenas: Cualitativa Ordinal
\end{itemize}

\item \textbf{Relación entre el estrés laboral y la productividad en empleados del sector financiero en Reino Unido} \\ 
\textbf{Población:} Empleados del sector financiero en Reino Unido. \\ 
\textbf{Variables:}
\begin{itemize}
  \item Estrés laboral: Cuantitativa Continua  
  \item Productividad laboral: Cuantitativa Continua
  \item Tipo de puesto (junior, senior, directivo): Cualitativa Nominal.
\end{itemize}

\item \textbf{Desigualdad de género en la distribución de roles en el hogar en familias de clase media en Francia} \\ 
\textbf{Población:} Familias de clase media residentes en Francia. \\ 
\textbf{Variables:}
\begin{itemize}
  \item Género de los miembros del hogar: Cualitativa Nominal   
  \item Distribución de roles domésticos (tareas del hogar, cuidado de hijos, administración del hogar): Cuantitativa Discreta
  \item Percepción de equidad en la distribución de tareas: Cualitativa Ordinal
\end{itemize}

\item \textbf{Efectos de la urbanización en la cohesión social en barrios periféricos de Bogotá} \\ 
\textbf{Población:} Habitantes de barrios periféricos en Bogotá. \\ 
\textbf{Variables:}
\begin{itemize}
  \item Nivel de urbanización: Cuantitativa Continua 
  \item Cohesión social: Cuantitativa Continua 
  \item Nivel socioeconómico: Cualitativa Ordinales
  \item Participación en actividades comunitarias: Cuantitativa Discreta o Cualitativa Nominal
\end{itemize}

\item \textbf{Consumo de noticias falsas en redes sociales y su influencia en las creencias políticas en adultos jóvenes de EE.UU.} \\ 
\textbf{Población:} Adultos jóvenes (por ejemplo, entre 18 y 30 años) residentes en Estados Unidos. \\ 
\textbf{Variables:}
\begin{itemize}
  \item Consumo de noticias falsas en redes sociales: Cuantitativa Discreta y Cualitativa Ordinal
  \item Creencias políticas: Cualitativa Ordinal
  \item Nivel educativo: Cualitativa Ordinal
  \item Nivel de alfabetización mediática: Cualitativa Nominal o Cuantitativa
\end{itemize}


\item \textbf{Relación entre el acceso a la educación superior y la movilidad social en comunidades rurales de India} \\ 
\textbf{Población:} Individuos o familias que viven en comunidades rurales de India. \\ 
\textbf{Variables:}
\begin{itemize}
    \item Acceso a la educación superior: Cualitativa Ordinal
    \item Movilidad Social: Cuantitativa Continua
    \item Genero y Casta: Cualitativa Nominal
    \item Apoyo gubernamental: Cualitativa Nominal
    \item Nivel socioeconómico familiar: Cualitativa Ordinal o Cuantitativa Continua
\end{itemize}

\end{enumerate}


\end{document}
