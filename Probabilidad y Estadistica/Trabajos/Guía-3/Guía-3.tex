\documentclass[12pt, a4paper]{article} % Combinado: 12pt de Doc1, a4paper de ambos
\usepackage[utf8]{inputenc}
%\usepackage[spanish]{babel}
\usepackage{amsmath, amssymb, amsfonts} % Combinado y amsfonts añadido por si acaso (aunque amssymb suele cargarlo)
\usepackage{geometry}
\usepackage{graphicx} % De Doc 2
\usepackage{array}    % De Doc 2
\usepackage{booktabs} % De Doc 2
\usepackage{enumitem} % De Doc 2 (y útil para el enumerate de Doc 1)
\usepackage{siunitx}  % De Doc 2

% --- Configuraciones de Doc 1 ---
\geometry{a4paper, margin=1in} % Usamos la configuración de Doc 1

\title{Soluciones de Problemas de Estadística y Guía PSU} % Título combinado o modificado
\date{\today}
\author{Felipe Colli} % Tomado de Doc 2, o dejar vacío {} si se prefiere

% --- Configuraciones de Doc 2 ---
\sisetup{
    output-decimal-marker={,}, % Usar coma como separador decimal
    group-separator={.},       % Usar punto como separador de miles (si aplica)
    group-minimum-digits=4     % Agrupar a partir de 4 dígitos
}
\setlength{\parindent}{0pt} % Sin indentación de párrafo
\setlength{\parskip}{1.5ex} % Espacio entre párrafos

% --- Comandos personalizados de Doc 2 ---
\newcommand{\mean}[1]{\ensuremath{\bar{#1}}} % Comando para la media
\newcommand{\stdev}{\ensuremath{s}} % Comando para desviación estándar muestral
\newcommand{\pstdev}{\ensuremath{\sigma}} % Comando para desviación estándar poblacional
\newcommand{\var}{\ensuremath{s^2}} % Comando para varianza muestral
\newcommand{\pvar}{\ensuremath{\sigma^2}} % Comando para varianza poblacional

\begin{document}
\maketitle

% --- Contenido del Documento 1 ---
\begin{enumerate}
    \item[\bfseries 689)] La desviación estándar de los datos $4a, 4b$ y $4c$ es $0,16$. Sea $S$ la desviación estándar de los datos $a, b$ y $c$.
    Por la propiedad de la desviación estándar ante el escalamiento, la desviación estándar de $ka, kb, kc$ es $|k|S$.
    En este caso, $k=4$. Entonces, la desviación estándar de $4a, 4b, 4c$ es $4S$.
    Nos dan que $4S = 0,16$.
    Dividiendo por 4, obtenemos $S = \frac{0,16}{4} = 0,04$.
    La desviación estándar de los datos $a, b$ y $c$ es $0,04$.
    \textbf{Respuesta: B) 0,04}

    \item[\bfseries 690)] Se comparan dos muestras: mamuts y ratones.
    Mamuts: Peso promedio $\bar{x}_M = 7500$ kg. Desviación estándar $S_M$ desconocida.
    Ratones: Peso promedio $\bar{x}_R = 30$ g. Desviación estándar $S_R = 5$ g.
    Para comparar la dispersión (homogeneidad) de muestras con medias muy diferentes, se usa el Coeficiente de Variación (CV), $CV = S / |\bar{x}|$. Un menor CV indica mayor homogeneidad (menor dispersión relativa).
    $CV_{ratones} = \frac{5 \text{ g}}{30 \text{ g}} = \frac{1}{6} \approx 0,167$.
    No conocemos $S_M$, pero es biológicamente razonable asumir que la variación relativa del peso de los mamuts es menor. Por ejemplo, si $S_M = 500$ kg, $CV_{mamuts} = \frac{500 \text{ kg}}{7500 \text{ kg}} = \frac{1}{15} \approx 0,067$.
    Incluso con una desviación estándar grande en términos absolutos (ej. 1000 kg), el CV sería $1000/7500 = 2/15 \approx 0,133$, que sigue siendo menor que el CV de los ratones.
    Por lo tanto, es muy probable que $CV_{mamuts} < CV_{ratones}$. Esto significa que la muestra de los mamuts es más homogénea (menos dispersa relativamente) que la de los ratones.
    \textbf{Respuesta: B) La muestra de los mamuts es más homogénea que la de los ratones}

    \item[\bfseries 691)] Sean $a, b, c, d$ números positivos con varianza $\sigma^2$ y media $\bar{x}$. Buscamos la afirmación FALSA.
    A) Si $n > 0$, la varianza de $a+n, b+n, c+n, d+n$ es $\sigma^2$. La varianza es invariante ante traslaciones. La afirmación dice que es $\sigma^2 + n$, lo cual es FALSO.
    B) Si $a=b=c=d$, todos los datos son iguales, no hay dispersión, por lo tanto la varianza $\sigma^2 = 0$. VERDADERO.
    C) La varianza de $3a, 3b, 3c, 3d$ es $Var(3x_i) = 3^2 Var(x_i) = 9\sigma^2$. VERDADERO.
    D) Si $q > 0$, la media aritmética de $a+q, b+q, c+q, d+q$ es $Mean(x_i+q) = Mean(x_i) + q = \bar{x} + q$. VERDADERO.
    E) La varianza ($\sigma^2$) y la desviación estándar ($\sigma$) pueden ser iguales si $\sigma^2 = \sigma$, lo cual ocurre si $\sigma=0$ o $\sigma=1$. Es posible. VERDADERO.
\textbf{Respuesta: La A es la afirmación FALSA.}

    \item[\bfseries 692)] Datos: $(2p - 1), (2p + 1), (2p + 3), (2p + 5)$. N=4.
    Media: $\bar{x} = \frac{(2p-1) + (2p+1) + (2p+3) + (2p+5)}{4} = \frac{8p + 8}{4} = 2p + 2$.
    Varianza: Calculamos las desviaciones $(x_i - \bar{x})$:
    $(2p-1) - (2p+2) = -3$
    $(2p+1) - (2p+2) = -1$
    $(2p+3) - (2p+2) = 1$
    $(2p+5) - (2p+2) = 3$
    Desviaciones al cuadrado: $(-3)^2=9, (-1)^2=1, (1)^2=1, (3)^2=9$.
    Suma de cuadrados = $9 + 1 + 1 + 9 = 20$.
    Varianza $\sigma^2 = \frac{\text{Suma de cuadrados}}{N} = \frac{20}{4} = 5$.
    Desviación típica (estándar) $\sigma = \sqrt{\text{Varianza}} = \sqrt{5}$.
    \textbf{Respuesta: B) $(2p + 2)$ y $\sqrt{5}$}

    \item[\bfseries 693)] Conjunto: $\{n, n/2, 2n\}$, donde $n>0$. N=3.
    Media: $\bar{x} = \frac{n + n/2 + 2n}{3} = \frac{7n/2}{3} = \frac{7n}{6}$.
    Desviaciones $(x_i - \bar{x})$:
    $n - \frac{7n}{6} = -\frac{n}{6}$
    $\frac{n}{2} - \frac{7n}{6} = \frac{3n-7n}{6} = -\frac{4n}{6} = -\frac{2n}{3}$
    $2n - \frac{7n}{6} = \frac{12n-7n}{6} = \frac{5n}{6}$
    Desviaciones al cuadrado: $(-\frac{n}{6})^2 = \frac{n^2}{36}$, $(-\frac{2n}{3})^2 = \frac{4n^2}{9} = \frac{16n^2}{36}$, $(\frac{5n}{6})^2 = \frac{25n^2}{36}$.
    Suma de cuadrados = $\frac{n^2}{36} + \frac{16n^2}{36} + \frac{25n^2}{36} = \frac{(1+16+25)n^2}{36} = \frac{42n^2}{36} = \frac{7n^2}{6}$.
    Varianza $\sigma^2 = \frac{\text{Suma de cuadrados}}{N} = \frac{7n^2/6}{3} = \frac{7n^2}{18}$.
    Desviación estándar $\sigma = \sqrt{\frac{7n^2}{18}} = n \sqrt{\frac{7}{18}} = n \frac{\sqrt{7}}{\sqrt{18}} = n \frac{\sqrt{7}}{3\sqrt{2}} = n \frac{\sqrt{14}}{6}$.
    Revisamos las opciones:
    C) $\frac{1}{3}\sqrt{\frac{7}{2}}n = \frac{n}{3} \frac{\sqrt{7}}{\sqrt{2}} = \frac{n\sqrt{14}}{6}$. Coincide.
    \textbf{Respuesta: C) $\frac{1}{3}\sqrt{\frac{7}{2}}n$}

    \item[\bfseries 694)] Tabla de frecuencias:
    Clases: [0-4[, [4-8[, [8-12[
    Marcas de clase ($x_i$): 2, 6, 10
    Frecuencias ($f_i$): 2, 1, 2
    Total datos $N = 2+1+2 = 5$. (II es VERDADERA)
    Media $\bar{x} = \frac{\sum f_i x_i}{N} = \frac{2(2)+1(6)+2(10)}{5} = \frac{4+6+20}{5} = \frac{30}{5} = 6$. (I es VERDADERA)
    Varianza $\sigma^2 = \frac{\sum f_i (x_i - \bar{x})^2}{N} = \frac{2(2-6)^2 + 1(6-6)^2 + 2(10-6)^2}{5} = \frac{2(-4)^2 + 1(0)^2 + 2(4)^2}{5} = \frac{2(16) + 0 + 2(16)}{5} = \frac{32+32}{5} = \frac{64}{5} = 12,8$.
    Desviación estándar $\sigma = \sqrt{12,8}$. (III es VERDADERA)
    \textbf{Respuesta: E) I, II y III}

    \item[\bfseries 695)] Edades: {3, 4, 7, 9, 12}. N=5.
    I) Media $\bar{x} = (3+4+7+9+12)/5 = 35/5 = 7$. Si todos aumentan 1 año, la nueva media es $\bar{x}+1 = 7+1=8$. La afirmación dice que sería 5 unidades mayores. FALSO.
    II) La moda es el valor más frecuente. Todos los valores aparecen una vez. La muestra es amodal. VERDADERO.
    III) Media = 7. Desviaciones al cuadrado: $(3-7)^2=16$, $(4-7)^2=9$, $(7-7)^2=0$, $(9-7)^2=4$, $(12-7)^2=25$. Suma de cuadrados = $16+9+0+4+25 = 54$. Varianza $\sigma^2 = 54/5 = 10,8$. Desviación estándar $\sigma = \sqrt{10,8}$. VERDADERO.
    \textbf{Respuesta: E) Solo II y III}

    \item[\bfseries 696)] Tabla de frecuencias:
    Dato ($x_i$): 12, 13, 14, 15
    Frecuencia ($f_i$): 3, 1, 4, 2
    Total datos $N = 3+1+4+2 = 10$.
    Media $\bar{x} = \frac{\sum f_i x_i}{N} = \frac{3(12)+1(13)+4(14)+2(15)}{10} = \frac{36+13+56+30}{10} = \frac{135}{10} = 13,5$.
    Varianza $\sigma^2 = \frac{\sum f_i (x_i - \bar{x})^2}{N} = \frac{3(12-13,5)^2 + 1(13-13,5)^2 + 4(14-13,5)^2 + 2(15-13,5)^2}{10}$
    $= \frac{3(-1,5)^2 + 1(-0,5)^2 + 4(0,5)^2 + 2(1,5)^2}{10} = \frac{3(2,25) + 1(0,25) + 4(0,25) + 2(2,25)}{10}$
    $= \frac{6,75 + 0,25 + 1,00 + 4,50}{10} = \frac{12,5}{10} = 1,25$.
    \textbf{Respuesta: D) 1,25}

    \item[\bfseries 697)] Variable original: $n$ = nº respuestas correctas. Media($n$) = 30. Varianza($n$) = 9. Desviación estándar $S_n = \sqrt{9}=3$.
    Variable nueva: $P = \text{puntaje} = 4n + 64$. Es una transformación lineal $P = a n + b$ con $a=4, b=64$.
    Propiedad de la desviación estándar: $S_{a n + b} = |a| S_n$.
    $S_P = |4| S_n = 4 \times 3 = 12$.
    \textbf{Respuesta: C) 12}

    \item[\bfseries 698)] Datos: $x, y, z, w$. Varianza $Var(x,y,z,w) = \lambda$.
    Nuevos datos: $kx, ky, kz, kw$.
    Propiedad de la varianza: $Var(k \cdot \text{datos}) = k^2 Var(\text{datos})$.
    $Var(kx, ky, kz, kw) = k^2 Var(x, y, z, w) = k^2 \lambda$.
    \textbf{Respuesta: C) $k^2\lambda$}

    \item[\bfseries 699)] Tabla adjunta: $(x_i - \bar{x})^2$.
    Cuando $x_i=6$, $(x_i - \bar{x})^2 = 0$. Esto implica que la media $\bar{x} = 6$.
    Calculamos A y B:
    Para $x_i=7$, $A = (x_i - \bar{x})^2 = (7-6)^2 = 1^2 = 1$.
    Para $x_i=4$, $B = (x_i - \bar{x})^2 = (4-6)^2 = (-2)^2 = 4$.
    Evaluamos las proposiciones:
    I) $A+B = 1+4 = 5$. La proposición dice $A+B=3$. FALSO.
    II) Varianza $\sigma^2 = \text{Promedio de } (x_i - \bar{x})^2$. Los valores son (para $x_i=4,5,6,7,8$): $B, 1, 0, A, 4$, es decir, $4, 1, 0, 1, 4$.
    $\sigma^2 = \frac{4+1+0+1+4}{5} = \frac{10}{5} = 2$.
    Desviación estándar $\sigma = \sqrt{2}$. La proposición II dice $\sigma = \sqrt{2}$. VERDADERO.
    III) La varianza es 2. La proposición III dice que la varianza es 2. VERDADERO.
    \textbf{Respuesta: C) Solo II y III}

    \item[\bfseries 700)] Si todos los datos de una muestra $x_i$ se incrementan en 4 unidades (nuevos datos $x_i + 4$), la varianza no cambia. $Var(x_i + k) = Var(x_i)$.
    \textbf{Respuesta: C) Queda igual}

    \item[\bfseries 701)] Si todos los datos $x_i$ se multiplican por 4 (nuevos datos $4x_i$).
    I) El promedio se multiplica por 4: $Mean(4x_i) = 4 Mean(x_i)$. VERDADERO.
    II) La desviación típica (estándar) se multiplica por $|4|=4$: $Stdev(4x_i) = |4| Stdev(x_i) = 4 Stdev(x_i)$. VERDADERO.
    III) La varianza se multiplica por $4^2=16$: $Var(4x_i) = 4^2 Var(x_i) = 16 Var(x_i)$. La afirmación dice que se duplica. FALSO.
    \textbf{Respuesta: C) Solo I y II}

    \item[\bfseries 702)] Juan: Promedio=613, Desv. Est.=54,47. Pedro: Promedio=613, Desv. Est.=168,74.
    I) La desviación estándar mide la dispersión. Juan tiene menor desviación estándar ($54,47 < 168,74$), por lo que sus puntajes están más concentrados cerca de su promedio 613. VERDADERO.
    II) Ambos tienen el mismo promedio, pero esto no significa que obtuvieron los mismos puntajes individuales. FALSO.
    III) Es perfectamente posible tener la misma media y diferentes desviaciones estándar. No indica un error. FALSO.
    \textbf{Respuesta: A) Solo I}

    \item[\bfseries 703)] Muestra original: N=10, Desv. Est. $\sigma = 1,5$. Varianza $\sigma^2 = (1,5)^2 = 2,25$.
    Transformación: A cada elemento se le agregan 10 unidades ($x_i' = x_i + 10$).
    Efecto: Sumar una constante no cambia la desviación estándar ni la varianza.
    Nueva Desv. Est. = 1,5. Nueva Varianza = 2,25.
    \textbf{Respuesta: E) 1,5 \quad 2,25}

    \item[\bfseries 704)] Buscamos la alternativa FALSA.
    A) Desviación estándar pequeña significa datos concentrados cerca de la media. VERDADERO.
    B) Desviación estándar grande indica datos dispersos, lo que puede interpretarse como menor confianza en que la media sea representativa de todos los datos. VERDADERO (interpretación común).
    C) La desviación estándar ($\sigma = \sqrt{Var}$) es siempre no negativa ($\ge 0$). VERDADERO.
    D) Dos muestras con igual N y misma media pueden tener desviaciones estándar diferentes. Ejemplo: {0, 10} (N=2, Media=5, SD=5) y {4, 6} (N=2, Media=5, SD=1). FALSO. % Corregido: D es la falsa.
    E) La desviación estándar tiene las mismas unidades que los datos originales. VERDADERO.
    \textbf{Respuesta: D)}

    \item[\bfseries 705)] Muestra $n_1, n_2, n_3, n_4$ con promedio $\mu$. Se agrega dato $p$. N cambia de 4 a 5.
    I) Si $p = \mu$. El nuevo dato está en la media. La suma de cuadrados de desviaciones no cambia: $\sum_{i=1}^4 (n_i-\mu)^2$. La nueva varianza es $Var_{nueva} = \frac{\sum (n_i-\mu)^2}{5}$. La varianza original era $Var_{orig} = \frac{\sum (n_i-\mu)^2}{4}$. Si $Var_{orig}>0$, entonces $Var_{nueva} = \frac{4}{5} Var_{orig} < Var_{orig}$. La desviación estándar disminuye. La afirmación dice que aumenta. FALSO.
    II) Si $p = 0$. El efecto en la desviación estándar depende de los datos originales y $\mu$. Si los datos son {1,1,1,1}, $\mu=1, \sigma=0$. Agregando 0, la nueva muestra es {1,1,1,1,0}, $\bar{x}=4/5$, $\sigma^2 = [4(1-4/5)^2 + (0-4/5)^2]/5 = [4(1/25)+16/25]/5 = (20/25)/5 = (4/5)/5=4/25$. $\sigma = 2/5$. Aumentó. Si los datos son {-1,-1,1,1}, $\mu=0, \sigma=1$. Agregando 0, la muestra es {-1,-1,1,1,0}, $\bar{x}=0$, $\sigma^2 = [2(-1-0)^2+2(1-0)^2+(0-0)^2]/5 = (2+2+0)/5 = 4/5$. $\sigma = \sqrt{4/5} < 1$. Disminuyó. La afirmación dice que siempre disminuye. FALSO.
    III) Si $n_1, n_2, n_3, n_4, p$ son enteros consecutivos. Sean $k, k+1, k+2, k+3, k+4$. N=5. Media = $(k+k+4)/2 = k+2$. Desviaciones: -2, -1, 0, 1, 2. Cuadrados: 4, 1, 0, 1, 4. Suma = 10. Varianza $\sigma^2 = 10/5 = 2$. Desviación estándar $\sigma = \sqrt{2}$. VERDADERO.
    \textbf{Respuesta: C) Solo III}

    \item[\bfseries 706)] Comparar $S_A$ y $S_B$.
    Tabla A: Datos $x_i$: 3, 5, 7 con frec $f_i$: 3, 4, 2.
    Tabla B: Datos $y_i$: 555.553, 555.555, 555.557 con frec $f_i$: 3, 4, 2.
    Observamos que $y_i = x_i + 555.550$. Los datos de B son una traslación de los datos de A.
    La desviación estándar es invariante ante traslaciones ($Stdev(x_i+k) = Stdev(x_i)$).
    Por lo tanto, $S_A = S_B$.
    \textbf{Respuesta: E) $S_A = S_B$}

    \item[\bfseries 707)] Población {1, 3, p, q}. N=4. Promedio $\mu=3$. Varianza $\sigma^2=2$. Calcular $3p^2+3q^2$.
    Promedio: $\frac{1+3+p+q}{4} = 3 \implies 4+p+q = 12 \implies p+q = 8$.
    Varianza: $\frac{(1-3)^2 + (3-3)^2 + (p-3)^2 + (q-3)^2}{4} = 2$.
    $\frac{(-2)^2 + 0^2 + (p^2-6p+9) + (q^2-6q+9)}{4} = 2$.
    $\frac{4 + 0 + p^2-6p+9 + q^2-6q+9}{4} = 2$.
    $p^2 + q^2 - 6(p+q) + 22 = 8$.
    Sustituimos $p+q=8$: $p^2 + q^2 - 6(8) + 22 = 8$.
    $p^2 + q^2 - 48 + 22 = 8$.
    $p^2 + q^2 - 26 = 8$.
    $p^2 + q^2 = 34$.
    Se pide $3p^2 + 3q^2 = 3(p^2+q^2) = 3(34) = 102$.
    \textbf{Respuesta: D) 102}

    \item[\bfseries 713)] Datos X: {10, 12, 14, 16}. N=4.
    I) Mediana (X): Datos ordenados. Como N=4 (par), la mediana es el promedio de los dos centrales: $(12+14)/2 = 13$. VERDADERO.
    II) Varianza (X): Media $\bar{x} = (10+12+14+16)/4 = 52/4 = 13$. Desviaciones al cuadrado: $(10-13)^2=9, (12-13)^2=1, (14-13)^2=1, (16-13)^2=9$. Suma de cuadrados = $9+1+1+9 = 20$. Varianza $\sigma^2 = 20/4 = 5$. VERDADERO.
    III) Rango (X): Max - Min = $16 - 10 = 6$. VERDADERO.
    \textbf{Respuesta: E) I, II y III}

    \item[\bfseries 714)] Muestra original: $n$ elementos, media $\mu$, desv. est. $\sigma$, varianza $\sigma^2$.
    Nueva muestra: $y_i = 2x_i + 5$.
    Nueva Media: $Mean(y_i) = Mean(2x_i + 5) = 2 Mean(x_i) + 5 = 2\mu + 5$.
    Nueva Desv. Est.: $Stdev(y_i) = Stdev(2x_i + 5) = |2| Stdev(x_i) = 2\sigma$.
    Nueva Varianza: $Var(y_i) = Var(2x_i + 5) = 2^2 Var(x_i) = 4\sigma^2$.
    La afirmación correcta es: Media $2\mu+5$ y Varianza $4\sigma^2$.
    \textbf{Respuesta: D) Su media es $2\mu + 5$ y su varianza $4\sigma^2$}

    \item[\bfseries 715)] Determinar la mediana de una población de 100 datos.
    (1) La media aritmética es 39. Conocer la media no es suficiente para determinar la mediana. (Insuficiente).
    (2) La varianza es 0. Si la varianza es 0, todos los datos son iguales entre sí. Si todos los 100 datos son iguales, deben ser iguales a la media (que también sería 39, aunque no se use esa info de (1)). Si todos los datos son iguales a un valor $k$, la mediana también es $k$. (Suficiente).
    \textbf{Respuesta: B) (2) por sí sola}

\end{enumerate}

\newpage % Separador entre el contenido de los dos documentos originales

% --- Contenido del Documento 2 ---
\section*{Soluciones Ultima Pagina (Guía PSU)} % Título original de la sección

%----------------------------------------------------
\subsection*{Ejercicio 1: Análisis de Notas}
%----------------------------------------------------
\textbf{Datos proporcionados:}
\begin{itemize}[nosep]
    \item Media (\mean{x}) en ambos trimestres: \num{5,1}
    \item Nota máxima en ambos trimestres: \num{7,0}
    \item Nota mínima en ambos trimestres: \num{3,2}
    \item Número de estudiantes (N): 30 (contando las notas en cada tabla)
\end{itemize}

\begin{enumerate}
    \item \textbf{Coeficiente del rango (CRV):}
    El Rango (R) es la diferencia entre el valor máximo y el mínimo.
    \[ R = \text{Máximo} - \text{Mínimo} = \num{7,0} - \num{3,2} = \num{3,8} \]
    El rango es el mismo para ambos trimestres.
    El Coeficiente de Variación del Rango (CRV) se calcula como $CRV = \frac{R}{\mean{x}}$.
    \[ CRV = \frac{\num{3,8}}{\num{5,1}} \approx \num{0,745} \]
    \textbf{Respuesta:} El coeficiente del rango (CRV) es aproximadamente \num{0,745} para \textbf{ambos trimestres}. Basado en esta medida, ninguno tiene un coeficiente de rango menor que el otro.

    \item \textbf{Interpretación del CRV:}
    \textbf{Respuesta:} Dado que el CRV es el mismo para ambos trimestres, esta medida por sí sola \textbf{no indica} que un trimestre presente calificaciones más dispersas que el otro en relación al promedio.

    \item \textbf{Coeficiente de Desviación Media (CDM):}
    Primero, calculamos la Desviación Media (DM) para cada trimestre: $DM = \frac{\sum |x_i - \mean{x}|}{N}$.
    \begin{itemize}[nosep]
        \item \textbf{Trimestre 1:} La suma de las desviaciones absolutas es $\sum |x_i - \num{5,1}| = \num{24,6}$.
          \[ DM_1 = \frac{\num{24,6}}{30} = \num{0,82} \]
        \item \textbf{Trimestre 2:} La suma de las desviaciones absolutas es $\sum |x_i - \num{5,1}| = \num{17,6}$.
          \[ DM_2 = \frac{\num{17,6}}{30} \approx \num{0,587} \]
    \end{itemize}
    Luego, calculamos el Coeficiente de Desviación Media (CDM): $CDM = \frac{DM}{\mean{x}}$.
    \begin{itemize}[nosep]
        \item \textbf{Trimestre 1:} $CDM_1 = \frac{\num{0,82}}{\num{5,1}} \approx \num{0,161}$
        \item \textbf{Trimestre 2:} $CDM_2 = \frac{\num{0,587}}{\num{5,1}} \approx \num{0,115}$
    \end{itemize}
    \textbf{Respuesta:} El CDM es aprox. \num{0,161} para el primer trimestre y aprox. \num{0,115} para el segundo trimestre.

    \item \textbf{Interpretación del CDM y la "sensación":}
    \textbf{Respuesta:} El CDM mide la dispersión promedio relativa a la media. El segundo trimestre tiene un CDM menor (\num{0,115} vs \num{0,161}), lo que indica que sus notas, en promedio, estaban más cerca de la media \num{5,1}. Esto significa menor dispersión relativa en el segundo trimestre. Esto \textbf{corrobora la "sensación"} de los estudiantes; aunque la media era igual, las notas más agrupadas del segundo trimestre pueden percibirse como "mejores" o más consistentes.

    \item \textbf{Coeficiente de Desviación Estándar (Coeficiente de Variación, CV):}
    Primero, calculamos la Desviación Estándar Poblacional (\pstdev) para cada trimestre: $\pstdev = \sqrt{\frac{\sum (x_i - \mean{x})^2}{N}}$.
    \begin{itemize}[nosep]
        \item \textbf{Trimestre 1:} Suma de cuadrados de desviaciones $\sum (x_i - \num{5,1})^2 = \num{31,14}$.
          \[ \pvar_1 = \frac{\num{31,14}}{30} \approx \num{1,038} \implies \pstdev_1 = \sqrt{\num{1,038}} \approx \num{1,019} \]
        \item \textbf{Trimestre 2:} Suma de cuadrados de desviaciones $\sum (x_i - \num{5,1})^2 = \num{17,18}$.
          \[ \pvar_2 = \frac{\num{17,18}}{30} \approx \num{0,5727} \implies \pstdev_2 = \sqrt{\num{0,5727}} \approx \num{0,757} \]
    \end{itemize}
    Luego, calculamos el Coeficiente de Variación (CV): $CV = \frac{\pstdev}{\mean{x}}$.
    \begin{itemize}[nosep]
        \item \textbf{Trimestre 1:} $CV_1 = \frac{\num{1,019}}{\num{5,1}} \approx \num{0,1998} \approx \num{0,200}$
        \item \textbf{Trimestre 2:} $CV_2 = \frac{\num{0,757}}{\num{5,1}} \approx \num{0,148}$
    \end{itemize}
    \textbf{Respuesta:} El CV es aprox. \num{0,200} (o 20,0\%) para el primer trimestre y aprox. \num{0,148} (o 14,8\%) para el segundo trimestre.

    \item \textbf{Homogeneidad:}
    \textbf{Respuesta:} Las calificaciones más homogéneas son las que presentan menor dispersión relativa (menor CV). Por lo tanto, el \textbf{Segundo Trimestre} (CV $\approx$ \num{0,148}) presenta calificaciones más homogéneas.

    \item \textbf{Interpretación del CV:}
    \textbf{Respuesta:} El CV indica qué tan grande es la desviación estándar en relación a la media. En el primer trimestre, la desviación estándar es un 20\% de la media, mientras que en el segundo es solo un 14,8\%. Esto confirma cuantitativamente que las notas del segundo trimestre están más concentradas alrededor de la media \num{5,1}.

    \item \textbf{Mejor Trimestre:}
    \textbf{Respuesta:} Aunque la media, máximo y mínimo son iguales, el \textbf{Segundo Trimestre} puede considerarse "mejor" debido a su \textbf{menor dispersión} (menor DM, menor \pstdev, menor CV). Esto sugiere mayor consistencia en el rendimiento de los estudiantes, lo que puede ser preferible y coincide con la "sensación" reportada. Un grupo más homogéneo puede indicar un aprendizaje más uniforme.

    \item \textbf{Gráfica Representativa:}
    \textbf{Respuesta:} Un \textbf{diagrama de caja y bigotes (boxplot) comparativo} sería ideal. Mostraría la misma media (o medianas similares), los mismos extremos, pero visualizaría claramente la diferencia en la dispersión (longitud de la caja y/o bigotes) entre los dos trimestres. Alternativamente, histogramas o diagramas de puntos comparativos también servirían.

\end{enumerate}

%----------------------------------------------------
\subsection*{Ejercicio 2: Salto con Garrocha}
%----------------------------------------------------
\textbf{Datos:} \num{2,50}; \num{2,80}; \num{2,60}; \num{3,00}; \num{2,90} (metros). N = 5.

\begin{enumerate}
    \item \textbf{Suma de desviaciones respecto a la media:}
    Primero, calcular la media (\mean{x}):
    \[ \mean{x} = \frac{\num{2,50} + \num{2,80} + \num{2,60} + \num{3,00} + \num{2,90}}{5} = \frac{\num{13,80}}{5} = \num{2,76} \text{ m} \]
    Ahora, calcular las desviaciones $(x_i - \mean{x})$:
    \begin{itemize}[nosep]
        \item $\num{2,50} - \num{2,76} = \num{-0,26}$
        \item $\num{2,80} - \num{2,76} = \num{+0,04}$
        \item $\num{2,60} - \num{2,76} = \num{-0,16}$
        \item $\num{3,00} - \num{2,76} = \num{+0,24}$
        \item $\num{2,90} - \num{2,76} = \num{+0,14}$
    \end{itemize}
    Sumar las desviaciones:
    \[ (\num{-0,26}) + (\num{+0,04}) + (\num{-0,16}) + (\num{+0,24}) + (\num{+0,14}) = (\num{-0,42}) + (\num{+0,42}) = 0 \]
    \textbf{Respuesta:} Se comprueba que la suma de las desviaciones respecto a la media es \textbf{0}.

    \item \textbf{Desviación Media (DM):}
    Usamos las desviaciones absolutas de la parte (a): $|\num{-0,26}|=\num{0,26}$; $|\num{+0,04}|=\num{0,04}$; $|\num{-0,16}|=\num{0,16}$; $|\num{+0,24}|=\num{0,24}$; $|\num{+0,14}|=\num{0,14}$.
    Sumamos las desviaciones absolutas:
    \[ \sum |x_i - \mean{x}| = \num{0,26} + \num{0,04} + \num{0,16} + \num{0,24} + \num{0,14} = \num{0,84} \]
    Calculamos la Desviación Media:
    \[ DM = \frac{\sum |x_i - \mean{x}|}{N} = \frac{\num{0,84}}{5} = \num{0,168} \text{ m} \]
    \textbf{Respuesta:} La desviación media de los datos es \textbf{\num{0,168} metros}.
\end{enumerate}

%----------------------------------------------------
\subsection*{Ejercicio 3: Distribución de Frecuencias}
%----------------------------------------------------
\textbf{Datos:} Tabla de frecuencias agrupadas. N = \num{1800}.
\textbf{Tabla de Cálculos:} (x\textsubscript{m} = Marca de Clase)

\centering
% \begin{tabular}{crrcrr} % Original - Sin booktabs
% \toprule % Necesita booktabs
% Puntaje & Freq ($f_i$) & $x_{mi}$ & $f_i \cdot x_{mi}$ & $(x_{mi} - \mean{x})$ & $f_i \cdot (x_{mi} - \mean{x})^2$ \\
% \midrule % Necesita booktabs
% 0 - 2   & 21       & 1  & 21       & -13,82 & \num{4001,3} \\
% 3 - 5   & 50       & 4  & 200      & -10,82 & \num{5853,6} \\
% 6 - 8   & 110      & 7  & 770      & -7,82 & \num{6730,3} \\
% 9 - 11  & 241      & 10 & 2410     & -4,82 & \num{5599,7} \\
% 12 - 14 & 423      & 13 & 5499     & -1,82 & \num{1398,5} \\
% 15 - 17 & 457      & 16 & 7312     & 1,18 & \num{634,4} \\
% 18 - 20 & 275      & 19 & 5225     & 4,18 & \num{4806,2} \\
% 21 - 23 & 134      & 22 & 2948     & 7,18 & \num{6909,0} \\
% 24 - 26 & 66       & 25 & 1650     & 10,18 & \num{6840,1} \\
% 27 - 29 & 23       & 28 & 644      & 13,18 & \num{3994,8} \\
% \midrule % Necesita booktabs
% \textbf{Total} & \textbf{1800} &    & \textbf{26679} & & \textbf{46767,9} \\
% \bottomrule % Necesita booktabs
% \end{tabular}
% Usando booktabs y siunitx para alinear números
\begin{tabular}{l S[table-format=4.0] S[table-format=2.0] S[table-format=5.0] S[table-format=-2.2] S[table-format=5.1]}
\toprule
Puntaje & {$f_i$} & {$x_{mi}$} & {$f_i \cdot x_{mi}$} & {$(x_{mi} - \mean{x})$} & {$f_i \cdot (x_{mi} - \mean{x})^2$} \\
\midrule
0 - 2   & 21      & 1  & 21     & -13,82 & 4001,3 \\
3 - 5   & 50      & 4  & 200    & -10,82 & 5853,6 \\
6 - 8   & 110     & 7  & 770    & -7,82  & 6730,3 \\
9 - 11  & 241     & 10 & 2410   & -4,82  & 5599,7 \\
12 - 14 & 423     & 13 & 5499   & -1,82  & 1398,5 \\
15 - 17 & 457     & 16 & 7312   & 1,18   & 634,4 \\
18 - 20 & 275     & 19 & 5225   & 4,18   & 4806,2 \\
21 - 23 & 134     & 22 & 2948   & 7,18   & 6909,0 \\
24 - 26 & 66      & 25 & 1650   & 10,18  & 6840,1 \\
27 - 29 & 23      & 28 & 644    & 13,18  & 3994,8 \\
\midrule
\textbf{Total} & \textbf{1800} &    & \textbf{26679} & & \textbf{46767,9} \\
\bottomrule
\end{tabular}


\vspace{1ex}
Primero, calculamos la media (\mean{x}):
\[ \mean{x} = \frac{\sum (f_i \cdot x_{mi})}{N} = \frac{\num{26679}}{\num{1800}} \approx \num{14,82} \text{ puntos} \]

\begin{enumerate}
    \item \textbf{Desviación Estándar (\pstdev):}
    Calculamos la varianza poblacional (\pvar):
    \[ \pvar = \frac{\sum f_i \cdot (x_{mi} - \mean{x})^2}{N} = \frac{\num{46767,9}}{\num{1800}} \approx \num{25,982} \]
    La desviación estándar es la raíz cuadrada de la varianza:
    \[ \pstdev = \sqrt{\pvar} \approx \sqrt{\num{25,982}} \approx \num{5,097} \text{ puntos} \]
    (Nota: Usando la fórmula $\pvar = [\sum(f \cdot x_m^2) / N] - \mean{x}^2$, con $\sum(f \cdot x_m^2) = \num{442203}$, da $\pvar = \num{442203}/1800 - (\num{26679}/1800)^2 \approx \num{245,6683} - \num{14,82167}^2 \approx \num{245,6683} - \num{219,6819} \approx \num{25,986}$, que es muy similar).
    \textbf{Respuesta:} La desviación estándar de la distribución es aproximadamente \textbf{\num{5,10} puntos}.

    \item \textbf{Valores $\mean{x} + \pstdev$ y $\mean{x} - \pstdev$:}
    \[ \mean{x} - \pstdev \approx \num{14,82} - \num{5,10} = \num{9,72} \text{ puntos} \]
    \[ \mean{x} + \pstdev \approx \num{14,82} + \num{5,10} = \num{19,92} \text{ puntos} \]
    \textbf{Respuesta:} Los valores corresponden aproximadamente a \textbf{\num{9,72} puntos} y \textbf{\num{19,92} puntos}.
\end{enumerate}


%----------------------------------------------------
\subsection*{Ejercicio 4: Prueba de Matemática}
%----------------------------------------------------
\textbf{Datos:}
\begin{itemize}[nosep]
    \item Curso A: $\mean{x}_A = \num{5,3}$; $\stdev_A = \num{0,7}$ (Asumimos desviación estándar muestral 's', aunque podría ser poblacional $\pstdev$).
    \item Curso B: $\mean{x}_B = \num{5,4}$; $\stdev_B = \num{0,4}$
\end{itemize}

\begin{enumerate}
    \item \textbf{Comparación de Alumnos (Rendimiento Relativo):}
    Para comparar el rendimiento relativo a su curso, calculamos el puntaje Z (o puntaje estándar): $Z = \frac{x - \mean{x}}{\stdev}$.
    \begin{itemize}[nosep]
        \item \textbf{Alumno A:} Nota $x_A = \num{6,7}$.
          \[ Z_A = \frac{\num{6,7} - \num{5,3}}{\num{0,7}} = \frac{\num{1,4}}{\num{0,7}} = \num{2,0} \]
          (Este alumno está 2 desviaciones estándar por encima de la media de su curso).
        \item \textbf{Alumno B:} Nota $x_B = \num{6,6}$.
          \[ Z_B = \frac{\num{6,6} - \num{5,4}}{\num{0,4}} = \frac{\num{1,2}}{\num{0,4}} = \num{3,0} \]
          (Este alumno está 3 desviaciones estándar por encima de la media de su curso).
    \end{itemize}
    Comparamos los puntajes Z: $Z_B (\num{3,0}) > Z_A (\num{2,0})$.
    \textbf{Respuesta:} Al alumno del \textbf{Curso B} le fue mejor en la prueba en relación a su curso, ya que su rendimiento relativo (medido por el puntaje Z) fue superior al del alumno del Curso A.

    \item \textbf{Justificación:}
    \textbf{Respuesta:} La justificación se basa en el \textbf{puntaje Z}. Aunque la nota absoluta del alumno A (\num{6,7}) es ligeramente mayor que la del alumno B (\num{6,6}), el puntaje Z mide qué tan excepcional es esa nota dentro del contexto de su propio grupo. El Curso B era más homogéneo ($\stdev_B=\num{0,4}$) que el Curso A ($\stdev_A=\num{0,7}$). Por lo tanto, obtener una nota de \num{6,6} en el Curso B representa una desviación mucho mayor y más positiva respecto a la media de su grupo (3 desviaciones estándar) que obtener \num{6,7} en el Curso A (solo 2 desviaciones estándar sobre su media). El alumno B se destacó más dentro de su curso.
\end{enumerate}

\end{document}
