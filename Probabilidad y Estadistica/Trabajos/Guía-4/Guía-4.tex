\documentclass[11pt]{article}
\usepackage{amsmath}  % Math
\usepackage{amssymb}  % Symbols
\usepackage{graphicx} % Images
\usepackage[utf8]{inputenc}
\usepackage[T1]{fontenc}
\usepackage[margin=1in]{geometry}
\usepackage[spanish]{babel} % Spanish language support

\title{Respuestas Guía 4}
\author{Felipe Colli}
\date{\today}

\begin{document}

\maketitle

\begin{enumerate}
\section{Respuestas 01-20}

    


\section{Respuestas 21-40}


\section{Respuestas 41-60}


\section{Respuestas 61-80}


\section{Respuestas 81-100}

    \item \textbf{Respuesta C} Es la C, ya que el primer cuartil y segundo cuartil, sus maximos estan en 25 y 30 segundos,por lo cual al aumentar sus valores en 30 segundos, su maximo ahora sera 55 y 60 segundos. Mientras que el tercer cuartil tiene su maximo en 46 segundos, siendo el extremo superior del diagrama 70 segundos, por lo cual al aumentar sus valores, su maximo ahora es de 61 segundos y el extremo es de 85 segundos. Por lo cual podemos concluir que al menos 25 personas duraran mas de 60s %81
    \item \textbf{Respuesta A} LA respuesta es A, ya que el percentil 20 o el primer quintil, son 30 personas (20\% de 150), y la priemra frecduencia acumulada igual o mayor a 30, es 36, en el intervalo [30,35[ % 82
    \item \textbf{B} El percentil 40 se encuntra en el puntaje 30, por lo cula solos los estudiantes con un puntaje menor a 30 podran ir a por la prueba recuperativa, estos siendo 75 (por frecuencia acumulada) % 83

    \item \textbf{Respuesta D} ya que la desviacion estandar $\sigma$ es menor en el curso A, comparado al curso B, lo que hace q los datos esten mas agrupados, y como no sabemos mas datos, no podemos saber casos especificos  % 84

    \item \textbf{Respuesta C} Sabemos cuanto valen $x_1-\bar{x}$, $x_2-\bar{x}$, $x_3-\bar{x}$, pero no sabemos cuanto vale $x_4-\bar{x}$. Le asignamos un valor z a $x_4 - \bar{x}$ , y si sumamos los 4 datos menos el promedio, nos da 4 veces el proemdio menos 4 veces el promedio, lo ucla es 0, y sabemos que 3 mas z es 0, por lo cual z es igual a -3. \\ 
Remplazando en la formula de la varianza, llegamos a la varianza es $\frac{15}{2}$ % 85

    \item \textbf{Respuesta }




\end{enumerate}

% extremos, 25%, mediana(50%), 75%, y máximo

\end{document}
