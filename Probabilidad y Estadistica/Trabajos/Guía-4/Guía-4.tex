\documentclass[11pt]{article}
\usepackage{amsmath}  % Math
\usepackage{amssymb}  % Symbols
\usepackage{graphicx} % Images
\usepackage[utf8]{inputenc}
\usepackage[T1]{fontenc}
\usepackage[margin=1in]{geometry}
\usepackage[spanish]{babel} % Spanish language support

\title{Respuestas Guía 4}
\author{Felipe Colli}
\date{\today}

\begin{document}

\maketitle

\begin{enumerate}
\setcounter{enumi}{80}

\section{Respuestas 81-100}

    \item \textbf{Respuesta C} El 25\% de las personas con los tiempos más altos (el grupo sobre el tercer cuartil, $Q_3=46$ s) aumentarán su tiempo en 15 s. Sus nuevos tiempos estarán en el rango $[46+15, 70+15]$, es decir, $[61, 85]$. Como este grupo está compuesto por el 25\% de 100 personas, al menos 25 personas durarán más de 60 segundos. %81
    
    \item \textbf{Respuesta A} Para encontrar el percentil 20 en una muestra de 150 personas, calculamos su posición: $150 \cdot (20/100) = 30$. Buscamos el intervalo que contiene al dato número 30. La frecuencia acumulada del primer intervalo $[25, 30[$ es 12. La del segundo intervalo $[30, 35[$ es $12 + 24 = 36$. Por lo tanto, el dato 30 se encuentra en el intervalo $[30, 35[$. % 82
    
    \item \textbf{Respuesta B} Para un total de 200 estudiantes, la posición del percentil 40 es $200 \cdot (40/100) = 80$. Según la tabla de frecuencia acumulada, el estudiante en la posición 80 obtuvo un puntaje de 30. Los estudiantes que están "bajo el percentil 40" son aquellos con puntajes menores que 30. Estos son los que obtuvieron 10 o 20 puntos. La frecuencia acumulada hasta el puntaje 20 es de 75. Por lo tanto, 75 estudiantes pueden optar a la prueba. % 83

    \item \textbf{Respuesta D} La desviación estándar mide la dispersión de los datos. El curso A tiene una desviación estándar ($0,2$) menor que la del curso B ($0,4$). Esto significa que las notas del curso A están más concentradas alrededor del promedio (son más homogéneas) que las del curso B. Por lo tanto, la afirmación de que el curso A presenta menor dispersión es siempre verdadera. % 84

    \item \textbf{Respuesta C} La suma de las desviaciones respecto a la media ($x_i - \bar{x}$) es siempre cero. Así, $(x_1-\bar{x})+(x_2-\bar{x})+(x_3-\bar{x})+(x_4-\bar{x}) = 0$. Reemplazando los valores: $1 + (-2) + 4 + (x_4 - \bar{x}) = 0$, lo que nos da $(x_4 - \bar{x}) = -3$. La varianza es el promedio de los cuadrados de estas desviaciones: $\sigma^2 = \frac{(1)^2+(-2)^2+(4)^2+(-3)^2}{4} = \frac{1+4+16+9}{4} = \frac{30}{4} = \frac{15}{2}$. % 85
    
    \item \textbf{Respuesta C} Primero, se calcula el promedio ($\bar{x}$) y la desviación estándar ($\sigma$) de los puntajes (6, 8, 9, 12, 15). El promedio es 10. La varianza ($\sigma^2$) es 10, por lo que la desviación estándar ($\sigma$) es $\sqrt{10} \approx 3.16$. El intervalo de aceptación es $[10-\sqrt{10}, 10+\sqrt{10}]$, aproximadamente $[6.84, 13.16]$. Se eliminan los puntajes 6 y 15 por estar fuera del intervalo. El nuevo promedio se calcula con los puntajes restantes (8, 9, 12): $\frac{8+9+12}{3} = \frac{29}{3}$. %86
    
    \item \textbf{Respuesta B} Los datos del grupo B se obtienen sumando una constante (1000) a cada dato del grupo A. Una propiedad fundamental de la varianza es que es invariante a la traslación, es decir, no cambia si se suma una constante a todos los datos. La dispersión es la misma, solo se desplaza el conjunto de datos. Por lo tanto, $\text{Var}(A) = \text{Var}(B)$. La opción B lo justifica correctamente. %87
    
    \item \textbf{Respuesta D} La desviación estándar es cero solo si todos los datos de un conjunto son iguales. Como se nos dice que $p < q < r$, los valores en el Grupo 1 (p, q, r) no son todos iguales, por lo que su desviación estándar es mayor que cero. De igual manera, los valores en el Grupo 2 ($p-1$, q, $r+1$) tampoco son iguales, por lo que su desviación estándar también es mayor que cero. %88
    
    \item \textbf{Respuesta C} Con los datos de la fila "1 hijo" (frecuencia 2, frecuencia relativa 0,2), calculamos el total de familias ($N$): $N = 2 / 0,2 = 10$. Con $N=10$, calculamos las frecuencias absolutas: $\text{Frec}(0) = 10 \cdot 0,1 = 1$; $\text{Frec}(2) = 10 \cdot 0,3 = 3$; $\text{Frec}(3) = 10 \cdot 0,4 = 4$. El conjunto de datos es un '0', dos '1', tres '2' y cuatro '3', lo que corresponde a la opción C. %89
    
    \item \textbf{Respuesta B} El total de niños es 19. Del gráfico, las frecuencias son 1, $m$, 4, $n$. Entonces, $1+m+4+n=19$, lo que simplifica a $m+n=14$. La diferencia entre los niños de 4 años ($n$) y 2 años ($m$) es a lo más 2, es decir, $n-m \leq 2$. Con $n > m$ y $m+n=14$, la única solución entera es $m=6$ y $n=8$, que satisface $8-6=2 \leq 2$. %90

    \item \textbf{Respuesta A} Se evalúa la afirmación A: El número de países en 1995 era 112 y en 2006 era 150. El incremento absoluto es $150 - 112 = 38$. El incremento porcentual es $(\frac{38}{112}) \cdot 100 \approx 33.92\%$. La afirmación es correcta. Las otras opciones son incorrectas al verificar los datos del gráfico. %91
    
    \item \textbf{Respuesta B} Para obtener las frecuencias de cada intervalo (barras del histograma), se restan las frecuencias acumuladas consecutivas. Frec$[0,5[ = 40$. Frec$[5,10[ = 70-40=30$. Frec$[10,15[ = 84-70=14$. Frec$[15,20[ = 92-84=8$. Frec$[20,25[ = 98-92=6$. Frec$[25,30] = 100-98=2$. El histograma de la opción B representa correctamente estas frecuencias. %92
    
    \item \textbf{Respuesta A} El total de datos es $N = 8+x+9+12 = 29+x$. La mediana es 5. Si $x=14$, $N=43$ (impar) y la posición de la mediana es $(43+1)/2 = 22$. La frecuencia acumulada hasta la nota 4 es 8 y hasta la nota 5 es $8+14=22$. Por lo tanto, el dato en la posición 22 es un 5, y la mediana es 5. Para otros valores, como $x=13$, la mediana sería $5.5$. Por lo tanto, $x=14$ es un valor posible. %93

    \item \textbf{Respuesta D} Cuando todos los datos se multiplican por una constante $k$, la media y la mediana también se multiplican por $k$. La mediana original es $m$ y la nueva es $r$, así que $r = k \cdot m$, de donde $k = \frac{r}{m}$. La media original es $p$, por lo que la nueva media será $k \cdot p$. Sustituyendo $k$, la nueva media es $\frac{r}{m} \cdot p$. %94

    \item \textbf{Respuesta C} El promedio se calcula como $\frac{\sum(\text{valor} \cdot \text{frecuencia})}{\sum(\text{frecuencias})}$. El numerador es $(0\cdot5 + 1\cdot0 + 2\cdot7 + 3\cdot6 + 4\cdot5 + 5\cdot7) = 87$. El denominador es la suma de las frecuencias $5+0+7+6+5+7 = 30$. El promedio es $\frac{87}{30}$. La opción C muestra la estructura de cálculo correcta. %95
    
    \item \textbf{Respuesta C} El total de datos es $N=6+4+5+5=20$. Se busca la posición del percentil 80: $0.80 \cdot 20 = 16$. Debemos encontrar el intervalo que contiene al dato número 16. La frecuencia acumulada hasta el intervalo $[24, 26[$ es $6+4+5=15$. La frecuencia acumulada del siguiente intervalo $[26, 28]$ es 20. Por lo tanto, el dato 16 se encuentra dentro del intervalo $[26, 28]$. %96
    
    \item \textbf{Respuesta D} Por definición, el tercer cuartil ($Q_3$) de un conjunto de datos es el valor por debajo del cual se encuentra el 75\% de los datos. En el diagrama de caja del Colegio Educa, $Q_3$ es 5,9. Por lo tanto, es siempre verdadero que el 75\% de los estudiantes de ese colegio tiene un promedio anual menor o igual a 5,9. %97
    
    \item \textbf{Respuesta C} El total de personas es 200. La posición del tercer cuartil ($Q_3$) es $\frac{3}{4} \cdot 200 = 150$. Se nos dice que $Q_3=3$, por lo que el dato en la posición 150 debe ser 3. La frecuencia acumulada hasta 2 televisores es $48+72=120$. La frecuencia acumulada hasta 3 televisores es $120+h$. Para que el dato 150 esté en esta categoría, se debe cumplir que $120 + h \geq 150$, lo que implica $h \geq 30$. El valor mínimo para $h$ es 30. %98

    \item \textbf{Respuesta E} Primero, se calcula la frecuencia acumulada: $[6,10[ \rightarrow 5$; $[10,14[ \rightarrow 5+17=22$; $[14,18[ \rightarrow 22+8=30$; $[18,22[ \rightarrow 30+15=45$; $[22,26] \rightarrow 45+10=55$. II) La etiqueta $P$ en el gráfico corresponde a la frecuencia acumulada para el puntaje 14, que es 22. La afirmación es correcta. III) La etiqueta $Q$ está en el punto más alto de la ojiva, que representa la frecuencia acumulada total. El total de datos es 55, entonces $Q=55$. La afirmación es incorrecta. Por tanto, solo II es verdadera. (Nota: Hay un error en la pregunta original o las opciones, asumiendo que el gráfico es consistente con la tabla, la respuesta correcta sería solo II). %99
    
    \item \textbf{Respuesta D} El gráfico muestra la frecuencia relativa acumulada. Los puntos son $(5, 0.2)$, $(10, 0.7)$, $(15, 0.8)$ y $(20, 1.0)$. Como la muestra es de 100 personas, la frecuencia acumulada absoluta se obtiene multiplicando la relativa por 100. Las frecuencias acumuladas son: 20 (para el intervalo hasta 5), 70 (hasta 10), 80 (hasta 15) y 100 (hasta 20). La tabla de la opción D coincide con estos valores. %100


\end{enumerate}

\end{document}

