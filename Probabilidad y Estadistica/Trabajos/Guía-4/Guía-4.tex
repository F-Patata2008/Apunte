\documentclass[11pt]{article}
\usepackage{amsmath}  % Math
\usepackage{amssymb}  % Symbols
\usepackage{graphicx} % Images
\usepackage[utf8]{inputenc}
\usepackage[T1]{fontenc}
\usepackage[margin=1in]{geometry}
\usepackage[spanish]{babel} % Spanish language support

\title{Respuestas Guía 4}
\author{Felipe Colli}
\date{\today}

\begin{document}

\maketitle

\begin{enumerate}
    \section{Respuestas 1-20}

    \item \textbf{Respuesta C} Primero completamos la tabla. El total de datos es $N=100$. Frecuencia del intervalo $[63, 66[$ es $23-5=18$. Frecuencia del último intervalo es $100-92=8$. La afirmación C dice "La moda es 42". La moda es el valor o intervalo con la mayor frecuencia. La mayor frecuencia es 42, que corresponde al intervalo $[66, 69[$. Por lo tanto, 42 es una frecuencia, no la moda. Esta afirmación es FALSA. %1

    \item \textbf{Respuesta C} Para determinar la mediana, necesitamos conocer la posición del dato central, que depende del total de datos $N=40+x$, y el valor de los datos, que depende de $a$.
    \begin{itemize}
        \item (1) por sí sola: $a=5$. Los datos son 5, 6, 7, 8. Aún no conocemos $x$ para hallar la posición de la mediana. Insuficiente.
        \item (2) por sí sola: $x=7$. El total es $N=47$. La mediana está en la posición $(47+1)/2=24$. La frecuencia acumulada para el dato $a$ es 8, para $a+1$ es $8+12=20$, y para $a+2$ es $20+7=27$. El dato en la posición 24 es $a+2$. No conocemos $a$. Insuficiente.
        \item (1) y (2) juntas: $a=5$ y $x=7$. Sabemos que la mediana es el dato $a+2$, que ahora es $5+2=7$. Se puede determinar.
    \end{itemize} %2

    \item \textbf{Respuesta B} (Considerando la no aplicabilidad de la mediana a datos nominales). I) Verdadero: 10 de 50 estudiantes prefieren Diseño Gráfico, que es $\frac{10}{50}=20\%$. II) Verdadero: La frecuencia más alta es 13, que corresponde a "Pintura" y "Educación Artística" (distribución bimodal). III) Falso: Los datos son nominales (categorías sin orden). La mediana no está definida para este tipo de datos. Aunque se puede encontrar la "categoría mediana" si se ordenan, técnicamente la afirmación es inválida. %3
    
    \item \textbf{Respuesta E} A) Verdadero (50\% obtuvo 2 correctas). B) Verdadero (10\% no obtuvo correctas, es $\frac{1}{10}$). C) Verdadero (1 correcta: 25\%; todas o ninguna: $15\%+10\%=25\%$). D) Verdadero (Promedio = $0\cdot0.1+1\cdot0.25+2\cdot0.5+3\cdot0.15 = 1.7$). E) Falso: La mediana es el valor del percentil 50. La frecuencia acumulada para 0 correctas es 10\%, para 1 correcta es $10+25=35\%$, para 2 correctas es $35+50=85\%$. El percentil 50 se encuentra en la categoría de "2 respuestas correctas". La mediana es 2, no 1.5. %4
    
    \item \textbf{Respuesta D} I) Verdadero: Nota 4 la obtuvieron 14 alumnos; Nota 3, 10 alumnos. El aumento es de 4 alumnos. El aumento porcentual es $\frac{4}{10} = 40\%$. II) Falso: Nota 7 la obtuvieron 2 alumnos; Nota 2, 2 alumnos. El número es el mismo (100\%), no el 50\%. III) Verdadero: Nota 6 la obtuvieron 6 alumnos; Nota 7, 2 alumnos. La relación es $\frac{6}{2} = 3$, que corresponde al 300\%. Son verdaderas I y III. %5

    \item \textbf{Respuesta C} (2) por sí sola, asumiendo que k y m son enteros positivos, nos da $k+m=8$. Los pares (k,m) con $k \neq m$ son $(1,7), (2,6), (3,5)$. En todos estos casos, al ordenar la lista $\{2,3,3,8,k,m\}$, la mediana resulta ser 3. Sin embargo, si no se asume que son enteros (lo que se indica en (1)), (2) no es suficiente. Por tanto, se necesitan ambas: (1) para saber que son enteros y (2) para la relación que determina el valor de la mediana. %6
    
    \item \textbf{Respuesta E} El peso total del grupo de 50 jóvenes es $50 \cdot 56 = 2800$ kg. El peso total de los 30 hombres es $30 \cdot 64 = 1920$ kg. Hay $50-30=20$ mujeres. El peso total de las mujeres es $2800 - 1920 = 880$ kg. El promedio de peso de las mujeres es $\frac{880}{20} = 44$ kg. %7
    
    \item \textbf{Respuesta B} El peso total original de los 10 niños era $10 \cdot 20 = 200$ kg. El peso total de los 9 niños restantes es $9 \cdot 19 = 171$ kg. El peso del niño cuya ficha se perdió es la diferencia: $200 - 171 = 29$ kg. %8
    
    \item \textbf{Respuesta D} El total de datos es $N=24$. Posición del Cuartil 3 ($Q_3$): $\frac{3}{4} \cdot 24 = 18$. Se busca el intervalo que contiene al 18º dato. La frecuencia acumulada hasta $[6,8[$ es 13. La frecuencia acumulada hasta $[8,10[$ es $13+6=19$. Por lo tanto, el 18º dato se encuentra en el intervalo $[8,10[$. %9
    
    \item \textbf{Respuesta D} Para hallar $a$, sumamos los porcentajes conocidos y restamos de 100: $10+18+a+18+10=100 \Rightarrow a=44\%$. Para hallar el Cuartil 3 ($Q_3$, o percentil 75), buscamos en qué categoría se supera el 75\% de frecuencia acumulada. Frec. Acum: $X=1 \to 10\%$; $X=2 \to 28\%$; $X=3 \to 72\%$; $X=4 \to 90\%$. El 75\% se alcanza en la categoría $X=4$. %10
    
    \item \textbf{Respuesta C} El valor de K es la mediana. En un diagrama de caja, la mediana puede ser cualquier valor entre el primer y el tercer cuartil (entre 9 y 15 en este caso). Afirmar que "K es igual a 12" es una posibilidad, pero no una certeza, por lo que es una afirmación que \textit{podría ser falsa}. Las otras afirmaciones se deducen directamente de la estructura del diagrama. %11

    \item \textbf{Respuesta E} (1) Conocer el rango intercuartil ($Q_3-Q_1=11$) no nos da información sobre los valores absolutos ni sobre el mínimo $a$. (2) Conocer el percentil 75 ($Q_3=19$) tampoco nos da información sobre el mínimo. Juntando ambas informaciones, podemos hallar $Q_1=8$, pero seguimos sin conocer el valor del mínimo $a$, ya que la longitud del bigote inferior es desconocida. Se requiere información adicional. %12

    \item \textbf{Respuesta B} El rango de un conjunto de $n$ números enteros consecutivos es siempre $n-1$. (1) Conocer el promedio no permite determinar $n$. (2) Si la muestra tiene $n=9$ elementos, el rango es $9-1=8$. Esta información por sí sola es suficiente. %13

    \item \textbf{Respuesta E} Primero, ordenamos los datos. Como $x>1$, el orden es $\{x-2, x, 2x, 2x+3\}$. La mediana de 4 datos es el promedio de los dos centrales: $\frac{x+2x}{2} = 15 \Rightarrow 3x=30 \Rightarrow x=10$. El conjunto es $\{8, 10, 20, 23\}$. Se extrae el dato menor (8), y el nuevo conjunto es $\{10, 20, 23\}$. La nueva mediana es el valor central, 20. %14
    
    \item \textbf{Respuesta D} Sea $P_k$ la proporción de alumnos con $k$ respuestas correctas. Sabemos que $P_0=0.2$ y el promedio es 1.5. El promedio se calcula como $0 \cdot P_0 + 1 \cdot P_1 + 2 \cdot P_2 = 1.5$, lo que da $P_1+2P_2=1.5$. También sabemos que $P_0+P_1+P_2=1$, lo que da $P_1+P_2=0.8$. Resolviendo el sistema de ecuaciones, obtenemos $P_2=0.7$, que es un 70\%. %15
    
    \item \textbf{Respuesta A} (Asumiendo que las opciones corresponden a I, II, III). Una distribución bimodal tiene dos modas, es decir, dos valores con la frecuencia más alta.
    \begin{itemize}
        \item Gráfico I: Las barras P y Q tienen la misma altura máxima. Es bimodal.
        \item Gráfico II: Las tres barras tienen la misma altura. Es una distribución uniforme (o trimodal), no bimodal.
        \item Gráfico III: Solo la barra Q tiene la altura máxima. Es unimodal.
    \end{itemize}
    Solo el gráfico I representa una muestra bimodal. %16

    \item \textbf{Respuesta E} La mediana es el valor que corresponde al 50\% en el gráfico de frecuencia acumulada. El 50\% en el eje Y se encuentra en el segmento de recta que une los puntos $(2, 40\%)$ y $(4, 60\%)$. Usando interpolación lineal para encontrar el valor de X (visitas) que corresponde a Y=50\%: $X = 2 + (4-2) \frac{50-40}{60-40} = 2 + 2 \cdot \frac{10}{20} = 2 + 1 = 3$. La mediana es 3 visitas. %17

    \item \textbf{Respuesta E} I) Falso: La moda es 6 (el dato con la mayor frecuencia, que es 9). II) Verdadero: Total de datos $N=26$. La mediana es el promedio de los datos en posición 13 y 14. La frecuencia acumulada hasta el dato 4 es 13. El dato 13 es un 4. El dato 14 es un 6. Mediana = $\frac{4+6}{2}=5$. III) Falso: Promedio = $\frac{2\cdot5+4\cdot8+6\cdot9+8\cdot4}{26} = \frac{128}{26} \approx 4.92$, que no es mayor que 5. Como solo II es verdadera y no hay una opción "Solo II", las opciones son incorrectas. "Ninguna de ellas" sería la respuesta si se considera la lista de opciones dada. %18
    
    \item \textbf{Respuesta D} Para un conjunto de tres números naturales consecutivos $\{n-1, n, n+1\}$: I) Verdadero: El promedio es $\frac{3n}{3}=n$, y la mediana es el valor central, $n$. II) Verdadero: El rango es $(n+1)-(n-1)=2$. III) Verdadero: La varianza es $\frac{(-1)^2+0^2+1^2}{3}=\frac{2}{3}$, y la desviación estándar es $\sqrt{\frac{2}{3}}$. Todas son verdaderas. %19
    
    \item \textbf{Respuesta A} I) Verdadero: Un conjunto de datos es más homogéneo si su dispersión es menor. El curso A tiene una desviación estándar (60) menor que el curso B (100), por lo que es más homogéneo. II) Falso: El curso B tiene mayor desviación estándar, por lo tanto, mayor dispersión. III) Falso: No se puede determinar la media combinada sin saber el número de alumnos en cada curso. No es siempre verdadera. %20
        

    \section{Respuestas 21-40}

    \item \textbf{Respuesta: D}
    \begin{itemize}
        \item[I)] Falso. En Matemática, $Q_1=4.1$, por lo que menos del 25\% reprobó (nota $<4.0$). En Lenguaje, $Q_1=4.6$, por lo que también menos del 25\% reprobó. No se puede inferir cuál grupo tuvo más reprobados.
        \item[II)] Verdadero. El primer cuartil ($Q_1$) en Matemática es 4.1. Esto significa que el 75\% de los estudiantes obtuvo una nota igual o superior a 4.1. Como la nota de aprobación es 4.0, se puede inferir que al menos el 75\% del curso aprobó.
        \item[III)] Verdadero. La nota máxima en la prueba de lenguaje fue 6.8.
    \end{itemize} %21

    \item \textbf{Respuesta C} Se busca la afirmación FALSA. Frecuencias del gráfico: 16, 12, 8, 22, 12. El total es $16+12+8+22+12 = 70$. La afirmación C dice que el total es 58. Por lo tanto, C es Falsa. La afirmación D también es falsa ($16 > 70/2$ es falso), indicando un error en la pregunta, pero C es una discrepancia directa con la suma total. %22

    \item \textbf{Respuesta E} No podemos deducir si al menos el 10\% tiene mas de 18 primos, ya que nos entregan que el intervalo es de [18,21[, y no sabemos su distribución.
    
    \item \textbf{Respuesta C} Por definición de percentiles, el percentil $k$ ($P_k$) es el valor bajo el cual se encuentra el $k\%$ de los datos. La mediana ($m$) es el percentil 50 ($P_{50}$). Como $15 < 50$, el valor del percentil 15 siempre será menor o igual al valor del percentil 50. Por lo tanto, $P_{15} \leq m$ es siempre verdadero. %24
    
    \item \textbf{Respuesta A} De la tabla, la frecuencia acumulada porcentual nos permite definir las frecuencias relativas: A=25\%, B=87\%-25\%=62\%, C=100\%-87\%=13\%. En términos de frecuencia total F, tenemos: $D=0.25F$, $E=0.87F$ y $F=F$. La relación A dice $F > D+E$, lo que se traduce en $F > 0.25F + 0.87F$ o $F > 1.12F$. Esto es falso para cualquier frecuencia $F > 0$. %25
    
    \item \textbf{Respuesta E} I) Promedio Mujeres: $\frac{0.9+1.2+1+0.4+0.5}{5}=0.8$. Promedio Hombres: $\frac{1.2-0.5+1.3+1.5+0.5}{5}=0.8$. Son iguales. (Verdadero). II) Mediana Mujeres (ordenado: 0.4, 0.5, \textbf{0.9}, 1, 1.2) es 0.9. Mediana Hombres (ordenado: -0.5, 0.5, \textbf{1.2}, 1.3, 1.5) es 1.2. La de las mujeres está por debajo. (Verdadero). III) El rango de los hombres ($2.0$) es mayor que el de las mujeres ($0.8$), y sus datos están más dispersos, por lo que su desviación estándar es mayor. (Verdadero). %26
    
    \item \textbf{Respuesta A} (1) por sí sola: Se igualan las medias: $\frac{4p+160}{p+30} = \frac{5p+150}{p+30}$. Esto lleva a $4p+160=5p+150$, de donde $p=10$. Es suficiente. (2) por sí sola: Se igualan las medianas. Para ambos cursos la mediana es 5 si $0 < p < 30$. Esto no determina un valor único para $p$. No es suficiente. %27
    
    \item \textbf{Respuesta C} Por definición, los percentiles dividen los datos ordenados. $P$ es el percentil 45 ($P_{45}$). La mediana es $P_{50}$ y el tercer cuartil es $P_{75}$. Por orden, siempre se cumplirá que $P_{45} \leq P_{50} \leq P_{75}$. Por lo tanto, se puede deducir que $P$ es menor o igual al tercer cuartil. Las opciones con desigualdad estricta ("mayor que", "menor que") son imprecisas, pero la relación de orden de la opción C es la correcta. %28
    
    \item \textbf{Respuesta A} El primer cuartil ($Q_1=13$) indica que el 25\% de las personas tiene 13 años o menos. El tercer cuartil ($Q_3=18$) indica que el 25\% de las personas tiene 18 años o más. Juntos, estos dos grupos (los menores de 13 y los mayores de 18) representan el $25\% + 25\% = 50\%$ del total del grupo. Las demás afirmaciones no se pueden deducir de un diagrama de caja. %29
    
    \item \textbf{Respuesta D} La afirmación "Por lo menos un estudiante consiguió nota 7" no se puede deducir. El intervalo "Muy Bueno" es $[6, 7]$, y contiene a 6 estudiantes. Sin embargo, es posible que los 6 estudiantes tuvieran notas como 6.0, 6.2, 6.5, etc., sin que ninguno alcanzara el 7.0. No tenemos certeza. Las demás opciones se pueden verificar directamente con los datos de la tabla. %30
    

    \item \textbf{Respuesta B} Ya que el promedio se quilibria en el centro, por lo cual debe ser simetrico el grafico, y como la moda es igual a la mediana y al promedio, este debe estar siosi en e centro.

    \item \textbf{Respuesta A} La encuesta fue a 300 estudiantes. A) Frecuencia relativa de "Muy de acuerdo" (30\%) es $\frac{30}{100} = \frac{3}{10}$. (Verdadero). B) Frec("Ni de acuerdo...") = $0.37 \cdot 300 = 111$. Frec("Algo de acuerdo") = $0.29 \cdot 300 = 87$. La diferencia es 24, no 8. (Falso). C) La moda es "Ni de acuerdo..." (37\%), es unimodal. (Falso). D) Frec("No contesta") = $0.02 \cdot 300 = 6$. (Falso). %32
    
    \item \textbf{Respuesta D} Del primer dato (8 personas son el 16\%), se deduce que el total es $N = 8/0.16 = 50$ personas. Con esto, se completan las frecuencias relativas de los otros intervalos: $[18,24[$ es 28\%, $[24,30[$ es 32\%. La afirmación D dice que el 38\% tiene menos de 30 años. El porcentaje real es la suma de los tres primeros intervalos: $16\%+28\%+32\%=76\%$. Por lo tanto, la afirmación D es falsa. %33
    
    \item \textbf{Respuesta E} Las frecuencias de cada intervalo se obtienen restando los valores de la frecuencia acumulada en los extremos del intervalo. $R = 10-0 = 10$. $S = 20-10 = 10$. $T = 25-20 = 5$. $Q = 28-25 = 3$. Los valores son R=10, S=10, T=5, Q=3. %34

    \item \textbf{Respuesta B} La pregunta es sobre el promedio de "gastos por viaje". En la tabla se detallan los gastos para 4 viajes a 4 localidades distintas (San Antonio, Valparaíso, Rancagua, Litueche). Para calcular el gasto promedio por viaje, se debe sumar el total de todos los gastos y dividirlo por el número total de viajes, que es 4. %35

    \item \textbf{Respuesta C} Promedio de Rodrigo (P): $\frac{5.2+4.8+5.0+6.0+4.0}{5} = 5.0$. Promedio de Mariel (Q): $\frac{5.8+5.2+4.0+4.5+5.5}{5} = 5.0$. Por lo tanto, $P=Q$. Mediana de Rodrigo (R): datos ordenados 4.0, 4.8, \textbf{5.0}, 5.2, 6.0 $\rightarrow R=5.0$. Mediana de Mariel (S): datos ordenados 4.0, 4.5, \textbf{5.2}, 5.5, 5.8 $\rightarrow S=5.2$. Por lo tanto, $R<S$. La relación correcta es $P=Q$ y $R<S$. %36
    
    \item \textbf{Respuesta E} Con la fila 2 (frec=5, frec. rel.=0.2) se halla el total de trabajadores: $N=5/0.2 = 25$. II) "menos de 3 días" es el intervalo $[0, 3[$, cuya frecuencia es 15. La frecuencia relativa es $15/25=0.6=60\%$. (Verdadero). III) "menos de 6 días" son los intervalos $[0, 3[$ y $[3, 6[$. La suma de frecuencias es $15+5=20$. (Verdadero). I) es falsa, ya que 25 es el número de trabajadores, no de ausencias. %37
    
    \item \textbf{Respuesta E} Total de datos N=200. I) Posición de $Q_1$: $200 \cdot 1/4 = 50$. Frec. Acumulada: 10, 28, 41, 60. El dato 50 está en el grupo de suma 5. (Verdadero). II) Posición 3er quintil: $200 \cdot 3/5 = 120$. Frec. Acumulada: ...86, 110, 135. El dato 120 está en el grupo de suma 8. (Verdadero). III) Posición $P_{54}$: $200 \cdot 54/100 = 108$. Frec. Acumulada: ...86, 110. El dato 108 está en el grupo de suma 7. (Verdadero). %38
    
    \item \textbf{Respuesta E} Que la media, mediana y moda sean iguales describe una distribución unimodal y simétrica, pero no implica que los datos deban ser todos iguales (contraejemplo: \{1, 2, 2, 3\}). Por lo tanto, la desviación estándar no tiene por qué ser 0, y el grupo puede tener más de un dato. Ninguna de las afirmaciones es siempre verdadera. %39
    
    \item \textbf{Respuesta A} Media: Muestra A tiene más peso en el valor $q$ y menos en $p$ que la muestra B. Como $p<q$, la media de A será mayor: $m>n$. Mediana: Ambas muestras tienen 12 datos, la mediana es el promedio del 6º y 7º dato. En la muestra A, la frec. acumulada es 3 (para p) y 8 (para q), por lo que el 6º y 7º dato son $q$. Mediana $s=q$. En la muestra B, la frec. acumulada es 5 (para p) y 8 (para q), por lo que el 6º y 7º dato también son $q$. Mediana $t=q$. Entonces, $s=t$. La relación es $m>n, s=t$. %40
    
    \section{Respuestas 41-60}
    \item \textbf{Respuesta E} La información proporcionada corresponde a una muestra de 5 estudiantes, que es solo el 10\% del curso. No se puede deducir con certeza ninguna característica (promedio, mediana, distribución) de la población total de 50 estudiantes a partir de una muestra tan pequeña. Todas las afirmaciones son generalizaciones inválidas. %41

    \item \textbf{Respuesta B} I) Falso. Los intervalos $[Q_1, Q_2]$ y $[Q_2, Q_3]$ contienen, por definición, el 25\% de los datos cada uno. No se puede afirmar que uno contenga "la mayor cantidad". III) Falso. La media no se puede determinar solo con los cuartiles. II) Verdadero. El rango intercuartil, entre $Q_1=75$ kg y $Q_3=90$ kg, contiene exactamente al 50\% central de los estudiantes. La afirmación es una descripción correcta de este rango. %42
    
    \item \textbf{Respuesta A} El rango intercuartil (RIC o IQR) se calcula como la diferencia entre el tercer cuartil ($Q_3$) y el primer cuartil ($Q_1$). Según el diagrama de caja, $Q_3 = 750.000$ y $Q_1 = 500.000$. Por lo tanto, el RIC es $750.000 - 500.000 = 250.000$. Esta afirmación es siempre verdadera y se deduce directamente del gráfico. Las otras opciones confunden mediana con promedio o hacen suposiciones incorrectas sobre la distribución de los datos. %43
    
    \item \textbf{Respuesta E} Analizando la ojiva de frecuencia acumulada porcentual para 300 estudiantes: A) El intervalo modal tiene la mayor frecuencia. Las frecuencias por intervalo son: $[350,550[\to25\%$, $[550,650[\to24\%$, $[650,750[\to32\%$, $[750,850]\to19\%$. El modal es $[650,750[$. Falso. B) Menos de 650 puntos es el 49\% de 300, o sea 147 estudiantes. Falso. E) El punto $(550, 25)$ en el gráfico indica que el 25\% de los estudiantes obtiene 550 puntos o menos. Verdadero. %44
    
    \item \textbf{Respuesta C} Se busca la afirmación FALSA. A) El total de usuarios es $45+38+30+45+36+15 = 209$. Verdadero. B) La frecuencia más alta es 45, que corresponde a los intervalos $[0, 5[$ y $[15, 20[$. Verdadero. C) "A lo menos 20 minutos" corresponde a los intervalos $[20, 25[$ y $[25, 30]$. La suma de usuarios es $36 + 15 = 51$. La afirmación dice 158. Falso. %45

    \item \textbf{Respuesta D} Una propiedad de la desviación estándar es que si todos los datos de un conjunto se multiplican por una constante positiva $n$, la nueva desviación estándar es la desviación estándar original multiplicada por $n$. Si la desviación estándar de $\{a, b, c\}$ es $\sigma$, entonces la de $\{na, nb, nc\}$ es $n\sigma$. %46

    \item \textbf{Respuesta E} La desviación estándar ($\sigma$) es una medida de la dispersión de los datos. Es cero solo si todos los datos son iguales. Si los elementos de A son enteros positivos distintos, no son todos iguales, por lo que su dispersión es necesariamente mayor que cero, es decir, $\sigma > 0$. Las otras afirmaciones son falsas bajo ciertas condiciones (ej. $\sigma=1$ o $0 < \sigma < 1$). %47
    
    \item \textbf{Respuesta D} I) Verdadera. Es la definición de varianza cero. II) Falsa. Dos conjuntos pueden tener la misma media pero diferente dispersión (ej. $\{4,6\}$ y $\{0,10\}$). III) Verdadera. Si $|x_i - \mu| = 1$ para todos los datos, entonces la varianza, que es el promedio de $(x_i - \mu)^2$, será el promedio de $1^2$, que es 1. Por lo tanto, I y III son siempre verdaderas. %48
    
    \item \textbf{Respuesta E} Para comparar las desviaciones estándar $\sigma_A$ y $\sigma_B$, necesitamos información sobre la dispersión de los datos. (1) El rango es una medida de dispersión, pero no es suficiente. (2) Conocer la media de los cuadrados ($E[X^2]$) no es suficiente sin conocer la media de los datos ($E[X]$), ya que $\sigma^2 = E[X^2] - (E[X])^2$. Se requiere información adicional, como la media de cada conjunto, para poder comparar. %49

    \item \textbf{Respuesta B} Se tienen dos ecuaciones. De la media: $\frac{2+3+a+b}{4} = 4 \Rightarrow 5+a+b = 16 \Rightarrow a+b=11$. De la varianza: $\sigma^2 = E[X^2] - (E[X])^2$. Entonces, $2.5 = \frac{2^2+3^2+a^2+b^2}{4} - 4^2$. Resolviendo: $2.5 = \frac{13+a^2+b^2}{4} - 16 \Rightarrow 18.5 = \frac{13+a^2+b^2}{4} \Rightarrow 74 = 13+a^2+b^2 \Rightarrow a^2+b^2 = 61$. %50

    \item \textbf{Respuesta B} La tabla indica las frecuencias para cada cantidad de horas: 1 hora $\to$ 5 personas, 2 horas $\to$ 3 personas, 3 horas $\to$ 2 personas. El gráfico de barras de la opción B representa correctamente estas alturas para cada categoría. %51

    \item \textbf{Respuesta C} El histograma muestra las frecuencias para cada intervalo de sueldos. Se debe leer la altura (frecuencia) de cada barra: para el intervalo $[1, 2[$ la frecuencia es 7; para $[2, 3[$ es 5; para $[3, 4[$ es 2; y para $[4, 5[$ es 1. La tabla de la opción C es la única que representa correctamente estos intervalos con sus respectivas frecuencias. %52

    \item \textbf{Respuesta B} Primero, se ordenan los datos: $\{6, 12, 12, 14, 16\}$. A) El rango es $16-6=10$. Falso. B) El promedio es $\frac{6+12+12+14+16}{5} = \frac{60}{5} = 12$. Verdadero. C) La mediana (el valor central) es 12. Falso. D) La moda (el valor más repetido) es 12. Falso. %53

    \item \textbf{Respuesta C} Se deben sumar las ventas de la sexta semana a las ventas acumuladas y comparar los totales para las 6 semanas. P1: $120+80=200$. P2: $200+80=280$. P3: $200+100=300$. P4: $250+40=290$. El producto 3 tuvo el mayor total de ventas (300), por lo que también tuvo el mayor promedio de ventas semanal. %54

    \item \textbf{Respuesta B} A) Rango A = $25-10=15$. Rango B = $25-8=17$. Falso. B) Moda A (valor con mayor frecuencia) es 10. Moda B es 25. La afirmación dice que la moda de A es 15 años menor que la de B ($10 = 25 - 15$). Verdadero. C) Mediana A (promedio de datos 6 y 7) es 14. Mediana B (promedio de datos 5 y 6) es 17. Falso. %55

    \item \textbf{Respuesta B} Para calcular el promedio, se suman todos los datos y se divide por la cantidad de datos (12 familias). Suma = $4+1+1+0+3+2+2+3+0+1+1+6 = 24$. Promedio = $\frac{24}{12} = 2$. %56

    \item \textbf{Respuesta B} El subsidio es para el 60\% de menores ingresos de la \textbf{población del país}. Para asegurar que se puede optar al subsidio, una persona debe tener un ingreso que esté por debajo del percentil 60 de los ingresos del país. Si su ingreso es menor que el percentil 40 del país, entonces con toda seguridad es también menor que el percentil 60 del país. Las opciones sobre la comuna son irrelevantes. %57
    
    \item \textbf{Respuesta D} La diferencia entre la temperatura máxima y mínima corresponde al rango total de cada diagrama de caja (distancia del extremo del bigote superior al extremo del bigote inferior). Comparando visualmente la longitud total de cada diagrama, el de La Habana es el más corto, indicando la menor diferencia o rango de temperaturas. %58

    \item \textbf{Respuesta D} Los datos ordenados son: $\{75, 77, 84, 98, 101, 116, 129, 132, 145, 152, 163, 176\}$. Hay 12 datos. Para encontrar el percentil $k$, se calcula la posición $P_k = \frac{k}{100} \cdot N$. Para el percentil 40, la posición es $\frac{40}{100} \cdot 12 = 4.8$, lo que significa que tomamos el valor en la 5ta posición. El 5to dato es 101 g, que supera los 100 g. Los percentiles menores (10, 20, 30) corresponden a posiciones inferiores cuyos valores no superan los 100 g. %59
    
    \item \textbf{Respuesta B} Se debe estimar el "Índice de lluvia diaria" (milímetros / días) para cada mes. Junio tiene el pico más alto de lluvia total (aprox. 200 mm) y uno de los valores más altos de días de lluvia (aprox. 10.5). El cociente es aprox. $\frac{200}{10.5} \approx 19$. En Mayo es aprox. $\frac{155}{9.5} \approx 16.3$. En los otros meses, la relación es visiblemente menor. Por lo tanto, Junio presenta el mayor índice. %60

    
    \section{Respuestas 61-80}

    \item \textbf{Respuesta D} Se busca el argumento válido. A) Falso. No se puede justificar la diferencia de medianas por el nivel de exigencia. B) Falso. El rango intercuartil ($Q_3-Q_1$) para alemán es $70-30=40$, pero la justificación dada es la definición del rango semi-intercuartil. C) Falso. El máximo puntaje obtenido fue 100, pero esto no significa que la mayoría lo obtuvo. D) Verdadero. El rango intercuartil (RIC) mide la longitud de la "caja". Para japonés: $Q_3-Q_1=80-40=40$. Para alemán: $Q_3-Q_1=70-30=40$. Son iguales, y la justificación es la definición correcta de RIC. %61

    \item \textbf{Respuesta B} Grupo Niñas: Datos $\{3,4,5\}$ con la misma frecuencia. Promedio = $\frac{3\cdot10+4\cdot10+5\cdot10}{30}=4$. Grupo Niños: Datos $\{8,9,10\}$ con la misma frecuencia. Promedio = $\frac{8\cdot10+9\cdot10+10\cdot10}{30}=9$. Los promedios son distintos. La desviación estándar depende de la dispersión respecto a la media. Para las niñas, los datos se desvían en -1, 0, 1 de su media. Para los niños, los datos también se desvían en -1, 0, 1 de su media. Como la dispersión relativa es idéntica, la desviación estándar es la misma para ambos grupos. %62
    
    \item \textbf{Respuesta A} El promedio general se calcula como la suma de todos los puntajes dividida por el total de estudiantes. Total de estudiantes = $(x+2x)+(2y+y)=3x+3y=3(x+y)$. Suma de puntajes en Horario 1 = (Estudiantes H1) $\cdot$ (Promedio H1) = $(x+2y)p$. Suma de puntajes en Horario 2 = (Estudiantes H2) $\cdot$ (Promedio H2) = $(2x+y)q$. Promedio general = $\frac{(x+2y)p+(2x+y)q}{3(x+y)}$. %63
    
    \item \textbf{Respuesta B} (2) por sí sola. La mediana es el valor que divide la mitad superior de los datos de la mitad inferior. Si la mediana es 5.3, esto garantiza que al menos el 50\% de los estudiantes obtuvo una nota de 5.3 o superior. Como $5.3 > 5.2$, se puede asegurar que al menos el 50\% obtuvo una nota superior a 5.2. (1) Conocer el promedio no da certeza sobre la distribución de más del 50\% de las notas. %64
    
    \item \textbf{Respuesta A} El total de estudiantes es 25. Las frecuencias relativas nos permiten hallar las frecuencias absolutas: Frec(1)=$0.08 \cdot 25=2$. Frec(2)=$0.16 \cdot 25=4$. Frec(4)=$0.36 \cdot 25=9$. Frec(5)=$0.12 \cdot 25=3$. La suma de estas frecuencias es $2+4+9+3=18$. La frecuencia faltante (para 3 días) es $25-18=7$. Pero, hay un error en la pregunta, ya que Frec(1) debería ser 2 (está en la tabla). El total de estudiantes es 25, y la suma de frecuencias conocidas es 2+4+9=15. Faltan dos valores. Si asumimos que solo falta Frec(3), entonces $2+4+\text{Frec}(3)+9+\text{Frec}(5) = 25$. Usando las relativas: Frec(3)=$1-(0.08+0.16+0.36+0.12) = 0.28 \Rightarrow 0.28 \cdot 25=7$.  (Nota: la pregunta tiene valores inconsistentes, pero la intención apunta a calcular la frecuencia faltante). Si se asume que Frec(3) es el único valor faltante en la tabla, la suma de frecuencias conocidas (2+4+9) es 15. La frecuencia faltante para que el total sea 25 es $25-15=10$. %65

    \item \textbf{Respuesta C} Para saber cuántas familias viven en el edificio, simplemente se debe sumar la columna de frecuencias: $6+3+2+3+1=15$ familias en total. %66

    \item \textbf{Respuesta B} Primero, se calcula el promedio de las calificaciones actuales ($p$): $\bar{p}=\frac{5.0+4.0+4.0+3.0+4.0}{5}=4.0$. La relación de la nueva calificación ($N$) es $N=\frac{6}{5}p$. Como la transformación es lineal, el nuevo promedio ($\bar{N}$) será la transformación del promedio antiguo: $\bar{N} = \frac{6}{5}\bar{p} = \frac{6}{5} \cdot 4.0 = 4.8$. %67
    
    \item \textbf{Respuesta A} Hay 50 estudiantes en total. Se busca la afirmación verdadera. A) "Totalmente en desacuerdo" corresponde al 30\% del total. $30\%$ de 50 es 15 estudiantes. La cuarta parte de los estudiantes es $50/4 = 12.5$. Como $15 > 12.5$, la afirmación es verdadera. B) "En desacuerdo" es el 50\%, que es la mitad, no más de la mitad. Falso. C) Tres quintas partes es el 60\%. Falso. D) La octava parte es 12.5\%. "Ni de acuerdo ni en desacuerdo" es 8\%. Falso. %68
    
    \item \textbf{Respuesta C} Se debe contar la frecuencia de cada edad. Edad 12: 4 veces. Edad 13: 10 veces. Edad 14: 18 veces. Edad 15: 5 veces. La edad con la mayor frecuencia es 14, por lo que se debe comprar una mayor cantidad de regalos para los adolescentes de 14 años. %69

    \item \textbf{Respuesta A} El gráfico original muestra: 2024 (2\%), 2025 (3\%), 2026 (2\%), 2027 (3\%). El economista propone disminuir el porcentaje en un punto porcentual a partir del segundo año (2025). La nueva proyección será: 2024 (2\%, se mantiene), 2025 (3\%-1\%=2\%), 2026 (2\%-1\%=1\%), 2027 (3\%-1\%=2\%). El gráfico de la opción A representa esta nueva proyección. %70
    
    \item \textbf{Respuesta D} La fundación ayudará al 25\% de los hogares con menores ingresos. Para determinar el ingreso máximo que puede tener un hogar para ser beneficiado, se debe encontrar el valor que separa al 25\% más bajo del resto. Esta medida de posición es, por definición, el primer cuartil ($Q_1$) o el percentil 25. %71
    
    \item \textbf{Respuesta D} Hay 36 datos. Los que no lograron eximirse son los que obtuvieron menos de 90 puntos. Hay 25 de estos estudiantes (todos hasta el puntaje 85). Se debe crear un diagrama de caja para estos 25 datos. Mín=33, Máx=85. Mediana (posición 13): 63. $Q_1$ (posición 7): 43. $Q_3$ (posición 19): 73. El diagrama de caja de la opción D muestra estos valores: Min=33, $Q_1=43$, Med=63, $Q_3=73$, Max=85. %72

    \item \textbf{Respuesta C} El rango de $\{g_1, g_2, g_3, g_4, g_5\}$ es cero, lo que implica que todos los datos son iguales. Sea $g_i = k$ para todo $i$. El nuevo conjunto de datos es $\{k-2, k-2, k, k+1, k+3\}$. La media es $\bar{x}=\frac{5k}{5}=k$. La desviación estándar se calcula a partir de las desviaciones de cada dato respecto a la media: $\{-2, -2, 0, 1, 3\}$. La varianza es $\sigma^2 = \frac{(-2)^2+(-2)^2+0^2+1^2+3^2}{5} = \frac{4+4+0+1+9}{5} = \frac{18}{5}$. La desviación estándar es $\sigma = \sqrt{\frac{18}{5}} = \frac{\sqrt{18}}{\sqrt{5}} = \frac{3\sqrt{2}}{\sqrt{5}}$. %73
    
    \item \textbf{Respuesta C} Los tramos son: Tramo 1 (0-40\%), Tramo 2 (40-50\%), Tramo 3 (50-60%), etc. (1) "Sobre el percentil 10" lo ubica en cualquier tramo del 1 al 7. Insuficiente. (2) "Debajo del cuartil 1 ($P_{25}$)" lo ubica en el Tramo 1. Insuficiente. Ambas juntas: (1) nos dice que está por encima de $P_{10}$ y (2) por debajo de $P_{25}$. Esto significa que el hogar se encuentra en el Tramo 1 (entre el 10\% y el 25\% de menores ingresos). Se puede conocer el tramo. %74

    \item \textbf{Respuesta B} La respuesta mayoritaria corresponde al dato con la mayor frecuencia (la moda). En el gráfico de barras, la altura de cada barra representa la frecuencia. La barra más alta corresponde a la cantidad de "3 plantas", que tiene una frecuencia de 4. Este es el argumento correcto. %75
    
    \item \textbf{Respuesta A} La cantidad total de personas original es $h+200+t+50 = 250+h+t$. Si la frecuencia de todos los datos aumenta en un 20\%, esto equivale a multiplicar cada frecuencia por 1.2. El nuevo total de personas será la suma de las nuevas frecuencias: $1.2h + 1.2(200) + 1.2t + 1.2(50) = 1.2(h+200+t+50) = 1.2(250+h+t)$. Esto es $300 + 1.2h + 1.2t$. %76
    
    \item \textbf{Respuesta D} El gráfico muestra una tendencia clara de aumento en el porcentaje de la población que consume legumbres al menos una vez por semana, pasando de 47\% en 2018 a 90\% en 2020. Esta es una forma sostenida de aumento en el consumo durante los años estudiados. Las otras afirmaciones son incorrectas o no se pueden deducir. %77
    
    \item \textbf{Respuesta D} Analizando cada partido: 1º: Favor 1, Contra 1 (Empate). 2º: Favor 2, Contra 2 (Empate). 3º: Favor 3, Contra 1 (Victoria). 4º: Favor 1, Contra 3 (Derrota). D) Diferencia de goles: 1º (0), 2º (0), 3º (2), 4º (2). La mayor diferencia fue de 2 goles, que ocurrió en el tercer y cuarto partido. La afirmación es verdadera. A) Total goles = $1+1+2+2+3+1+1+3 = 14 \neq 7$. Falso. B) Ganó 1, perdió 1. Falso. C) Es verdadero que hizo goles, pero la opción D es más específica y precisa. %78
    
    \item \textbf{Respuesta D} Sea $x$ la nota de la tercera prueba. Para que el promedio de las tres pruebas sea al menos 5.9, se debe cumplir la inecuación: $\frac{5.3+5.9+x}{3} \geq 5.9$. Resolviendo: $11.2+x \geq 17.7 \Rightarrow x \geq 17.7 - 11.2 \Rightarrow x \geq 6.5$. La nota mínima que debe obtener es 6.5. %79
    
    \item \textbf{Respuesta D} El promedio $\bar{X}$ de los sueldos se calcula como la suma total de los sueldos pagados dividida por el número total de trabajadores. Suma de sueldos = $(n \cdot 800000) + (m \cdot 500000) + (w \cdot 1100000)$. Total de trabajadores = $n+m+w$. La fórmula correcta es $\bar{X} = \frac{800000n + 500000m + 1100000w}{n+m+w}$. %80



\section{Respuestas 81-100}

    \item \textbf{Respuesta C} El 25\% de las personas con los tiempos más altos (el grupo sobre el tercer cuartil, $Q_3=46$ s) aumentarán su tiempo en 15 s. Sus nuevos tiempos estarán en el rango $[46+15, 70+15]$, es decir, $[61, 85]$. Como este grupo está compuesto por el 25\% de 100 personas, al menos 25 personas durarán más de 60 segundos. %81
    
    \item \textbf{Respuesta A} Para encontrar el percentil 20 en una muestra de 150 personas, calculamos su posición: $150 \cdot (20/100) = 30$. Buscamos el intervalo que contiene al dato número 30. La frecuencia acumulada del primer intervalo $[25, 30[$ es 12. La del segundo intervalo $[30, 35[$ es $12 + 24 = 36$. Por lo tanto, el dato 30 se encuentra en el intervalo $[30, 35[$. % 82
    
    \item \textbf{Respuesta B} Para un total de 200 estudiantes, la posición del percentil 40 es $200 \cdot (40/100) = 80$. Según la tabla de frecuencia acumulada, el estudiante en la posición 80 obtuvo un puntaje de 30. Los estudiantes que están "bajo el percentil 40" son aquellos con puntajes menores que 30. Estos son los que obtuvieron 10 o 20 puntos. La frecuencia acumulada hasta el puntaje 20 es de 75. Por lo tanto, 75 estudiantes pueden optar a la prueba. % 83

    \item \textbf{Respuesta D} La desviación estándar mide la dispersión de los datos. El curso A tiene una desviación estándar ($0,2$) menor que la del curso B ($0,4$). Esto significa que las notas del curso A están más concentradas alrededor del promedio (son más homogéneas) que las del curso B. Por lo tanto, la afirmación de que el curso A presenta menor dispersión es siempre verdadera. % 84

    \item \textbf{Respuesta C} La suma de las desviaciones respecto a la media ($x_i - \bar{x}$) es siempre cero. Así, $(x_1-\bar{x})+(x_2-\bar{x})+(x_3-\bar{x})+(x_4-\bar{x}) = 0$. Reemplazando los valores: $1 + (-2) + 4 + (x_4 - \bar{x}) = 0$, lo que nos da $(x_4 - \bar{x}) = -3$. La varianza es el promedio de los cuadrados de estas desviaciones: $\sigma^2 = \frac{(1)^2+(-2)^2+(4)^2+(-3)^2}{4} = \frac{1+4+16+9}{4} = \frac{30}{4} = \frac{15}{2}$. % 85
    
    \item \textbf{Respuesta C} Primero, se calcula el promedio ($\bar{x}$) y la desviación estándar ($\sigma$) de los puntajes (6, 8, 9, 12, 15). El promedio es 10. La varianza ($\sigma^2$) es 10, por lo que la desviación estándar ($\sigma$) es $\sqrt{10} \approx 3.16$. El intervalo de aceptación es $[10-\sqrt{10}, 10+\sqrt{10}]$, aproximadamente $[6.84, 13.16]$. Se eliminan los puntajes 6 y 15 por estar fuera del intervalo. El nuevo promedio se calcula con los puntajes restantes (8, 9, 12): $\frac{8+9+12}{3} = \frac{29}{3}$. %86
    
    \item \textbf{Respuesta B} Los datos del grupo B se obtienen sumando una constante (1000) a cada dato del grupo A. Una propiedad fundamental de la varianza es que es invariante a la traslación, es decir, no cambia si se suma una constante a todos los datos. La dispersión es la misma, solo se desplaza el conjunto de datos. Por lo tanto, $\text{Var}(A) = \text{Var}(B)$. La opción B lo justifica correctamente. %87
    
    \item \textbf{Respuesta D} La desviación estándar es cero solo si todos los datos de un conjunto son iguales. Como se nos dice que $p < q < r$, los valores en el Grupo 1 (p, q, r) no son todos iguales, por lo que su desviación estándar es mayor que cero. De igual manera, los valores en el Grupo 2 ($p-1$, q, $r+1$) tampoco son iguales, por lo que su desviación estándar también es mayor que cero. %88
    
    \item \textbf{Respuesta C} Con los datos de la fila "1 hijo" (frecuencia 2, frecuencia relativa 0,2), calculamos el total de familias ($N$): $N = 2 / 0,2 = 10$. Con $N=10$, calculamos las frecuencias absolutas: $\text{Frec}(0) = 10 \cdot 0,1 = 1$; $\text{Frec}(2) = 10 \cdot 0,3 = 3$; $\text{Frec}(3) = 10 \cdot 0,4 = 4$. El conjunto de datos es un '0', dos '1', tres '2' y cuatro '3', lo que corresponde a la opción C. %89
    
    \item \textbf{Respuesta B} El total de niños es 19. Del gráfico, las frecuencias son 1, $m$, 4, $n$. Entonces, $1+m+4+n=19$, lo que simplifica a $m+n=14$. La diferencia entre los niños de 4 años ($n$) y 2 años ($m$) es a lo más 2, es decir, $n-m \leq 2$. Con $n > m$ y $m+n=14$, la única solución entera es $m=6$ y $n=8$, que satisface $8-6=2 \leq 2$. %90

    \item \textbf{Respuesta A} Se evalúa la afirmación A: El número de países en 1995 era 112 y en 2006 era 150. El incremento absoluto es $150 - 112 = 38$. El incremento porcentual es $(\frac{38}{112}) \cdot 100 \approx 33.92\%$. La afirmación es correcta. Las otras opciones son incorrectas al verificar los datos del gráfico. %91
    
    \item \textbf{Respuesta B} Para obtener las frecuencias de cada intervalo (barras del histograma), se restan las frecuencias acumuladas consecutivas. Frec$[0,5[ = 40$. Frec$[5,10[ = 70-40=30$. Frec$[10,15[ = 84-70=14$. Frec$[15,20[ = 92-84=8$. Frec$[20,25[ = 98-92=6$. Frec$[25,30] = 100-98=2$. El histograma de la opción B representa correctamente estas frecuencias. %92
    
    \item \textbf{Respuesta A} El total de datos es $N = 8+x+9+12 = 29+x$. La mediana es 5. Si $x=14$, $N=43$ (impar) y la posición de la mediana es $(43+1)/2 = 22$. La frecuencia acumulada hasta la nota 4 es 8 y hasta la nota 5 es $8+14=22$. Por lo tanto, el dato en la posición 22 es un 5, y la mediana es 5. Para otros valores, como $x=13$, la mediana sería $5.5$. Por lo tanto, $x=14$ es un valor posible. %93

    \item \textbf{Respuesta D} Cuando todos los datos se multiplican por una constante $k$, la media y la mediana también se multiplican por $k$. La mediana original es $m$ y la nueva es $r$, así que $r = k \cdot m$, de donde $k = \frac{r}{m}$. La media original es $p$, por lo que la nueva media será $k \cdot p$. Sustituyendo $k$, la nueva media es $\frac{r}{m} \cdot p$. %94

    \item \textbf{Respuesta C} El promedio se calcula como $\frac{\sum(\text{valor} \cdot \text{frecuencia})}{\sum(\text{frecuencias})}$. El numerador es $(0\cdot5 + 1\cdot0 + 2\cdot7 + 3\cdot6 + 4\cdot5 + 5\cdot7) = 87$. El denominador es la suma de las frecuencias $5+0+7+6+5+7 = 30$. El promedio es $\frac{87}{30}$. La opción C muestra la estructura de cálculo correcta. %95
    
    \item \textbf{Respuesta C} El total de datos es $N=6+4+5+5=20$. Se busca la posición del percentil 80: $0.80 \cdot 20 = 16$. Debemos encontrar el intervalo que contiene al dato número 16. La frecuencia acumulada hasta el intervalo $[24, 26[$ es $6+4+5=15$. La frecuencia acumulada del siguiente intervalo $[26, 28]$ es 20. Por lo tanto, el dato 16 se encuentra dentro del intervalo $[26, 28]$. %96
    
    \item \textbf{Respuesta D} Por definición, el tercer cuartil ($Q_3$) de un conjunto de datos es el valor por debajo del cual se encuentra el 75\% de los datos. En el diagrama de caja del Colegio Educa, $Q_3$ es 5,9. Por lo tanto, es siempre verdadero que el 75\% de los estudiantes de ese colegio tiene un promedio anual menor o igual a 5,9. %97
    
    \item \textbf{Respuesta C} El total de personas es 200. La posición del tercer cuartil ($Q_3$) es $\frac{3}{4} \cdot 200 = 150$. Se nos dice que $Q_3=3$, por lo que el dato en la posición 150 debe ser 3. La frecuencia acumulada hasta 2 televisores es $48+72=120$. La frecuencia acumulada hasta 3 televisores es $120+h$. Para que el dato 150 esté en esta categoría, se debe cumplir que $120 + h \geq 150$, lo que implica $h \geq 30$. El valor mínimo para $h$ es 30. %98

    \item \textbf{Respuesta E} Primero, se calcula la frecuencia acumulada: $[6,10[ \rightarrow 5$; $[10,14[ \rightarrow 5+17=22$; $[14,18[ \rightarrow 22+8=30$; $[18,22[ \rightarrow 30+15=45$; $[22,26] \rightarrow 45+10=55$. II) La etiqueta $P$ en el gráfico corresponde a la frecuencia acumulada para el puntaje 14, que es 22. La afirmación es correcta. III) La etiqueta $Q$ está en el punto más alto de la ojiva, que representa la frecuencia acumulada total. El total de datos es 55, entonces $Q=55$. La afirmación es incorrecta. Por tanto, solo II es verdadera. (Nota: Hay un error en la pregunta original o las opciones, asumiendo que el gráfico es consistente con la tabla, la respuesta correcta sería solo II). %99
    
    \item \textbf{Respuesta D} El gráfico muestra la frecuencia relativa acumulada. Los puntos son $(5, 0.2)$, $(10, 0.7)$, $(15, 0.8)$ y $(20, 1.0)$. Como la muestra es de 100 personas, la frecuencia acumulada absoluta se obtiene multiplicando la relativa por 100. Las frecuencias acumuladas son: 20 (para el intervalo hasta 5), 70 (hasta 10), 80 (hasta 15) y 100 (hasta 20). La tabla de la opción D coincide con estos valores. %100


\end{enumerate}

\end{document}

