\documentclass[11pt]{article}
\usepackage{amsmath}  % Math
\usepackage{amssymb}  % Symbols
\usepackage{graphicx} % Images
\usepackage[utf8]{inputenc}
\usepackage[T1]{fontenc}
\usepackage[margin=1in]{geometry}

\title{Guía 5 Probabilidad y Estadistica}
\author{Felipe Colli Olea \thanks{Profesor: Sergio Díaz}}
\date{\today}

\begin{document}

\maketitle
\tableofcontents
\newpage

\section{Principios de Adicion y Multiplicación}
    \subsection*{¿De cuántas formas se puede cruzar un río una vez, si se cuenta con 1 bote y 2 barcos?}
    Se puede cruzar de 3 formas el rio, debido a que se apilca el principio aditivo de la combinatoria.

    \subsection*{¿De cuántas formas se puede vestir una persona que tiene 2 pantalones y 3 camisas?}
    Usando el principio Multiplicativo, se determina qye se peude vestir de 6 formas distintas, ya que por cada pantalon, peude usar 3 camisas.

    \subsection*{}


    \newpage

\section{Permutaciones y Combinaciones}

\end{document}
