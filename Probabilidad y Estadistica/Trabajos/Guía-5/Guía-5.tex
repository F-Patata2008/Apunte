\documentclass[11pt]{article}
\usepackage{amsmath}  % Paquetes para matemáticas
\usepackage{amssymb}  % Paquetes para símbolos
\usepackage{graphicx} % Paquetes para imágenes
\usepackage[utf8]{inputenc} % Codificación de caracteres
\usepackage[T1]{fontenc}    % Codificación de fuentes
\usepackage[spanish]{babel} % Soporte para el idioma español
\usepackage[margin=1in]{geometry} % Márgenes del documento

\graphicspath{ {./images/}} 



\title{Guía 05 de Probabilidad y Estadística: \\ Principios de Conteo, Permutaciones y Combinaciones}
\author{Felipe Colli Olea}
\date{\today}

\begin{document}

\maketitle
\begin{figure}[h!]
    \centering
    \includegraphics[width=0.75\textwidth]{gatito}
\end{figure}
\tableofcontents
\newpage

\section{Principios de Adición y Multiplicación}
    \subsection*{1. ¿De cuántas formas se puede cruzar un río una vez, si se cuenta con 1 bote y 2 barcos?}
    \textbf{Tipo:} Principio de Adición. \\
    Se puede cruzar de 3 formas el río. Al ser opciones excluyentes (o se cruza en bote o en barco), se aplica el principio aditivo: $1 + 2 = 3$ formas.

    \subsection*{2. ¿De cuántas formas se puede vestir una persona que tiene 2 pantalones y 3 camisas?}
    \textbf{Tipo:} Principio de Multiplicación. \\
    Usando el principio multiplicativo, se determina que se puede vestir de 6 formas distintas, ya que por cada pantalón, puede usar 3 camisas: $2 \times 3 = 6$ formas.

    \subsection*{3. ¿Cuántos resultados se pueden obtener si se lanza un dado 2 veces?}
    \textbf{Tipo:} Principio de Multiplicación (o Permutación con repetición). \\
    Cada lanzamiento del dado tiene 6 posibles resultados. Como los lanzamientos son independientes, el total de resultados es el producto de las posibilidades de cada lanzamiento: $6 \times 6 = 36$ resultados.

    \subsection*{4. ¿De cuántas formas se puede ordenar una pizza, si hay 2 opciones de masa (tradicional y especial), y 4 sabores? Solo se puede pedir una masa y un sabor.}
    \textbf{Tipo:} Principio de Multiplicación. \\
    Para cada una de las 2 opciones de masa, hay 4 opciones de sabor. Por el principio multiplicativo, el total de formas de ordenar es: $2 \times 4 = 8$ formas.

    \subsection*{5. ¿Cuántos resultados se pueden obtener si se lanza una moneda o un dado?}
    \textbf{Tipo:} Principio de Adición. \\
    Lanzar una moneda tiene 2 resultados. Lanzar un dado tiene 6 resultados. Como se realiza una acción O la otra, se aplica el principio aditivo: $2 + 6 = 8$ resultados posibles.

    \subsection*{6. a) ¿Cuántos resultados distintos se puede obtener si se lanza una moneda 3 veces? b) ¿Y si se lanza 5 veces?}
    \textbf{Tipo:} Principio de Multiplicación (Permutación con repetición).
    \begin{itemize}
        \item[a)] Cada lanzamiento tiene 2 resultados. Para 3 lanzamientos, el total es $2 \times 2 \times 2 = 2^3 = 8$ resultados.
        \item[b)] Para 5 lanzamientos, el total es $2^5 = 32$ resultados.
    \end{itemize}

    \subsection*{7. Un repuesto de automóvil se vende en 3 tiendas de Santiago y en 8 tiendas de Lima. ¿De cuántas formas se puede adquirir el repuesto?}
    \textbf{Tipo:} Principio de Adición. \\
    Se puede comprar en Santiago O en Lima. Se aplica el principio aditivo: $3 + 8 = 11$ formas de adquirir el repuesto.

    \subsection*{8. ¿De cuántas formas distintas puede cenar una persona si hay: 5 aperitivos, 3 entradas, 4 platos de fondo, 3 bebidas y 2 postres? Tener en cuenta que solo se puede elegir una opción de cada cosa.}
    \textbf{Tipo:} Principio de Multiplicación. \\
    Se elige una opción de cada categoría. Por el principio multiplicativo, el total de formas es: $5 \times 3 \times 4 \times 3 \times 2 = 360$ formas distintas de cenar.

    \subsection*{9. Una sala de lectura tiene 5 puertas: a) ¿de cuántas maneras puede entrar a la sala un estudiante y salir por una puerta diferente? b) ¿y si sale por cualquier puerta?}
    \begin{itemize}
        \item[a)] \textbf{Tipo:} Principio de Multiplicación (Permutación lineal sin repetición). \\
        Hay 5 opciones para entrar. Para salir por una puerta diferente, quedan 4 opciones. Total: $5 \times 4 = 20$ maneras.
        \item[b)] \textbf{Tipo:} Principio de Multiplicación (Permutación lineal con repetición). \\
        Hay 5 opciones para entrar y 5 opciones para salir. Total: $5 \times 5 = 25$ maneras.
    \end{itemize}

    \subsection*{10. De la ciudad A a la ciudad B, se puede ir mediante 2 buses o 3 trenes. De la ciudad B a la ciudad C se puede ir mediante 2 barcos, 2 trenes o 3 aviones. ¿De cuántas formas se puede ir de la ciudad A a la ciudad C, pasando por B?}
    \textbf{Tipo:} Principio de Adición y Principio de Multiplicación. \\
    Formas de ir de A a B (opciones excluyentes): $2 + 3 = 5$ formas.
    Formas de ir de B a C (opciones excluyentes): $2 + 2 + 3 = 7$ formas.
    Total de formas de A a C, pasando por B (principio multiplicativo): $5 \times 7 = 35$ formas.

    \subsection*{11. ¿Cuántos números de dos cifras pueden formarse con los dígitos: 1; 2; 3; 4 y 5, si: a) Si se pueden repetir los dígitos. b) No se pueden repetir los dígitos.}
    \begin{itemize}
        \item[a)] \textbf{Tipo:} Permutación con repetición. \\
        Con repetición: 5 opciones para el primer dígito y 5 para el segundo. Total: $5 \times 5 = 25$ números.
        \item[b)] \textbf{Tipo:} Permutación lineal sin repetición. \\
        Sin repetición: 5 opciones para el primer dígito y 4 para el segundo. Total: $5 \times 4 = 20$ números.
    \end{itemize}

    \subsection*{12. ¿Cuántos números de tres dígitos se pueden formar sin dígitos repetidos?}
    \textbf{Tipo:} Permutación lineal sin repetición (con restricciones). \\
    El primer dígito no puede ser 0 (9 opciones). El segundo puede ser cualquiera de los 9 dígitos restantes (incluido el 0). El tercero puede ser cualquiera de los 8 restantes. Total: $9 \times 9 \times 8 = 648$ números.

    \subsection*{13. ¿Cuántas placas diferentes de autos se pueden formar con 3 letras, seguidas de 4 números del 0 al 9? Considere que el alfabeto cuenta con 27 letras.}
    \textbf{Tipo:} Principio de Multiplicación y Permutación con repetición. \\
    Suponiendo que se permite la repetición. Para las letras: $27 \times 27 \times 27 = 27^3$. Para los números: $10 \times 10 \times 10 \times 10 = 10^4$.
    Total de placas: $27^3 \times 10^4 = 19,683 \times 10,000 = 196,830,000$.

    \subsection*{14. ¿Cuántos números pares de 3 cifras empiezan con 5 o 7?}
    \textbf{Tipo:} Principios de Adición y Multiplicación. \\
    Caso 1 (empiezan con 5): Primer dígito es 5 (1 opción). Para ser par, el último dígito debe ser 0, 2, 4, 6, 8 (5 opciones). El dígito del medio puede ser cualquiera (10 opciones). Total: $1 \times 10 \times 5 = 50$.
    Caso 2 (empiezan con 7): Mismo razonamiento. Total: $1 \times 10 \times 5 = 50$.
    Total general (principio aditivo): $50 + 50 = 100$ números.

    \subsection*{15. ¿De cuántas maneras diferentes podrá viajar una persona de A a E sin pasar ni regresar por el mismo camino?}
    \textbf{Tipo:} Conteo de rutas en un grafo. \\
    La respuesta proporcionada en la guía es 33. Este es un problema de conteo de rutas en un grafo. Se deben sumar todas las rutas posibles sin repetir nodos.

    \subsection*{16. ¿Cuántos números del 1 al 1000, no contienen la cifra 4?}
    \textbf{Tipo:} Principio de Multiplicación (con restricciones). \\
    Contamos los números de 3 dígitos (de 000 a 999) que no usan el 4. Los dígitos disponibles son \{0,1,2,3,5,6,7,8,9\} (9 opciones).
    Hay $9 \times 9 \times 9 = 729$ números entre 0 y 999 sin la cifra 4.
    Estos números son los del rango [0, 999]. La pregunta es para [1, 1000]. De los 729, excluimos el 0 y verificamos el 1000. El número 1000 no tiene la cifra 4, así que lo incluimos. El resultado es $729 - 1 (el \: 0) + 1 (el \: 1000) = 729$ números.

    \subsection*{17. ¿Cuántos números de 3 cifras empiezan con 5 u 8?}
    \textbf{Tipo:} Principios de Adición y Multiplicación. \\
    Caso 1 (empiezan con 5): Primer dígito es 5 (1 opción). Los otros dos pueden ser cualquiera de 0 a 9 (10 opciones cada uno). Total: $1 \times 10 \times 10 = 100$.
    Caso 2 (empiezan con 8): Mismo razonamiento. Total: $1 \times 10 \times 10 = 100$.
    Total general: $100 + 100 = 200$ números.

    \subsection*{18. Los números telefónicos de la ciudad de Lima son de ocho dígitos, de los cuales el primero tiene que ser 4 y el segundo no puede ser 0, 1 ni 7. ¿Cuántos números telefónicos diferentes se pueden formar?}
    \textbf{Tipo:} Principio de Multiplicación (con restricciones). \\
    Primer dígito: 1 opción (4).
    Segundo dígito: 7 opciones (10 - 3 excluidos).
    Dígitos 3 al 8 (6 dígitos): 10 opciones para cada uno ($10^6$).
    Total: $1 \times 7 \times 10^6 = 7,000,000$ números.

\newpage

\section{Permutaciones y Combinaciones}
    \subsection*{1. Carlos, Pedro y Sandra correrán los 100 metros planos. ¿De cuántas formas puede quedar el podio de primer y segundo lugar? Solo competirán ellos tres.}
    \textbf{Tipo:} Permutación lineal sin repetición. \\
    Dado que importa el orden (quién es primero y quién es segundo), es un problema de permutaciones. Se deben ordenar 2 personas de un total de 3.
    La fórmula es $P(n,k) = \frac{n!}{(n-k)!}$.
    $P(3,2) = \frac{3!}{(3-2)!} = \frac{3!}{1!} = 3 \times 2 \times 1 = 6$ formas.

    \subsection*{2. ¿De cuántas formas se puede preparar una ensalada de frutas con solo 2 ingredientes, si se cuenta con plátano, manzana y uva?}
    \textbf{Tipo:} Combinación. \\
    En una ensalada, el orden de los ingredientes no importa. Por lo tanto, es un problema de combinaciones. Se deben elegir 2 frutas de un total de 3.
    La fórmula es $C(n,k) = \frac{n!}{k!(n-k)!}$.
    $C(3,2) = \frac{3!}{2!(3-2)!} = \frac{3 \times 2 \times 1}{(2 \times 1)(1)} = 3$ formas.

    \subsection*{3. ¿De cuántas formas pueden hacer cola 5 amigos para entrar al cine?}
    \textbf{Tipo:} Permutación lineal sin repetición. \\
    El orden en una cola es importante. Es una permutación de 5 elementos.
    La fórmula es $P(n) = n!$.
    $P(5) = 5! = 5 \times 4 \times 3 \times 2 \times 1 = 120$ formas.

    \subsection*{4. ¿De cuántas formas puede un juez otorgar el primero, segundo y tercer premio en un concurso que tiene ocho concursantes?}
    \textbf{Tipo:} Permutación lineal sin repetición. \\
    El orden de los premios (primero, segundo, tercero) es diferente, por lo que es una permutación. Se deben ordenar 3 ganadores de 8 concursantes.
    $P(8,3) = \frac{8!}{(8-3)!} = \frac{8!}{5!} = 8 \times 7 \times 6 = 336$ formas.

    \subsection*{5. El capitán de un barco solicita 2 marineros para realizar un trabajo, sin embargo, se presentan 10. ¿De cuántas formas podrá seleccionar a los 2 marineros?}
    \textbf{Tipo:} Combinación. \\
    El orden en que se seleccionan los marineros para el trabajo no importa; el grupo es el mismo. Es una combinación de 10 elementos tomados de 2 en 2.
    $C(10,2) = \frac{10!}{2!(10-2)!} = \frac{10 \times 9}{2 \times 1} = 45$ formas.

    \subsection*{6. Eduardo tiene 7 libros, ¿de cuántas maneras puede acomodar cinco de ellos en un estante?}
    \textbf{Tipo:} Permutación lineal sin repetición. \\
    El orden de los libros en un estante sí importa. Es una permutación de 7 libros tomados de 5 en 5.
    $P(7,5) = \frac{7!}{(7-5)!} = \frac{7!}{2!} = 7 \times 6 \times 5 \times 4 \times 3 = 2520$ maneras.

    \subsection*{7. En un salón de 10 alumnos, ¿de cuántas maneras se puede formar un comité formado por 2 de ellos?}
    \textbf{Tipo:} Combinación. \\
    En un comité, los cargos no están definidos, por lo que el orden no importa. Es una combinación de 10 alumnos tomados de 2 en 2.
    $C(10,2) = \frac{10!}{2!(10-2)!} = \frac{10 \times 9}{2} = 45$ maneras.

    \subsection*{8. a) ¿Cuántas señales diferentes son posibles si las cuatro banderas son utilizadas? b) ¿Cuántas señales diferentes son posibles si al menos una bandera es utilizada?}
    \begin{itemize}
        \item[a)] \textbf{Tipo:} Permutación lineal sin repetición. \\
        Si se usan las 4 banderas, el orden en el asta importa. Es una permutación de 4 elementos. $P(4) = 4! = 24$ señales.
        \item[b)] \textbf{Tipo:} Suma de Permutaciones lineales sin repetición. \\
        "Al menos una bandera" significa que se pueden usar 1, 2, 3 o 4 banderas. Sumamos las permutaciones para cada caso:
        Señales con 1 bandera: $P(4,1) = 4$.
        Señales con 2 banderas: $P(4,2) = 4 \times 3 = 12$.
        Señales con 3 banderas: $P(4,3) = 4 \times 3 \times 2 = 24$.
        Señales con 4 banderas: $P(4,4) = 4! = 24$.
        Total: $4 + 12 + 24 + 24 = 64$ señales.
    \end{itemize}

    \subsection*{9. Un club de vóley tiene 12 jugadoras, una de ellas es la capitana María. ¿Cuántos equipos diferentes de 6 jugadoras se pueden formar, sabiendo que en todos ellos siempre estará la capitana María?}
    \textbf{Tipo:} Combinación. \\
    El equipo debe tener 6 jugadoras. Como María siempre está en el equipo, ya tenemos 1 jugadora fija. Solo necesitamos elegir las 5 jugadoras restantes de las 11 que quedan. El orden no importa, es una combinación.
    $C(11,5) = \frac{11!}{5!(11-5)!} = \frac{11 \times 10 \times 9 \times 8 \times 7}{5 \times 4 \times 3 \times 2 \times 1} = 462$ equipos.

    \subsection*{10. Con 4 frutas diferentes, ¿cuántos jugos surtidos se pueden preparar? *Un jugo surtido se prepara con 2 frutas al menos.}
    \textbf{Tipo:} Suma de Combinaciones. \\
    "Al menos 2 frutas" significa que el jugo puede tener 2, 3 o 4 frutas. Sumamos las combinaciones para cada caso:
    Jugos con 2 frutas: $C(4,2) = \frac{4!}{2!2!} = 6$.
    Jugos con 3 frutas: $C(4,3) = \frac{4!}{3!1!} = 4$.
    Jugos con 4 frutas: $C(4,4) = \frac{4!}{4!0!} = 1$.
    Total: $6 + 4 + 1 = 11$ jugos.
    
    \subsection*{11. a) ¿De cuántas maneras pueden posar tres hombres y dos mujeres en línea? b) ¿Y si una mujer debe estar en cada extremo? c) ¿Y si las personas del mismo sexo están juntas? d) ¿Y si las mujeres están separadas?}
    \begin{itemize}
        \item[a)] \textbf{Tipo:} Permutación lineal sin repetición. \\
        Hay un total de 5 personas. Ponerlas en línea es una permutación de 5. $5! = 120$ maneras.
        \item[b)] \textbf{Tipo:} Permutación lineal sin repetición (con restricciones). \\
        Se fijan las posiciones de las mujeres en los extremos. Hay 2 mujeres para 2 puestos: $2!$ maneras. Los 3 hombres se permutan en los 3 puestos del medio: $3!$ maneras. Total: $2! \times 3! = 2 \times 6 = 12$ maneras.
        \item[c)] \textbf{Tipo:} Permutación de bloques y permutación interna. \\
        Se consideran a los hombres como un bloque (H) y a las mujeres como otro (M). Se ordenan los 2 bloques: $2!$. Dentro del bloque de hombres, ellos se ordenan: $3!$. Dentro del bloque de mujeres, ellas se ordenan: $2!$. Total: $2! \times 3! \times 2! = 2 \times 6 \times 2 = 24$ maneras.
        \item[d)] \textbf{Tipo:} Permutación lineal sin repetición (con restricciones). \\
        Para que las mujeres estén separadas, primero ordenamos a los hombres ($3!$). Esto crea 4 espacios (\_H\_H\_H\_) donde podemos colocar a las mujeres. Se eligen 2 de estos 4 espacios y se ordenan las mujeres en ellos ($P(4,2)$). Total: $3! \times P(4,2) = 6 \times \frac{4!}{2!} = 6 \times 12 = 72$ maneras.
    \end{itemize}

    \subsection*{12. ¿Cuántas palabras diferentes se pueden formar con las letras de la palabra REMEMBER?}
    \textbf{Tipo:} Permutación con repetición. \\
    Es una permutación con elementos repetidos. La palabra tiene 8 letras: R(2), E(3), M(2), B(1).
    Total de palabras: $\frac{8!}{2! \cdot 3! \cdot 2!} = \frac{40320}{2 \cdot 6 \cdot 2} = \frac{40320}{24} = 1680$ palabras.

    \subsection*{13. Un dado es tirado siete veces y el orden de los tiros es considerado. ¿De cuántas maneras pueden ocurrir dos números 2, tres 3, un 4 y un 5?}
    \textbf{Tipo:} Permutación con repetición. \\
    Se trata de ordenar una secuencia de 7 resultados donde hay repeticiones: \{2,2, 3,3,3, 4, 5\}. Es una permutación con repetición de 7 elementos.
    Total de maneras: $\frac{7!}{2! \cdot 3! \cdot 1! \cdot 1!} = \frac{5040}{2 \cdot 6} = 420$ maneras.

    \subsection*{14. Para la Copa Mundial de fútbol, hay 25 comentaristas, de los cuales sólo seis hablan español. ¿De cuántas maneras se pueden formar grupos de cuatro, con la condición de que por lo menos se integren dos que hablen español?}
    \textbf{Tipo:} Suma de Combinaciones. \\
    Hay 6 que hablan español (E) y 19 que no (NE). Se necesita un grupo de 4 con "al menos 2 E". Sumamos los casos:
    \begin{itemize}
        \item 2 E y 2 NE: $C(6,2) \times C(19,2) = 15 \times 171 = 2565$.
        \item 3 E y 1 NE: $C(6,3) \times C(19,1) = 20 \times 19 = 380$.
        \item 4 E y 0 NE: $C(6,4) \times C(19,0) = 15 \times 1 = 15$.
    \end{itemize}
    Total de maneras: $2565 + 380 + 15 = 2960$ maneras.

    \subsection*{15. ¿Cuántas palabras diferentes se pueden formar con las letras de la palabra AGARRAR?}
    \textbf{Tipo:} Permutación con repetición. \\
    Es una permutación con elementos repetidos. La palabra tiene 7 letras: A(3), R(3), G(1).
    Total de palabras: $\frac{7!}{3! \cdot 3!} = \frac{5040}{6 \cdot 6} = \frac{5040}{36} = 140$ palabras.

    \subsection*{16. ¿De cuántas formas se pueden sentar 6 amigos alrededor de una mesa circular?}
    \textbf{Tipo:} Permutación circular. \\
    Es una permutación circular de $n$ elementos, cuya fórmula es $(n-1)!$.
    Total de formas: $(6-1)! = 5! = 120$ formas.

    \subsection*{17. En un grupo de 6 amigos, hay una pareja de novios. ¿De cuántas maneras pueden sentarse alrededor de una fogata, si los novios deben sentarse siempre juntos?}
    \textbf{Tipo:} Permutación circular (con elementos juntos). \\
    Se considera a la pareja de novios como una sola unidad. Ahora tenemos 5 "elementos" (la pareja y los otros 4 amigos) para sentar en círculo. Esto es una permutación circular de 5 elementos: $(5-1)! = 4! = 24$.
    Además, la pareja puede ordenarse de $2! = 2$ maneras entre sí (novio-novia o novia-novio).
    Total de formas: $4! \times 2! = 24 \times 2 = 48$ formas.
    
    \subsection*{18. Se va a programar un torneo de ajedrez para los 10 integrantes de un club. ¿Cuántos partidos se deben programar si cada integrante jugará con cada uno de los demás sin partidos de revancha?}
    \textbf{Tipo:} Combinación. \\
    Cada partido es una selección de 2 jugadores de los 10 disponibles. El orden no importa (un partido entre A y B es el mismo que entre B y A). Es una combinación.
    Total de partidos: $C(10,2) = \frac{10!}{2!(10-2)!} = \frac{10 \times 9}{2} = 45$ partidos.

    \subsection*{19. Una empresa desea contratar 3 nuevos empleados, pero hay 8 candidatos, 6 de los cuales son hombres y 2 son mujeres. a) ¿De cuántas maneras diferentes se pueden elegir? b) ¿De cuántas maneras se puede elegir a un solo candidato hombre? c) ¿De cuántas maneras se puede elegir por lo menos a un candidato hombre?}
    \begin{itemize}
        \item[a)] \textbf{Tipo:} Combinación. \\
        Elegir 3 empleados de 8 candidatos, sin importar el orden: $C(8,3) = \frac{8 \cdot 7 \cdot 6}{3 \cdot 2 \cdot 1} = 56$ maneras.
        \item[b)] \textbf{Tipo:} Combinación (con condiciones). \\
        Elegir 1 solo hombre significa elegir 1 hombre de 6 Y 2 mujeres de 2: $C(6,1) \times C(2,2) = 6 \times 1 = 6$ maneras.
        \item[c)] \textbf{Tipo:} Combinación (con condiciones). \\
        "Al menos un hombre" es el total de maneras menos las maneras de no elegir ningún hombre. La única forma de no elegir hombres es elegir 3 mujeres, pero solo hay 2. Por tanto, es imposible no elegir al menos un hombre. Todas las selecciones posibles tendrán al menos un hombre. Total: 56 maneras.
    \end{itemize}

    \subsection*{20. Se ha diseñado el siguiente logotipo para cierto producto, que debe ser pintado con 7 colores distintos. ¿De cuántas maneras se puede pintar con colores diferentes en cada circunferencia?}
    \textbf{Tipo:} Principio de Multiplicación y Permutación Circular. \\
    El logotipo tiene un círculo central y 6 círculos alrededor.
    \begin{enumerate}
        \item Se elige un color para el círculo central. Hay 7 opciones.
        \item Quedan 6 colores para los 6 círculos exteriores. Como están dispuestos en círculo, es una permutación circular de 6 elementos: $(6-1)! = 5! = 120$.
    \end{enumerate}
    Por el principio de multiplicación, el total de maneras es: $7 \times (6-1)! = 7 \times 120 = 840$ maneras.
    
\end{document}
