\documentclass[11pt]{article}
\usepackage{amsmath}  % Math
\usepackage{amssymb}  % Symbols
\usepackage{graphicx} % Images
\usepackage[utf8]{inputenc}
\usepackage[T1]{fontenc}
\usepackage[margin=1in]{geometry}

\title{Guía 5 Probabilidad y Estadistica}
\author{Felipe Colli Olea \thanks{Profesor: Sergio Díaz}}
\date{\today}

\begin{document}

\maketitle
\tableofcontents
\newpage

\section{Principios de Adición y Multiplicación}
    \subsection*{1. ¿De cuántas formas se puede cruzar un río una vez, si se cuenta con 1 bote y 2 barcos?}
    Se puede cruzar de 3 formas el río. Al ser opciones excluyentes (o se cruza en bote o en barco), se aplica el principio aditivo: $1 + 2 = 3$ formas.

    \subsection*{2. ¿De cuántas formas se puede vestir una persona que tiene 2 pantalones y 3 camisas?}
    Usando el principio multiplicativo, se determina que se puede vestir de 6 formas distintas, ya que por cada pantalón, puede usar 3 camisas: $2 \times 3 = 6$ formas.

    \subsection*{3. ¿Cuántos resultados se pueden obtener si se lanza un dado 2 veces?}
    Cada lanzamiento del dado tiene 6 posibles resultados. Como los lanzamientos son independientes, el total de resultados es el producto de las posibilidades de cada lanzamiento: $6 \times 6 = 36$ resultados.

    \subsection*{4. ¿De cuántas formas se puede ordenar una pizza, si hay 2 opciones de masa (tradicional y especial), y 4 sabores? Solo se puede pedir una masa y un sabor.}
    Para cada una de las 2 opciones de masa, hay 4 opciones de sabor. Por el principio multiplicativo, el total de formas de ordenar es: $2 \times 4 = 8$ formas.

    \subsection*{5. ¿Cuántos resultados se pueden obtener si se lanza una moneda o un dado?}
    Lanzar una moneda tiene 2 resultados. Lanzar un dado tiene 6 resultados. Como se realiza una acción O la otra, se aplica el principio aditivo: $2 + 6 = 8$ resultados posibles.

    \subsection*{6. a) ¿Cuántos resultados distintos se puede obtener si se lanza una moneda 3 veces? b) ¿Y si se lanza 5 veces?}
    a) Cada lanzamiento tiene 2 resultados. Para 3 lanzamientos, el total es $2 \times 2 \times 2 = 2^3 = 8$ resultados.
    b) Para 5 lanzamientos, el total es $2^5 = 32$ resultados.

    \subsection*{7. Un repuesto de automóvil se vende en 3 tiendas de Santiago y en 8 tiendas de Lima. ¿De cuántas formas se puede adquirir el repuesto?}
    Se puede comprar en Santiago O en Lima. Se aplica el principio aditivo: $3 + 8 = 11$ formas de adquirir el repuesto.

    \subsection*{8. ¿De cuántas formas distintas puede cenar una persona si hay: 5 aperitivos, 3 entradas, 4 platos de fondo, 3 bebidas y 2 postres? Tener en cuenta que solo se puede elegir una opción de cada cosa.}
    Se elige una opción de cada categoría. Por el principio multiplicativo, el total de formas es: $5 \times 3 \times 4 \times 3 \times 2 = 360$ formas distintas de cenar.

    \subsection*{9. Una sala de lectura tiene 5 puertas: a) ¿de cuántas maneras puede entrar a la sala un estudiante y salir por una puerta diferente? b) ¿y si sale por cualquier puerta?}
    a) Hay 5 opciones para entrar. Para salir por una puerta diferente, quedan 4 opciones. Total: $5 \times 4 = 20$ maneras.
    b) Hay 5 opciones para entrar y 5 opciones para salir. Total: $5 \times 5 = 25$ maneras.

    \subsection*{10. De la ciudad A a la ciudad B, se puede ir mediante 2 buses o 3 trenes. De la ciudad B a la ciudad C se puede ir mediante 2 barcos, 2 trenes o 3 aviones. ¿De cuántas formas se puede ir de la ciudad A a la ciudad C, pasando por B?}
    Formas de ir de A a B (opciones excluyentes): $2 + 3 = 5$ formas.
    Formas de ir de B a C (opciones excluyentes): $2 + 2 + 3 = 7$ formas.
    Total de formas de A a C, pasando por B (principio multiplicativo): $5 \times 7 = 35$ formas.

    \subsection*{11. ¿Cuántos números de dos cifras pueden formarse con los dígitos: 1; 2; 3; 4 y 5, si: a) Si se pueden repetir los dígitos. b) No se pueden repetir los dígitos.}
    a) Con repetición: 5 opciones para el primer dígito y 5 para el segundo. Total: $5 \times 5 = 25$ números.
    b) Sin repetición: 5 opciones para el primer dígito y 4 para el segundo. Total: $5 \times 4 = 20$ números.

    \subsection*{12. ¿Cuántos números de tres dígitos se pueden formar sin dígitos repetidos?}
    El primer dígito no puede ser 0 (9 opciones). El segundo puede ser cualquiera de los 9 dígitos restantes (incluido el 0). El tercero puede ser cualquiera de los 8 restantes. Total: $9 \times 9 \times 8 = 648$ números.

    \subsection*{13. ¿Cuántas placas diferentes de autos se pueden formar con 3 letras, seguidas de 4 números del 0 al 9? Considere que el alfabeto cuenta con 27 letras.}
    Suponiendo que se permite la repetición. Para las letras: $27 \times 27 \times 27 = 27^3$. Para los números: $10 \times 10 \times 10 \times 10 = 10^4$.
    Total de placas: $27^3 \times 10^4 = 19,683 \times 10,000 = 196,830,000$.

    \subsection*{14. ¿Cuántos números pares de 3 cifras empiezan con 5 o 7?}
    Caso 1 (empiezan con 5): Primer dígito es 5 (1 opción). Para ser par, el último dígito debe ser 0, 2, 4, 6, 8 (5 opciones). El dígito del medio puede ser cualquiera (10 opciones). Total: $1 \times 10 \times 5 = 50$.
    Caso 2 (empiezan con 7): Mismo razonamiento. Total: $1 \times 10 \times 5 = 50$.
    Total general (principio aditivo): $50 + 50 = 100$ números.

    \subsection*{15. ¿De cuántas maneras diferentes podrá viajar una persona de A a E sin pasar ni regresar por el mismo camino?}
    La respuesta proporcionada en la guía es 33. Este es un problema de conteo de rutas en un grafo. Se deben sumar todas las rutas posibles sin repetir nodos.

    \subsection*{16. ¿Cuántos números del 1 al 1000, no contienen la cifra 4?}
    Contamos los números de 3 dígitos (de 000 a 999) que no usan el 4. Los dígitos disponibles son \{0,1,2,3,5,6,7,8,9\} (9 opciones).
    Hay $9 \times 9 \times 9 = 729$ números entre 0 y 999 sin la cifra 4.
    Estos números son los del rango [0, 999]. La pregunta es para [1, 1000]. De los 729, excluimos el 0 y verificamos el 1000. El número 1000 no tiene la cifra 4, así que lo incluimos. El resultado es $729 - 1 (el \: 0) + 1 (el \: 1000) = 729$ números.

    \subsection*{17. ¿Cuántos números de 3 cifras empiezan con 5 u 8?}
    Caso 1 (empiezan con 5): Primer dígito es 5 (1 opción). Los otros dos pueden ser cualquiera de 0 a 9 (10 opciones cada uno). Total: $1 \times 10 \times 10 = 100$.
    Caso 2 (empiezan con 8): Mismo razonamiento. Total: $1 \times 10 \times 10 = 100$.
    Total general: $100 + 100 = 200$ números.

    \subsection*{18. Los números telefónicos de la ciudad de Lima son de ocho dígitos, de los cuales el primero tiene que ser 4 y el segundo no puede ser 0, 1 ni 7. ¿Cuántos números telefónicos diferentes se pueden formar?}
    Primer dígito: 1 opción (4).
    Segundo dígito: 7 opciones (10 - 3 excluidos).
    Dígitos 3 al 8 (6 dígitos): 10 opciones para cada uno ($10^6$).
    Total: $1 \times 7 \times 10^6 = 7,000,000$ números.

    \newpage

\section{Permutaciones y Combinaciones}
    \subsection*{1. Carlos, Pedro y Sandra correrán los 100 metros planos. ¿De cuántas formas puede quedar el podio de primer y segundo lugar? Solo competirán ellos tres.}
    Importa el orden (primero vs segundo). Es una permutación de 3 elementos tomados de 2 en 2. $P(3,2) = \frac{3!}{(3-2)!} = 3 \times 2 = 6$ formas.

    \subsection*{2. ¿De cuántas formas se puede preparar una ensalada de frutas con solo 2 ingredientes, si se cuenta con plátano, manzana y uva?}
    No importa el orden de los ingredientes. Es una combinación de 3 elementos tomados de 2 en 2. $C(3,2) = \frac{3!}{2!(3-2)!} = 3$ formas.

    \subsection*{3. ¿De cuántas formas pueden hacer cola 5 amigos para entrar al cine?}
    Es una permutación de 5 elementos, ya que importa el orden en la cola. $P(5) = 5! = 120$ formas.

    \subsection*{4. ¿De cuántas formas puede un juez otorgar el primero, segundo y tercer premio en un concurso que tiene ocho concursantes?}
    Importa el orden de los premios. Es una permutación de 8 elementos tomados de 3 en 3. $P(8,3) = \frac{8!}{(8-3)!} = 8 \times 7 \times 6 = 336$ formas.

    \subsection*{5. El capitán de un barco solicita 2 marineros para realizar un trabajo, sin embargo, se presentan 10. ¿De cuántas formas podrá seleccionar a los 2 marineros?}
    No importa el orden en que se seleccionan. Es una combinación de 10 elementos tomados de 2 en 2. $C(10,2) = \frac{10!}{2!(10-2)!} = \frac{10 \times 9}{2} = 45$ formas.

    \subsection*{6. Eduardo tiene 7 libros, ¿de cuántas maneras puede acomodar cinco de ellos en un estante?}
    Importa el orden de los libros en el estante. Es una permutación de 7 elementos tomados de 5 en 5. $P(7,5) = \frac{7!}{(7-5)!} = 7 \times 6 \times 5 \times 4 \times 3 = 2520$ maneras.

    \subsection*{7. En un salón de 10 alumnos, ¿de cuántas maneras se puede formar un comité formado por 2 de ellos?}
    No importa el orden en un comité. Es una combinación de 10 elementos tomados de 2 en 2. $C(10,2) = \frac{10!}{2!(10-2)!} = 45$ maneras.

    \subsection*{8. a) ¿Cuántas señales diferentes son posibles si las cuatro banderas son utilizadas? b) ¿Cuántas señales diferentes son posibles sí al menos una bandera es utilizada?}
    a) Usar las 4 banderas en un asta implica orden. Permutación de 4 elementos: $4! = 24$ señales.
    b) "Al menos una" es la suma de las señales con 1, 2, 3 o 4 banderas.
    1 bandera: $P(4,1) = 4$. 2 banderas: $P(4,2) = 12$. 3 banderas: $P(4,3) = 24$. 4 banderas: $P(4,4) = 24$.
    Total: $4 + 12 + 24 + 24 = 64$ señales.

    \subsection*{9. Un club de vóley tiene 12 jugadoras, una de ellas es la capitana María. ¿Cuántos equipos diferentes de 6 jugadoras se pueden formar, sabiendo que en todos ellos siempre estará la capitana María?}
    El equipo es de 6. Como María siempre está, solo necesitamos elegir las 5 jugadoras restantes de las 11 que quedan. No importa el orden. $C(11,5) = \frac{11!}{5!(11-5)!} = 462$ equipos.

    \subsection*{10. Con 4 frutas diferentes, ¿cuántos jugos surtidos se pueden preparar? *Un jugo surtido se prepara con 2 frutas al menos.}
    Sumamos las combinaciones de jugos de 2, 3 y 4 frutas.
    Con 2 frutas: $C(4,2)=6$. Con 3 frutas: $C(4,3)=4$. Con 4 frutas: $C(4,4)=1$.
    Total: $6+4+1=11$ jugos.
    
    \subsection*{11. a) 3 hombres y 2 mujeres posan en línea. b) Una mujer en cada extremo. c) Personas del mismo sexo juntas. d) Mujeres separadas.}
    a) Total 5 personas. Arreglarlas en línea: $5! = 120$ maneras.
    b) 2 mujeres en los extremos: $P(2,2)=2!$ maneras. 3 hombres en medio: $3!$ maneras. Total: $2! \times 3! = 2 \times 6 = 12$ maneras.
    c) Bloque de Hombres (HHH) y bloque de Mujeres (MM). Se arreglan los 2 bloques ($2!$) y las personas dentro de cada bloque ($3!$ y $2!$). Total: $2! \times 3! \times 2! = 2 \times 6 \times 2 = 24$ maneras.
    d) Colocar a los hombres primero (\_H\_H\_H\_), creando 4 espacios para las mujeres. Se eligen y ordenan 2 mujeres en esos 4 espacios $P(4,2)$. Los hombres se ordenan entre sí ($3!$). Total: $P(4,2) \times 3! = 12 \times 6 = 72$ maneras.

    \subsection*{12. ¿Cuántas palabras diferentes se pueden formar con las letras de la palabra REMEMBER?}
    Permutación con repetición. Palabra de 8 letras. R se repite 2 veces, E se repite 3 veces, M se repite 2 veces.
    Total: $\frac{8!}{2! \cdot 3! \cdot 2!} = \frac{40320}{2 \cdot 6 \cdot 2} = \frac{40320}{24} = 1680$ palabras.

    \subsection*{13. Un dado es tirado siete veces. ¿De cuántas maneras pueden ocurrir dos números 2, tres 3, un 4 y un 5?}
    Tenemos 7 tiros con resultados fijos: \{2,2,3,3,3,4,5\}. Se trata de ordenar estos 7 resultados. Es una permutación con repetición.
    Total: $\frac{7!}{2! \cdot 3!} = \frac{5040}{2 \cdot 6} = 420$ maneras.

    \subsection*{14. 25 comentaristas (6 hablan español), formar grupos de 4 con al menos 2 que hablen español.}
    Hay 6 que hablan español (S) y 19 que no (NS). Casos posibles:
    (2S, 2NS): $C(6,2) \times C(19,2) = 15 \times 171 = 2565$.
    (3S, 1NS): $C(6,3) \times C(19,1) = 20 \times 19 = 380$.
    (4S, 0NS): $C(6,4) \times C(19,0) = 15 \times 1 = 15$.
    Total: $2565 + 380 + 15 = 2960$ maneras.

    \subsection*{15. ¿Cuántas palabras diferentes se pueden formar con las letras de la palabra AGARRAR?}
    Permutación con repetición. Palabra de 7 letras. A se repite 3 veces, R se repite 3 veces.
    Total: $\frac{7!}{3! \cdot 3!} = \frac{5040}{6 \cdot 6} = 140$ palabras.

    \subsection*{16. ¿De cuántas formas se pueden sentar 6 amigos alrededor de una mesa circular?}
    Permutación circular de n elementos es $(n-1)!$.
    Total: $(6-1)! = 5! = 120$ formas.

    \subsection*{17. 6 amigos (una pareja de novios) se sientan alrededor de una fogata, si los novios deben sentarse siempre juntos.}
    Se considera a la pareja como un solo bloque. Se ordenan 5 "elementos" (el bloque y 4 amigos) en círculo: $(5-1)! = 4! = 24$.
    La pareja puede ordenarse de $2! = 2$ maneras dentro de su bloque.
    Total: $24 \times 2 = 48$ formas.
    
    \subsection*{18. Torneo de ajedrez con 10 integrantes, sin revancha. ¿Cuántos partidos se deben programar?}
    Cada partido es una pareja de 2 jugadores. El orden no importa. Es una combinación de 10 elementos tomados de 2 en 2.
    Total: $C(10,2) = \frac{10 \times 9}{2} = 45$ partidos.

    \subsection*{19. Contratar 3 empleados de 8 candidatos (6 hombres, 2 mujeres). a) ¿Maneras de elegir? b) ¿Elegir 1 solo hombre? c) ¿Elegir por lo menos 1 hombre?}
    a) Elegir 3 de 8: $C(8,3) = \frac{8 \cdot 7 \cdot 6}{3 \cdot 2 \cdot 1} = 56$ maneras.
    b) Elegir 1 hombre y 2 mujeres: $C(6,1) \times C(2,2) = 6 \times 1 = 6$ maneras.
    c) "Al menos un hombre" es el total menos "ningún hombre". Es imposible elegir 0 hombres, ya que se necesitan 3 empleados y solo hay 2 mujeres. Por lo tanto, cualquier selección tendrá al menos un hombre. El total es 56 maneras.

    \subsection*{20. Pintar un logotipo (1 círculo central, 6 alrededor) con 7 colores distintos.}
    Se elige un color para el círculo central (7 opciones).
    Los 6 colores restantes se usan para los 6 círculos exteriores. Como están en círculo, es una permutación circular de 6 elementos: $(6-1)! = 5! = 120$.
    Total: $7 \times 120 = 840$ maneras.
    
\end{document}
